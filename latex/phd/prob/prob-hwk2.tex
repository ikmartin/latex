\documentclass[hidelinks,11pt,dvipsnames]{article}
% xcolor commonly causes option clashes, this fixes that
\PassOptionsToPackage{dvipsnames,table}{xcolor}
\usepackage[tmargin=1in, bmargin=1in, lmargin=0.8in, rmargin=1in]{geometry}

%%%%%%%%%%%%%%%%%%%%%%%%%%%%%%%%%%%%%%%%%%%%%%%%%%%%%%%%%%%%%%%%%%%%
%%% For inkscape-figures
%%% Assumes the following directory structure:
%%% master.tex
%%% figures/
%%%     figure1.pdf_tex
%%%     figure1.svg
%%%     figure1.pdf
%%%%%%%%%%%%%%%%%%%%%%%%%%%%%%%%%%%%%%%%%%%%%%%%%%%%%%%%%%%%%%%%%%%%
%\usepackage{import}
\usepackage{pdfpages}
\usepackage{transparent}

\newcommand{\incfig}[2][1]{%
    \def\svgwidth{#1\columnwidth}
    \import{./figures/}{#2.pdf_tex}
}

\pdfsuppresswarningpagegroup=1

% enable synctex for inverse search, whatever synctex is
\synctex=1
\usepackage{float,macrosabound,homework,theorem-env}
\usepackage{microtype}


% font stuff
\usepackage{sectsty}
\allsectionsfont{\sffamily}
\linespread{1.1}

% bibtex stuff
\usepackage[backend=biber,style=alphabetic,sorting=anyt]{biblatex}
\addbibresource{main.bib}

% colored text shortcuts
\newcommand{\blue}[1]{\color{MidnightBlue}{#1}}
\newcommand{\red}[1]{\textcolor{Mahogany}{#1}}
\newcommand{\green}[1]{\textcolor{ForestGreen}{#1}}


% use mathptmx pkg while using default mathcal font
\DeclareMathAlphabet{\mathcal}{OMS}{cmsy}{m}{n}

% fixes the positioning of subscripts in $$ $$
\renewcommand{\det}{\operatorname{det}}

\usetikzlibrary{positioning, arrows.meta}
\newcommand{\here}[2]{\tikz[remember picture]{\node[inner sep=0](#2){#1}}}

%%%%%%%%%%%%%%%%%%%%%%%%%%%%%%%%%%%%%%%%%%%%%%%%%%%%%%%%%%%%%%%%%%%%%
%%% Entry Counter
%%%%%%%%%%%%%%%%%%%%%%%%%%%%%%%%%%%%%%%%%%%%%%%%%%%%%%%%%%%%%%%%%%%%%
\newcounter{entry-counter}
\newcommand{\entry}[1]
{
	\addtocounter{entry-counter}{1}
    \tchap{Entry \arabic{entry-counter}}
	%\addcontentsline{toc}{section}{Entry \arabic{entry-counter}: #1}
	\vspace{-1.5em}
    \begin{center}
		\small \emph{Written: #1}
    \end{center}
}

\usepackage{titling}
\renewcommand\maketitlehooka{\null\mbox{}\vfill}
\renewcommand\maketitlehookd{\vfill\null}


% for more leftrightarrow options
\usepackage{stackrel}
\newcommand\mapsfrom{\mathrel{\reflectbox{\ensuremath{\mapsto}}}}

\newcommand{\leftrarrows}{\mathrel{\raise.75ex\hbox{\oalign{%
  $\scriptstyle\leftarrow$\cr
  \vrule width0pt height.5ex$\hfil\scriptstyle\relbar$\cr}}}}
\newcommand{\lrightarrows}{\mathrel{\raise.75ex\hbox{\oalign{%
  $\scriptstyle\relbar$\hfil\cr
  $\scriptstyle\vrule width0pt height.5ex\smash\rightarrow$\cr}}}}
\newcommand{\Rrelbar}{\mathrel{\raise.75ex\hbox{\oalign{%
  $\scriptstyle\relbar$\cr
  \vrule width pt height.5ex$\scriptstyle\relbar$}}}}
\newcommand{\longleftrightarrows}{\leftrarrows\joinrel\Rrelbar\joinrel\lrightarrows}

\makeatletter
\def\leftrightarrowsfill@{\arrowfill@\leftrarrows\Rrelbar\lrightarrows}
\newcommand{\xleftrightarrows}[2][]{\ext@arrow 3399\leftrightarrowsfill@{#1}{#2}}
\makeatother
\begin{document}
\pagestyle{empty}
	\LARGE
\begin{center}
	Homework 2: Probability \\
	\Large
	Isaac Martin \\
    Last compiled \today
\end{center}
\normalsize
\vspace{-2mm}
\hru
\tchap{Chapter 1}
\begin{homework}[e]
	\prob[\textsc{Exercise 2.1}] (Stronger separation). Let $(S,\cS,\mu)$ be a measure space and let $f,g \in \cL^0(S,\cS)$ satisfy $\mu(\{x\in S ~:~ f(x) < g(x)\}) > 0$. Prove or construct a counterexample for the following statement:
	\begin{center}
		``There exist constants $a,b \in \bR$ such that $\mu(\{x \in S ~:~ f(x) \leq a < b \leq g(x)\}) > 0$.''
	\end{center}
	\begin{prf}
		We prove that this statement is true. Let $A = \{x \in S ~\mid~ f(x) < g(x)\}$ denote the set in questions, set $I = \{(a,b) \in \bQ^2 ~\mid~ a < b\}$ and for each pair of rational numbers $(a,b) \in I$, define
		\begin{align*}
			B_{a,b} = \left\{x \in S ~\mid~ f(x) \leq a < b \leq g(x)\right\}.
		\end{align*}
		Note that $B_{a,b}$ is countable for each $(a,b) \in I$ since both $f$ and $g$ are measurable functions and $B_{a,b} = f^{-1}((-\infty,a]) \cap g^{-1}([b,\infty))$. I claim that
		\begin{align*}
			A = \bigcup_{(a,b) \in I} B_{a,b} =: B
		\end{align*}
		Note first that because $B$ is the countable union of measurable sets, it too is measurable. Suppose that $x \in A$. Then $f(x) < g(x)$ and there exists some $a\in \bQ$ such that $f(x) < a < g(x)$ because $\bQ$ is dense in $\bR$. Similarly, there exists some $b \in \bQ$ such that $f(x) < a < b < g(x)$, and hence $x \in B_{a,b} \subseteq B$. Now suppose that $x \in B$. Then there exist some pair $(a,b) \in I$ such that $x \in B_{a,b}\implies f(x) \leq a < b \leq g(x)$. In particular, this means $f(x) < g(x)$ so $x \in A$. We therefore have that $A = B$.

		Now suppose that $\mu(B_{a,b}) = 0$ for each pair of rational numbers $(a,b) \in I$. We would then have that
		\begin{align*}
			\mu(A) = \mu(B) = \mu\left(\bigcup_{(a,b) \in I} B_{a,b}\right) \leq \sum_{(a,b) \in I} \mu(B_{a,b}) = \infty \cdot 0 = 0.
		\end{align*}
		Since $\mu(A) > 0$, it must be the case that $\mu(B_{a,b}) > 0$ for some $(a,b) \in I$.This proves the claim.
	\end{prf}
	\prob[\textsc{Exercise 2.2}] (A uniform distribution on a circle.) Let $S^1$ be the unit circle and let $f:[0,1) \to S^1$ be the ``winding map''
	\begin{align*}
		f(x) = \left(cos(2\pi x),\sin(2\pi x)\right), ~ x\in [0,1).
	\end{align*}
	\begin{enumerate}[(1)]
		\item Show that the map $f$ is $\big(\cB([0,1)), \cS^1\big)$-measurable, where $\cS^1$ denotes the Borel $\sigma$-algebra on $S^1$ (with topology inherited from $\bR^2$).
		\item For $\alpha \in (0,2\pi)$, let $R_\alpha$ denote the (counter-clockwise) rotation of $\bR^2$ with center $(0,0)$ and angle $\alpha$> Show that $R_\alpha(A) = \{R_\alpha(x) ~ : ~ x \in A\}$  is in $\cS^1$ if and only if $A \in \cS^1$.
		\item Let $\mu^1$ be the pushforward of the Lebesgue measure $\lambda$ by the map $f$. Show that $\mu^1$ is rotation-invariant, i.e. that $\mu^1(A) = \mu^1(R_\alpha(A))$. \small{\emph{Note:} The measure $\mu^1$ is called the \textbf{uniform measure} (or the \textbf{uniform distribution} on $S^1$)}.
	\end{enumerate}
	\begin{prf}$ $

		\noindent \emph{(1):} \hspace{0.5em} If this were a topology class, we'd simply state that ``it is clear that $f$ is continuous,'' as it is a continuous map in each component. Instead, we will prove that it is continuous, and hence Borel measurable. We take for granted the continuity of $\sin$ and $\cos$ as functions on $\bR$.

		Suppose $x,a \in [0,1)$, and consider $\|f(x) - f(a)\|^2$. With the help of trig identities, we have the following:
		\begin{align*}
			\|f(x) - f(a)\|^2 &= |(\cos(2\pi x)-\cos(2\pi a))^2 + (\sin(2\pi x) - \sin(2\pi a))^2| \\
							  &= |\cos^2(2\pi x) - 2\cos(2\pi x)\cos(2\pi a) + \cos^2(2\pi a) + \sin^2(2\pi x) \\ &\hspace{1em}- 2\sin(2\pi x)\sin(2\pi a) + \sin^2(2\pi a)| \\
							  &= |2 - \cos(2\pi x - 2\pi a) - \cos(2\pi x + 2\pi a) - \cos(2\pi x - 2\pi a) + \cos(2\pi x + 2\pi a)| \\
							  &= 2 - 2\cos(2\pi x - 2\pi a).
		\end{align*}
		Note that we may drop the absolute value in the final equality since $2\cos(2\pi x - 2\pi a) \leq 2$ for all $x,a\in [0,1)$. Thus, as $x$ approaches $a$ in $[0,1)$, we have that
		\begin{align*}
			\lim_{x \to a}\|f(x) - f(a)\| = \lim_{x\to a} \big(2 - 2\cos(2\pi x - 2\pi a) \big)= 2- 2\cos(0) = 0,
		\end{align*}
		and hence $f$ is continuous and therefore Borel measurable.

		\bigskip

		\noindent \emph{(2):} \hspace{0.5em} I claim that $R_\alpha$ is a homeomorphism on $\bR^2$, from which it will follow immediately that it induces a bijection on $\cS^1$. First, notice that rotation any point $x \in \bR^2$ first by $\alpha \in (0,2\pi)$ and then by $2\pi - \alpha$ gives back $x$, i.e. $R_{2\pi - \alpha} \circ R_\alpha = \id_{\bR^2}$. To see this more rigorously, we can realize $R_\alpha:\bR^2\to \bR^2$ as the $\bR$-linear map given by left multiplication by
		\begin{align*}
			\begin{pmatrix}	
				\cos(\alpha) & -\sin(\alpha) \\
				\sin(\alpha) & \cos(\alpha)
			\end{pmatrix},
		\end{align*}
		in which case the composition $R_\alpha$ with $R_{2\pi - \alpha}$ is the matrix product
		\begin{align*}
			\begin{pmatrix}	
				\cos(2\pi - \alpha) & -\sin(2\pi - \alpha) \\
				\sin(2\pi - \alpha) & \cos(2\pi - \alpha)
			\end{pmatrix}&
			\begin{pmatrix}	
				\cos(\alpha) & -\sin(\alpha) \\
				\sin(\alpha) & \cos(\alpha)
			\end{pmatrix} = 
			\begin{pmatrix}	
				\cos(\alpha) & \sin(\alpha) \\
				-\sin(\alpha) & \cos(\alpha)
			\end{pmatrix}
			\begin{pmatrix}	
				\cos(\alpha) & -\sin(\alpha) \\
				\sin(\alpha) & \cos(\alpha)
			\end{pmatrix} \\
			&\hspace{-5em}= 
			\begin{pmatrix}	
				\cos^2(\alpha) + \sin^2(\alpha) & -\sin(\alpha)\cos(\alpha) + \sin(\alpha)\cos(\alpha) \\
				-\sin(\alpha)\cos(\alpha) + \sin(\alpha)\cos(\alpha) & \sin^2(\alpha)+\cos^2(\alpha)
			\end{pmatrix} \\
			&\hspace{-5em}= 
			\begin{pmatrix}	1 & 0 \\ 0 & 1 \end{pmatrix}.
		\end{align*}
		We have a similar result for the composition $R_\alpha\circ R_{2\pi - \alpha}$. Since linear maps are continuous on $\bR^2$ (this is a fact from undergraduate analysis that I feel doesn't warrant proof) $R_\alpha$ is a continuous map with continuous inverse, and is hence a homeomorphism.

		Finally, note that $R_\alpha$ fixes $S^1$, which was implicitly assumed by the problem statement.

		Now suppose that $A \subseteq S^1$ is an open set. This means there must be some open set $U \subseteq \bR^2$ such that $A = S^1 \cap U$. Since $R_\alpha$ is a homeomorphism on $\bR^2$, $R_\alpha(U) = R^{-1}_{2\pi - \alpha}(U)$, which is open by the continuity of $R_{2\pi - \alpha}$. Since $R_\alpha$ fixes $S^1$,
		\begin{align*}
			R_\alpha(A) = R_\alpha(U \cap S^1) = R_\alpha(U) \cap S^1 = R^{-1}_{2\pi - \alpha}(U) \cap S^1,
		\end{align*}
		which is open in the subspace topology on $S^1$. Likewise, if $R_\alpha(A)$ is open, then $R^{-1}_{2\pi - \alpha}(R_\alpha(A)) = A$ is open.

		The Borel algebra on $S^1$ is generated by open sets, and since the maps $A \mapsto R_\alpha(A)$ and $R_\alpha(A) \mapsto A$ send open sets to open (and hence measurable) sets, by Proposition 1.10 in the notes we conclude that $R_\alpha$ induces a bisection on $\cS^1$.

		\bigskip

		\noindent \emph{(3):} \hspace{0.5em} Fix $\alpha \in (0,2\pi)$ and define a new measure $\mu^1_\alpha$ on $S^1$ by setting $\mu^1_\alpha(A) = \mu^1(R_\alpha(A))$. Note that this is actually the pullback measure $R_{2\pi-\alpha,*}\mu^1$, since by part (2) $R_\alpha(A) = R^{-1}_{2\pi - \alpha}(A)$, so $\mu^1_\alpha$ is indeed a measure on $S^1$. Let $\cP$ denote the set of all open arcs of $S^1$, or equivalently the collection of all open connected subsets of $S^1$. We prove that $\mu^1(A) = \mu^1_\alpha(A)$ for all $A\in \cP$.

		Let $A \subseteq \cP$ be an arc in $S^1$ and suppose that $(1,0)\not\in A$. Then $f^{-1}(A) = (a,b) \subseteq [0,1)$ for some $a,b \in \bR$, and hence
		\begin{align*}
			\mu_1(A) = \lambda(f^{-1}(A)) = \lambda((a,b)) = b - a.
		\end{align*}
		Now consider $\mu^1_\alpha$. First note that for $x \in [0,1)$ we have
		\begin{align*}
			R_\alpha(f(x)) &= 
			\begin{pmatrix}	
				\cos(\alpha) & -\sin(\alpha) \\
				\sin(\alpha) & \cos(\alpha)
			\end{pmatrix} \begin{pmatrix}	\cos(2\pi x) \\ \sin(2\pi x)\end{pmatrix} \\
							 &= 
			\begin{pmatrix}	
				\cos(\alpha)\cos(2\pi x) -\sin(\alpha)\sin(2\pi x) \\
				\sin(\alpha)\cos(2\pi x) + \cos(\alpha)\sin(2\pi x)
			\end{pmatrix} \\
							 &= 
			\begin{pmatrix}	
				\cos(\alpha + 2\pi x) \\
				\sin(\alpha + 2\pi x)
			\end{pmatrix} \\
							 &= 
							 \begin{cases}
								 f\left(\frac{\alpha}{2\pi} + x\right) & \text{if }~ \frac{\alpha}{2\pi} + x < 1 \\
								 f\left(\frac{\alpha}{2\pi}+ x - 1\right) &\text{otherwise}
							 \end{cases}.
		\end{align*}
		This means that if $(1,0) \not\in R_\alpha(A)$, then
		\begin{align*}
			\mu^1_\alpha(A) = \lambda\left(\left(\frac{\alpha}{2\pi} + a, \frac{\alpha}{2\pi} + b\right)\right) = b - a.
		\end{align*}
		If $(1,0) \in R_\alpha(A)$, then
		\begin{align*}
			\mu^1_\alpha(A) = \lambda\left(\left(\frac{\alpha}{2\pi} + a, 1\right)\cup [0,\frac{\alpha}{2\pi} + b - 1]\right) = b - a.
		\end{align*}
		In either case, $\mu^1(A) = \mu^1_\alpha(A)$.

		If $(1,0) \in A$, then $A \setminus \{(1,0)\}$ is the disjoint union $B \cup C$ of two arcs $B$ and $C$, neither of which contains $(1,0)$. By the first case, $\mu^1(B) = \mu^1_\alpha(B)$ and $\mu^1(C) = \mu^1_\alpha(C)$. Furthermore, $\mu^1(\{(1,0)\}) = \lambda(f^{-1}(\{(1,0)\})) = 0 = \mu^1_\alpha(\{(1,0)\})$ so by additivity,
		\begin{align*}
			\mu^1(A) = \mu^1(B) + \mu^1(C) + \mu^1(\{(1,0)\}) = \mu^1_\alpha(B) + \mu^1_\alpha(C) + \mu^1_\alpha(\{(1,0)\}) = \mu^1_\alpha(A).
		\end{align*}
		We conclude that $\mu^1(A) = \mu^1_\alpha(A)$ for all $A\in \cP$.

		Notice that the intersection of open arcs is still an open arc, hence $\cP$ is a $\pi$-system. Furthermore, $\Lambda = \{A\in \cS ~\mid~ \mu^1(A) = \mu^1_\alpha(A)\}$ is a $\lambda$-system. Since $\cP \subseteq \Lambda$, by Dynkin's $\pi-\lambda$ Theorem, $\sigma(\cP) \subseteq \cL$. However, the set of all open arcs is a basis for the subspace topology on $S^1$ inherited from $\bR^2$, hence $\cS \subseteq \Lambda$. Hence $\mu^1 = \mu^1_\alpha$, and because $\alpha$ was chosen arbitrarily, we conclude that the pushforward of Lebesgue measure on $S^1$ is rotation invariant.
	\end{prf}
	\prob[\textsc{Exercise 2.3}] (A change-of-variable formula). Let $(S,\cS,\mu)$ and $(T,\cT,\nu)$ be two measurable spaces, and let $F:S\to T$ be a measurable function with the property that $\nu = F_* \mu$ (i.e., $\nu$ is the push-forward of $\mu$ through $F$). Show that for every $f \in \cL^0_+(T,\cT)$ or $\cL^1(T,\cT)$, we have
	\begin{align*}
		\int f ~d\nu = \int (f\circ F) ~d\mu.
	\end{align*}
	\begin{prf}
		The following is a procedure roughly matching the standard Lebesgue yoga.

		First, we notice that if $f$ is a simple function given by $f(x) = \alpha_i$ for $x \in A_i \in \cS$ with $1\leq i\leq n$, then $f\circ F$ is a simple function defined by $f\circ F(x) = \alpha_i$ for $x \in F^{-1}(A_i)$. Then
		\begin{align*}
			\int f ~d\nu = \sum_{i=1}^n \alpha_i\nu(A_i) = \sum_{i=1}^n \alpha_i\mu(F^{-1}(A_i)) = \int (f\circ F) d\mu,
		\end{align*}
		so the desired equality holds for simple functions.

		Now suppose that $f \in \cL^0_+(T,\cT)$. By the simple approximation theorem (3.10 in the notes), we may find an increasing sequence of nonnegative simple functions $f_n$ which uniformly approach $f$. Likewise, $f_n \circ F$ is an increasing sequence of simply functions (by what we proved above) which approaches $f \circ F$ from below. By monotone convergence, we then immediately get
		\begin{align*}
			\int f ~d\nu = \lim_{n\to \infty} \int f_n ~d\mu = \lim_{n\to \infty} \int f\circ F ~d\mu = \int f\circ F ~d\mu.
		\end{align*}
		
		Finally, let $f \in \cL^1(T,\cT)$ be an arbitrary Lebesgue integrable function and let $f^+,f^-$ denote the typical $\cL^0_+(T,\cT)$ functions representing the positive and negative portions of $f$. I argue that $(f\circ F)^+ = f^+\circ F$ and $(f\circ F)^- = f^-\circ F$. Indeed, there is almost nothing to check:
		\begin{align*}
			(f\circ F)^+(x) = \max\{0, f(F(x))\} = f^+(F(x)),
		\end{align*}
		and we have something similar for $f^-$. Since $f^+,f^-$ are both $\cL^0_+(T,\cT)$, by what we have already shown we have
		\begin{align*}
			\int f ~d\nu = \int f^+ ~d\nu ~+~ \int f^- ~d\nu = \int (f\circ F)^+ ~d\mu ~+~ \int (f\circ F)^- ~d\mu = \int (f\circ F) ~d\mu,
		\end{align*}
		which concludes the proof.
	\end{prf}
	\prob[\textsc{Exercise 2.4}] (An integrability criterion). Let $(S,\cS,\mu)$ be a finite measure space, and let $f \in \cL^0_+$. Show that 
	\begin{align*}
		\int f d\mu < \infty ~\text{ if and only if } ~ \sum_{n\in \bN} \mu(\{f \geq n\}) < \infty
	\end{align*}
	where, as usual, $\{f \geq n\} = \{x \in S ~:~ f(x) \geq n\}$. \small{\emph{Hint:} Approximate $f$ from below and from above by a piecewise constant function.}
	\begin{prf}
		First, some setup. Define $A_n = \{f \geq n\} \subseteq S$ for $n \in \bN$. Note that this is a decreasing sequence, $A_n \supseteq A_{n+1}$, and that because $f\in \cL_+^0$ we have $S = A_0$. Now define $B_n = A_n \setminus A_{n+1} = A_n \cap (A_{n+1}^c)$; we'll think of $B_n$ as the ``outer shell'' of $A_n$. Since each $A_n$ is measurable, so is $B_n$. Furthermore, for each $x \in S$, if we set $k = \flr{f(x)}$ to be the ceiling of $f(x)$, then $k \leq f(x) < k+1$ and hence $x \in A_n$ but $x \not\in A_{k+1}$. This means $x \in B_k$, and so $\{B_n\}_{n\in \bN}$ forms a pairwise disjoint cover of $S$, i.e. a partition.

		We'll prove both implications via contrapositive. Suppose first that $\sum_{n \in \bN} \mu(A_n) = \infty$. Define a sequence of simple functions $g_n:S\to \bR$ with $B_0,...,B_n$ as their level sets:
		\begin{align*}
			g_n(x) =
			\begin{cases}
				k & x \in B_k ~ \text{ where } ~ 0\leq k\leq n \\
				0 &\text{otherwise}
			\end{cases}.
		\end{align*}
		This is well defined: $g_n(x)$ doesn't have contradictory definitions since $B_i \cap B_j = \emptyset$ whenever $i\neq j$ so $g_n$, and $g_n$ is defined on all of $S$ since $\{B_n\}_{n\in \bN}$ covers $S$. For $x \in B_k$ and $n \geq k$, we have by definition that $f(x) \geq k = g_n(x)$, hence
		\begin{align*}
			\int f ~d\mu \geq \int g_n ~d\mu = \int g_n ~d\mu &= \sum_{k = 0}^n k\mu(B_k).
		\end{align*}
		The above equality follows immediately from the definition of an integral of a simple function. We may take limits as this inequality doesn't depend on $n$, which gives us
		\begin{align*}
			\int f~d\mu &\geq \lim_{n\to \infty} \int g_n ~d\mu = \sum_{k=1}^\infty k\mu(B_k) \\
						&= \sum_{k=0}^\infty k(\mu(A_k) \setminus \mu(A_{k+1})) \\
						&= \sum_{k=0}^\infty k\mu(A_k) - (k-1)\mu(A_k) \\
						&= \sum_{k=0}^\infty \mu(A_k) = \sum_{k \in \bN} \mu(\{f\geq n\}) ~ - ~ \mu(S).
		\end{align*}
		Since $\mu(S)$ is finite and $\sum_{k \in \bN} \mu(\{f\geq n\})$ is infinite, we get that $\int g_n ~d\mu \to \infty$ and hence $\int f ~d\mu = \infty$ as well.

		\bigskip

		Now suppose $\int f ~d\mu = \infty$. Using the same $A_k$ and $B_k$ as before, we shift the $g_n:S\to \bR$ we used previously up by one:
		\begin{align*}	
			g_n(x) =
			\begin{cases}
				k+1 & x \in B_k ~ \text{ where } ~ 0\leq k\leq n \\
				0 &\text{otherwise}
			\end{cases}.
		\end{align*}
		Now for $x\in B_k$ we get $f(x) < k+1 = g_n(x)$. However, it is not the case that $\int f~d\mu \leq \int g_n d\mu$, as we'd like, since $g_n$ is zero outside of $B_0 \cup...\cup B_n$. To fix this, define $f_n:S\to \bR$ by
		\begin{align*}
			f_n(x) = f(x)\cdot 1_{B_0 \cup...\cup B_n}.
		\end{align*}
		Then $f_n \in \cL^0_+$ for each $n \in \bN$, $f_0(x) \leq f_1(x) \leq f_2(x) \leq ...$ and $\lim_{n\to \infty} f_n(x) \to f(x)$ for all $x \in S$, so $f_n$ satisfies the properties of the monotone convergence theorem and gives us
		\begin{align*}
			\lim_n \int f_n ~d\mu = \int f~d\mu.
		\end{align*}
		More importantly, $f_n$ is less than $g_n$ on $B_0\cup...\cup B_n$ and is zero everywhere else, giving us
		\begin{align*}
			\int f_n ~d\mu \leq \int g_n ~d\mu.
		\end{align*}
		Since this is true of all $n$ we can take limits to get
		\begin{align*}
			\int f~d\mu &= \lim_{n\to \infty} \int f_n ~d\mu \leq \lim_{n\to \infty} \int g_n ~d\mu = \sum_{k = 0}^\infty (k+1) \mu(B_k) \\
						&= \sum_{k \in \bN} (k+1)(\mu(A_k) - \mu(A_{k+1})) \\
						&= \sum_{k\in \bN} (k+1)\mu(A_k) - (k) \mu(A_k) \\
						&= \sum_{k\in \bN} \mu(\{f \geq k\}).
		\end{align*}
		Since $\infty = \int f d\mu \leq \sum_{k\in \bN}\mu(\{f\geq k\})$, we conclude that $\sum_{k \in \bN} \mu(\{f\geq k\}) = \infty$, proving the second implication of the problem.

		\bigskip

		Note: what we've really proven here is that $\sum_{n \in \bN} \mu(\{f\geq n\}) ~-~ \mu(S) \leq \int f ~d\mu \leq \sum_{n \in \bN} \mu(\{f\geq n\})$. From this inequality, it is clear that $\int f ~d\mu = \infty \implies \sum_{n \in \bN} \mu(\{f\geq n\}) = \infty$, and that the reverse implication is true when $\mu(S) < \infty$.
	\end{prf}
	\prob[\textsc{Exercise 2.5}]
	\begin{prf}
		\begin{center}
			$\sim$ no time -- that's all folks$\sim$
		\end{center}
	\end{prf}
\end{homework}
\end{document}
