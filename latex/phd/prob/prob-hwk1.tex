\documentclass[hidelinks,11pt,dvipsnames]{article}
% xcolor commonly causes option clashes, this fixes that
\PassOptionsToPackage{dvipsnames,table}{xcolor}
\usepackage[tmargin=1in, bmargin=1in, lmargin=0.8in, rmargin=1in]{geometry}

%%%%%%%%%%%%%%%%%%%%%%%%%%%%%%%%%%%%%%%%%%%%%%%%%%%%%%%%%%%%%%%%%%%%
%%% For inkscape-figures
%%% Assumes the following directory structure:
%%% master.tex
%%% figures/
%%%     figure1.pdf_tex
%%%     figure1.svg
%%%     figure1.pdf
%%%%%%%%%%%%%%%%%%%%%%%%%%%%%%%%%%%%%%%%%%%%%%%%%%%%%%%%%%%%%%%%%%%%
%\usepackage{import}
\usepackage{pdfpages}
\usepackage{transparent}

\newcommand{\incfig}[2][1]{%
    \def\svgwidth{#1\columnwidth}
    \import{./figures/}{#2.pdf_tex}
}

\pdfsuppresswarningpagegroup=1

% enable synctex for inverse search, whatever synctex is
\synctex=1
\usepackage{float,macrosabound,homework,theorem-env}
\usepackage{microtype}


% font stuff
\usepackage{sectsty}
\allsectionsfont{\sffamily}
\linespread{1.1}

% bibtex stuff
\usepackage[backend=biber,style=alphabetic,sorting=anyt]{biblatex}
\addbibresource{main.bib}

% colored text shortcuts
\newcommand{\blue}[1]{\color{MidnightBlue}{#1}}
\newcommand{\red}[1]{\textcolor{Mahogany}{#1}}
\newcommand{\green}[1]{\textcolor{ForestGreen}{#1}}


% use mathptmx pkg while using default mathcal font
\DeclareMathAlphabet{\mathcal}{OMS}{cmsy}{m}{n}

% fixes the positioning of subscripts in $$ $$
\renewcommand{\det}{\operatorname{det}}

\usetikzlibrary{positioning, arrows.meta}
\newcommand{\here}[2]{\tikz[remember picture]{\node[inner sep=0](#2){#1}}}

%%%%%%%%%%%%%%%%%%%%%%%%%%%%%%%%%%%%%%%%%%%%%%%%%%%%%%%%%%%%%%%%%%%%%
%%% Entry Counter
%%%%%%%%%%%%%%%%%%%%%%%%%%%%%%%%%%%%%%%%%%%%%%%%%%%%%%%%%%%%%%%%%%%%%
\newcounter{entry-counter}
\newcommand{\entry}[1]
{
	\addtocounter{entry-counter}{1}
    \tchap{Entry \arabic{entry-counter}}
	%\addcontentsline{toc}{section}{Entry \arabic{entry-counter}: #1}
	\vspace{-1.5em}
    \begin{center}
		\small \emph{Written: #1}
    \end{center}
}

\usepackage{titling}
\renewcommand\maketitlehooka{\null\mbox{}\vfill}
\renewcommand\maketitlehookd{\vfill\null}


<<<<<<< Updated upstream
% for more leftrightarrow options
\usepackage{stackrel}
\newcommand\mapsfrom{\mathrel{\reflectbox{\ensuremath{\mapsto}}}}

\newcommand{\leftrarrows}{\mathrel{\raise.75ex\hbox{\oalign{%
  $\scriptstyle\leftarrow$\cr
  \vrule width0pt height.5ex$\hfil\scriptstyle\relbar$\cr}}}}
\newcommand{\lrightarrows}{\mathrel{\raise.75ex\hbox{\oalign{%
  $\scriptstyle\relbar$\hfil\cr
  $\scriptstyle\vrule width0pt height.5ex\smash\rightarrow$\cr}}}}
\newcommand{\Rrelbar}{\mathrel{\raise.75ex\hbox{\oalign{%
  $\scriptstyle\relbar$\cr
  \vrule width0pt height.5ex$\scriptstyle\relbar$}}}}
\newcommand{\longleftrightarrows}{\leftrarrows\joinrel\Rrelbar\joinrel\lrightarrows}

\makeatletter
\def\leftrightarrowsfill@{\arrowfill@\leftrarrows\Rrelbar\lrightarrows}
\newcommand{\xleftrightarrows}[2][]{\ext@arrow 3399\leftrightarrowsfill@{#1}{#2}}
\makeatother
=======
>>>>>>> Stashed changes
\begin{document}
\pagestyle{empty}
	\LARGE
\begin{center}
	Title Goes Here \\
	\Large
	Isaac Martin \\
    Last compiled \today
\end{center}
\normalsize
\vspace{-2mm}
\hru
\tchap{Chapter 1}
\begin{homework}[e]
<<<<<<< Updated upstream
	\prob Let $(S,\cS)$ be a measurable space, and let $\{A_n\}_{n\in \bN}$ be a sequence in $\cS$. Show that the following sets are also in $\cS.$
	\begin{enumerate}[(1)]
		\item the set $C_1$ of all $x \in S$ such that $x \in A_n$ for at least 5 different values of $n$.
		\item the set $C_2$ of all $x \in S$ such that $x \in A_n$ for exactly $5$ different values of $n$.
		\item the set $C_3$ of all $x \in S$ such that $x \in A_n$ for all but finitely many $n$ (``finitely many'' includes none).
		\item the set $C_4$ of all $x \in S$ such that $x \in A_n$ for at most finitely many values of $n$.
	\end{enumerate}
	\begin{prf}
		The following lemma will be useful throughout this problem.
		\begin{lem}\label{lem:finite-subsets-of-bN}
			Let $X = \{A \subset \bN ~\mid~ |A| < \infty\}$ be the collection of all finite subsets of $X$. $X$ is countable.
		\end{lem}
		\begin{proof}
			Define a function $f:X \to \{0,1\}^\bN$ by $f(A) = b_A$ where $b_A = (b_0,b_1,...)$ with $b_k = 1$ if $k \in A$ and $0$ otherwise. If $f(A) = f(B)$, then for each $i \in A$, $(b_A)_i = 1 = (b_B)_i \implies i \in B$, and likewise for each $i \in B$, $(b_B)_i = 1 = (b_A)_i \implies i \in A$, so $A = B$ and $f$ is injective.
			Now construct a function $g:X\to \bN$ by 
			\begin{align*}
				g(A) = \sum_{i = 0}^\infty f(A)_i 2^k.
			\end{align*}
			Since each $A \in X$ is finite, $f(A)_i$ is well defined, and is injective by the uniqueness of binary expansions.
		\end{proof}
		We now tackle the four parts of this question.
		\begin{enumerate}[(1)]
			\item Let $\cI = \{I \subseteq \bN ~\mid~ |I| = 5\}$. Define 
				\begin{align*}
					B = \bigcup_{I \in \cI} \bigcap_{i \in I} A_i.
				\end{align*}
				Notice first that $B$ is measurable. Each intersection $\bigcap_{i \in I} A_i$ is an intersection of five measurable sets, and is hence itself measurable. The collection $\cI$ is a subset of $X$ defined in the above lemma, and is hence countable.

				I claim that $C_1 = B$. Indeed, if $x \in C_1$, then it is contained in $A_n$ for at least five different values of $n$. Let $I = \{i_1,...,i_5\}$ be five of these values. Then $x \in A_{i_1}\cap ... \cap A_{i_5}$ and hence $x \in B \implies C_1 \subseteq B$.

				Now suppose that $x \in B$. Then $x \in \bigcap_{i \in I} A_i$ for some five element set $I\subseteq \bN$. This means $x$ is contained in $A_n$ for at least five distinct values of $n$, and hence $x \in C_1\implies B \subseteq C_1$. This proves that $C_1 = B$ and is therefore measurable.

			\item Let $\cI$ be the same collection as in part (1). For each $I \in \cI$, define
				\begin{align*}
					B_I = \bigcap_{i\in I} A_i ~ \cap ~ \bigcap_{i \in \bN \setminus I} A_i^c.
				\end{align*}
				Each $B_I$ is a countable intersection of measurable sets and is hence measurable. Now define
				\begin{align*}
					B = \bigcap_{I \in \cI} B_I.
				\end{align*}
				The set $B$ is measurable as it is a countable union of measurable sets. I claim that $C_2 = B$.

				Suppose $x \in C_2$. Let $I\subseteq \bN$ be the collection of the unique five natural numbers $n$ for which $x \in A_n$. This means $x \not \in A_n$ for all $n \in \bN \setminus I$, or equivalently $x \in A_n^c$. This means $x \in B_I$ by the definition of $B_I$, and hence $x \in B$.

				Conversely, if $x \in B$, then there is some five-element subset $I \subset \bN$ such that $x \in B_I$. Since $x \in A_i$ for each $i \in I$, $x$ is in $A_n$ for five distinct values of $n.$ Since $x \in A_i^c \iff x \not\in A_i$ for all natural numbers $i$ not in $I$, $x$ is not in $A_n$ for any other value of $n$. This implies that $x \in C_2$.

				We have both inclusions, and hence $C_2 = B$ is measurable.

			\item Define
				\begin{align*}
					B = \bigcup_{n = 0}^\infty \bigcap_{k \geq n}^\infty A_k.
				\end{align*}
				This is a countable union and intersection of measurable sets and hence measurable.

				Suppose $x \in C_3$. Then $x$ avoids $A_n$ for only finitely many values of $n$, hence there is some $N \in \bN$ such that $x \in A_k$ for all $k > N$. This implies $x \in B$.

				Now suppose $x \in B$. Then $x \in \bigcap_{k > n}^\infty A_k$ for some $n \in \bN$, and is hence contained in all but at most finitely many $A_n$. This means $x \in C_3$.

				We conclude that $C_3 = B$ and is hence measurable.

			\item Define
				\begin{align*}
					B = \bigcup_{n = 0}^\infty \bigcap_{k \geq n}^\infty A_k^c.
				\end{align*}
				As in part (3), the set $B$ is measurable as it is a countable union and intersection of measurable sets.

				If $x \in C_4$, then $x$ is contained in at most finitely many $A_n$, or equivalently, that $x$ is contained in all but finitely many complements $A_n^c$. There must then be some $N \in \bN$ for which $x \in A_k$ for all $k> N$. This implies that $x \in B$.

				If $x \in B$, then $x \in \bigcap_{k \geq n}^\infty A_n^c~\iff~ x \not\in \bigcup_{k \geq n}^\infty A_k$. Hence, $x$ is contained in $A_i$ for at most finitely many values of $i$, and consequently $x \in C_4$.

				We conclude that $C_4 = B$ and is therefore measurable.
		\end{enumerate}
	\end{prf}
	\prob (Atomic structure of algebras). A \textbf{partition} of a set $S$ is a family $\cP$ of non-empty subsets of $S$ with the property that each $x \in S$ belongs to exactly one $A \in \cP$.
	\begin{enumerate}[(1)]
		\item How many algebras are there on the set $S = \{1,2,3\}$?
		\item By constructing a bisection between the two families, show that the number of different algebras on a finite set $S$ is equal to the number of different partitions of $S$. Note: the elements of the partition corresponding to an algebra are said to be its atoms.
		\item Does there exist an algebra with 754 elements?
	\end{enumerate}
	\begin{prf}
		\begin{enumerate}[(1)]
			\item We can complete this simply by enumerating them, there are five:
				\begin{align*}
					\cS_1 &= \left\{\emptyset, S\right\} \\
					\cS_2 &= \left\{\emptyset, \{1\}, \{2,3\}, S\right\} \\
					\cS_3 &= \left\{\emptyset, \{2\}, \{1,3\},S\right\} \\
					\cS_4 &= \left\{\emptyset, \{3\},\{1,2\},S\right\} \\
					\cS_5 &= \left\{\emptyset, \{1\},\{2\},\{3\},\{1,2\},\{1,3\},\{2,3\},S\right\}. 
				\end{align*}
			\item We first prove the following lemma, which will be useful in the last part of this problem too:
				\begin{lem}\label{lem:partition-sigma}
					Let $S$ be a finite set and $P = \{P_1,...,P_n\}$ be a partition of $S$. Define
					\begin{align*}
						\cS = \left\{\bigcup_{i \in I} P_i \midd I \subseteq \{1,...,n\}\right\} 
					\end{align*}
					to be the collection of subsets of $S$ obtained by taking unions of the $P_i$. Then $\cS = \sigma(P)$.
				\end{lem}
				\begin{proof}
					We need only show that $\cS$ is an algebra contained in $\sigma(P)$. It is clear that $\cS$ is contained in $\sigma(P)$: algebras are closed under finite unions, hence $\bigcup_{i\in I} P_i$ is measurable for any $I \subseteq \{1,...,n\}$. To see that $\cS$ is an algebra, first notice that 
					\begin{align*}
						\emptyset = \bigcup_{i\in \emptyset} P_i, ~\text{ and  }~ S = \bigcup_{i \in \{1,...,n\}} P_i
					\end{align*}
					so $\cS$ contains both $\emptyset$ and $S$. Next, if $I_1$ and $I_2$ are two subsets of $\{1,..,n\}$, then
					\begin{align*}
						\bigcup_{i \in I_1} P_i ~ \cup ~ \bigcup_{i \in I_2} P_i = \bigcup_{i \in I_1 \cup I_2} P_i,
					\end{align*}
					so $\cS$ is closed under finite unions. Finally,
					\begin{align*}
						\left(\bigcup_{i \in I} P_i\right)^c = S \setminus \left(\bigcup_{i \in I} P_i\right) = \bigcup_{i\in I^c} P_i
					\end{align*}
					where $I^c = \{1,...,n\}\setminus I$, so $\cS$ is closed under complement. This completes the proof. 
				\end{proof}

				We now proceed to the problem. Let $\cP$ denote the set of partitions of $S$ and $\cA$ the set of algebras on $S$. Define a function $f:\cP \to \cA$ by sending a partition $P$ to the algebra it generates, i.e. $f(P) = \sigma(P)$. To construct an inverse of $f$, I claim the following: given an algebra $\cS$ on $S$, the collection $A$ of atoms defined
				\begin{align*}
					A_\cS = \left\{M_x\midd x \in S\right\}, \text{ where } ~ M_x = \bigcap_{\substack{M \in \cS \\ x \in M}} M
				\end{align*}
				is a partition of $S$. Note that $M_x$ is measurable in $\cS$ since it is a finite intersection of measurable sets. First, notice that for every $x \in S$, $x \in M_x$, and hence $A_\cS$ forms a cover of $S$. Now suppose there is some $y \in M_x$ such that $x \neq y$. Then for every measurable set $M \in \cS$ which contains $x$, $y \in M$. If there were some measurable set $N$ with $y \in N$ but $x \not\in N$, then $N^c$ would be a measurable set containing $x$ but not $y$, which cannot happen. Thus, a measurable set contains $x$ if and only if it contains $y$, and hence $M_x = M_y$. It follows that if $x \neq y$ and $M_x \cap M_y \neq \emptyset$, then there is some $z \in M_x \cap M_y \implies M_x = M_z = M_y$. Thus, any two sets $M_x$ and $M_y$ are either disjoint or equal. Since $A_\cS$ covers $S$ by pairwise disjoint sets, it is a partition.


				Define $g:\cA\to \cP$ by $\cS \mapsto A_\cS$. I claim that $f$ and $g$ are inverses. We first show that the algebra generated by $A_\cS$ is exactly $\cS$, i.e. that $\sigma(A_\cS) = \cS$. If $M \in \cS$ is a measurable set, then for each $x \in M$, $M_x \subseteq M$. This implies that $M = \bigcup_{x \in M}M_x$. Since $S$ is finite, this union is also finite, and hence $M$ is measurable in $\sigma(\cA_\cS)$. As $\sigma(\cA_\cS)$ is the smallest algebra containing $\cA_\cS$, $\sigma(\cA_\cS) = \cS$, and hence $f(g(\cS)) = \cS$.

				Now suppose we start with a partition $P$ of $S$. By the Lemma, $\sigma(P)$ consists exclusively of unions of the $P_i$. Hence, if $x \in S$, $P_i$ is the unique partition element containing $x$ and $M$ is a measurable set in $\sigma(P)$ containing $x$, then there is some $I \subseteq \{1,...,n\}$ such that 
				\begin{align*}
					M = \bigcup_{j \in I} P_j
				\end{align*}
				and hence $j \in I$. This is a long way of saying that if $M$ is a measurable set containing an element $x \in S$, then $M$ also contains the unique $P_i$ containing $x$. Hence, $P_i \subseteq M_x \subseteq P_i$, so $M_x = P_i$. This means $A_{\sigma(P)} = P$, so $g(f(P)) = P$.

				Since $f(g(\cS)) = \cS$ and $g(f(P)) = P$, $f$ and $g$ are inverses and we therefore have a correspondence
				\begin{align*}
					\big\{~\text{partitions of $S$}\big\}~ \xleftrightarrows[P\mapsto \sigma(P)]{\cA_\cS \mapsfrom \cS} \big\{~\text{algebras on $S$}~\big\}.
				\end{align*}

			\item Short answer: No.

				Longer answer: If $S$ is a finite set and $\cS$ is an algebra on $S$, then $\cS = \sigma(P)$ for some partition $P = \{P_1,...,P_n\}$ of $S$ by part (2). I claim that $|\sigma(P)| = 2^{|P|}$. By the lemma proven at the start of part (2), every set $M \in \sigma(P)$ is of the form
				\begin{align*}
					M = \bigcup_{i\in I} P_i
				\end{align*}
				for some $I \subseteq \{1,...,n\}$. The map $f:\sigma(P) \to \{0,1\}^n$ defined
				\begin{align*}
					f\left(\bigcup_{i\in I} P_i\right) = (s_1,...,s_n), \text{ where } ~ s_j =
					\begin{cases}
						0 & j \not\in I \\
						1 & j \in I
					\end{cases}
				\end{align*}
				is a bijection. There are $2^n$ elements in $\{0,1\}^n$, so this proves the desired result. Since $754$ is not a power of $2$, there is no algebra with $754$ elements.
		\end{enumerate}
	\end{prf}
	\prob Show that $f:\bR\to \bR$ is measurable if it is either monotone or convex.
	\begin{prf}
		Define the following collection of subsets of $\bR$: $A = \{(-\infty,a) \mid a \in \bR\}$. I claim that $\sigma(A) = \cB(\bR)$, i.e. that $A$ generates the Borel algebra on $\bR$. To see this, it suffices to show that all open intervals are contained in $\sigma(A)$. Fix an open interval $(a,b) \subseteq \bR$. Sigma algebras are closed under intersection, so $(-\infty, b) \cap (-\infty,a) = [a,b)$ is in $\sigma(A)$. To obtain an open set, take a countable sequence of these half-open intervals and union them:
		\begin{align*}
			(a,b) = \bigcup_{n=1}^\infty \left[a+ \frac{1}{n},b\right).
		\end{align*}
		Hence $(a,b) \in \sigma(A)$, and since $A \subseteq \cB(\bR)$, we have $\sigma(A) = \cB(\bR)$. We now move on to the problem at hand.
		
		Suppose first that $f$ is monotone increasing, so that $x \leq y\implies f(x) \leq f(y)$. By a theorem from class, it suffices to prove $f$ is measurable on a generating set for the Borel algebra. Consider the interval $(-\infty,a)$ and set
		\begin{align*}
			b = \inf \{x \in \bR~\mid~ a \leq f(x)\}.
		\end{align*}
		If $f^{-1}((-\infty,a)) = \emptyset$, then it is measurable. If $f^{-1}((-\infty,a))$ is not empty, then I claim that it is either $(-\infty,b)$ or $(-\infty,b]$. Indeed, if $x \in (-\infty,b)$ then $x < b \implies f(x) \leq f(b)$ by monotonicity. If $a \leq f(x)$, then we would have $f(b) \leq f(x)$ by the definition of $b$ and hence $b = x$, which can't happen since $b\not\in (-\infty,b)$. Thus $f(x) < a$ and $(-\infty,b) \subseteq f^{-1}((-\infty,a))$.

		The reverse inclusion may not hold, as it is possible that $b \in f^{-1}((-\infty,a))$. If it is, then for all $x > b$ we have $f(b) \leq f(x)\implies a \leq f(x)$, and hence $x \not \in f^{-1}((-\infty,a)) \implies f^{-1}((-\infty,a)) = (-\infty,b]$. If $b \not\in f^{-1}((-\infty,a))$, then for all $x \geq b$ we have $a \leq f(b) \leq f(x)\implies x \not\in f^{-1}((-\infty,a))$, in which case $f^{-1}((-\infty,a)) = (-\infty,b)$. In any case, $f^{-1}((-\infty,a))$ is measurable, and because $f$ is measurable on a set which generates $\bB(\bR)$, we conclude $f$ is a measurable function.

		Now suppose that $f:\bR\to \bR$ is monotone decreasing. The function $g:\bR\to \bR$ defined $g(x) = - f(x)$ is then monotone increasing, and hence measurable by what we have already shown. Let $h:\bR\to \bR$ be the function $h(x) = -x$, which is a polynomial function and hence continuous. We have that $f(x) = h\circ g (x) = -g(x)$, so $f$ is a composition of measurable functions and hence itself measurable.

		Now suppose $f:\bR\to \bR$ is a continuous function. Fix $a \in \bR$ and suppose we have two elements $x, y\in \bR$ such that $x < y$ and $f(x), f(y) < a$. For every element $z$ between $x$ and $y$ we may find a $t \in [0,1]$ such that $z = (1-t)x + ty$ \big(specifically, $t = \frac{z - x}{y - x}$\big) and hence
		\begin{align*}
			f(z) = f((1-t)x + ty) \leq (1-t)f(x)+tf(y) < (1-t)a + ta = a
		\end{align*}
		by convexity. This means that for any two points $x,y \in f^{-1}((-\infty,a))$, the interval $[x,y]$ is contained in $f^{-1}((-\infty,a))$ too, and hence $f^{-1}((-\infty,a))$ is an interval (either open, closed or half open) and measurable.

		If no such two points exist, then $f^{-1}((-\infty,a))$ is either a singleton or the emptyset, both of which are measurable. In any case, $f^{-1}((-\infty,a))$ is measurable for all $a \in \bR$ and we conclude that $f$ is a measurable function.
	\end{prf}
	\prob One can obtain the product $\sigma$-algebra $\cS$ on $\{-1,1\}^\bN$ as the Borel $\sigma$-algebra corresponding to a particular topology which makes $\{-1,1\}^\bN$ compact. Here is how. Start by defining a mapping $d:\{-1,1\}^\bN \times \{-1,1\}^\bN \to[] [0,\infty)$ by
	\begin{align*}
		d(s^1,s^2) = 2^{-i(s^1,s^2)}, ~ \text{where }~ i(s^1,s^2) = \inf \{i \in \bN ~\mid~ s_i^1 \neq s_i^2\},
	\end{align*}
	for $s^j = (s^j_1,s^j_2,...)$, $j=1,2.$
	\begin{enumerate}[(1)]
		\item Show that $d$ is a metric on $\{-1,1\}^\bN$.
		\item Show that $\{-1,1\}^\bN$ is compact under $d.$
		\item Show that each cylinder of $\{-1,1\}^\bN$ is both open and closed under $d$.
		\item Show that each open ball is a cylinder.
		\item Show that $\{-1,1\}^\bN$ is separable, i.e., it admits a countable dense subset.
		\item Conclude that $\cS$ coincides with the Borel $\sigma$-algebra on $\{-1,1\}^\bN$ under the metric $d$.
	\end{enumerate}
	\begin{prf}
		Throughout this problem, $S = \{-1,1\}^\bN$ is the coin toss space and $\cS$ is the product $\sigma$-algebra.
		\begin{enumerate}[(1)]
			\item First,
				\begin{align*}
					d(s^1,s^2) = 2^{-i(s^1,s^2)} = 0 \iff i(s^1,s^2) \infty \iff s^1 = s^2.
				\end{align*}
				Second, for any two $s^1,s^2 \in S$, $i(s^1,s^2) = \inf\{i \in \bN ~\mid~ s_i^1 \neq s^2_i\}) = i(s^2,s^1)$ so $d(s^1,s^2) = d(s^2,s^1)$ so symmetry is satisfied.

				Finally, we show that $d$ satisfies the triangle inequality. In fact, we show that $d$ satisfies the ultrametric triangle inequality. Suppose we have $x,y,z \in S$ and set $i(x,y) = a$, $i(x,z) = b$ and $i(y,z) = c$. If $b \leq a$, then $d(x,y) \leq d(x,z) \leq \max\{d(x,z),d(y,z)\}$. If $b > a$, then $z_i = x_i$ for all $i \leq a$. Hence, $z_a = x_a \neq y_a$ and so $i(y,z) = a$. This implies $d(x,y) = d(y,z) \leq \max\{d(x,z),d(y,z)\}$.

				Hence, $d$ is an ultrametric on $S$.
			\item For metric spaces, compactness is equivalent to sequential compactness. We show that $S$ satisfies the latter. Let $\{s^n\}_{n\in \bN} \subset S$ be an arbitrary sequence in $S$. By the countable pigeonhole principle, we may choose $a_1 \in \{-1,1\}$ such that there exist infinitely many $i \in \bN$ such that $s^i_1 = a_1$. Now set $A_1 = \{s^i \mid s^i_1 = a_i\}$ and note that $|A_1| = \infty$.

				Suppose we have chosen $a_1,...,a_n$ and constructed subsets $\{s^n\}_{n\in \bN} \supset A_1 \supset ... \supset A_n$ such that for $1\leq k\leq n$, $|A_k| = \infty$ and for all $s \in A_k$, $s_i = a_i$. Since $A_n$ is an infinite set consisting of elements in $S$, there again is some $a_{n+1}$ such that the set $A_n = \{s \in A_n \mid s_{n+1} = a_{n+1}\}$ satisfies $|A_{n+1}| = \infty$.

				I claim that the element $a = (a_1,a_2,a_3,...) \in S$ is a limit point of $\{s^n\}_{n\in \bN}$. Choose $t^1 \in A_1$, and inductively choose $t^n \in A_n \setminus \{t^1,...,t^{n-1}\}$. Then, by construction, $t^k_i = a_i$ for all $1\leq i\leq k$, and hence
				\begin{align*}
					0 \leq \lim_{k \to \infty} d(t^k, a) \leq \lim_{k\to \infty} 2^{-k} = 0.
				\end{align*}
				Since $\{t^k\}_{k\in \bN}\subseteq \{s^n\}_{n\in \bN}$ is a subsequence and $\{s^n\}_{n\in \bN}$ was chosen arbitrarily, every sequence in $S$ has a subsequence which converges to an element of $S$. Hence $S$ is compact.
			\item We first describe the general form of a cylinder in $S$. The algebra on $\{-1,1\}$ is simply the power set $\{\emptyset,\{-1\},\{1\},\{-1,1\}\}$, so the set $B$ in the definition of a cylinder is simply any subset $B \subseteq \{-1,1\}^k$. A cylinder with base $B$ can then be specified fully with a multi-index $\alpha = (\alpha_1,...,\alpha_k)\in \bN^k$ by
				\begin{align*}
					C_{B,\alpha} = \{s \in S \mid s_\alpha \in B\},
				\end{align*}
			where for simplicity we denote by $s_\alpha$ the $k$-tuple $(s_{\alpha_1},...,s_{\alpha_k})$. We now wish to show that an arbitrary $C_{B,\alpha}$ is both open and closed, where $k = |\alpha|$ is the length of the multi-index. To see that it is open, it suffices to show that every point in $C_{B,\alpha}$ has an open neighborhood entirely contained in $C_{B,\alpha}$. Fix a point $s \in C_{B_\alpha}$ and choose 
			\begin{align*}
				n > \max\{\alpha_1,...,\alpha_k\}.
			\end{align*}
			Then every point $t$ in the open ball $B_{2^{-n}}(s)$ of radius $2^{-n}$ centered at $s$ is a sequence matching $s$ up until at least the $n$th component, i.e. $t_i = s_i$ for $i \leq n$. This implies that $t_\alpha = s_\alpha$ and hence $t \in C_{B,\alpha}$. We then have that $B_{2^{-n}}(s) \subseteq C_{B,\alpha}$, so $C_{B,\alpha}$ is open.

			Now consider the complement $D = S \setminus C_{B,\alpha}$. If $D$ is empty, then $C_{B,\alpha} = S$ and is hence closed. Otherwise, there exists some $s \in D$ which necessarily satisfies $s_\alpha \not\in B$. As before, choose $n > \max\{\alpha_1,...,\alpha_k\}$. Any $t \in B_{2^-n}(s)$ then necessarily satisfies $t_\alpha = s_\alpha$, as above, and hence $B_{2^{-n}}(s) \subseteq D$. This shows that $D$ is open, and hence $C_{B,\alpha}$ is closed. We conclude that cylinders are clopen in the metric topology on $S$ induced by $d$.
			\item Consider an open ball $B_{r}(s)$ of radius $r \in [0,\infty)$ centered at a point $s \in S$, and set $n = \inf\{k \in \bN ~\mid~ 2^{-k} \geq r\}$. By definition of $n$ we have $B_{2^{-n}}(s) \supseteq B_{r}(s)$, but I claim that the reverse inclusion holds as well. Indeed, if $t \in B_{2^{-n}}(s)$, then $2^{-i(s,t)} < 2^{-n}$ which implies $i(s,t) < n$. We cannot have $i_2^{-i(s,t)} \geq r$ as it would contradict the minimality of $n$, hence $t \in B_r(s)$ and $B_{2^{-n}}(s) \subseteq B_{r}(s)$.

				The rest of the argument is nearly immediate. A element $t \in S$ lies in $B_r(s) = B_{2^{-n}}(s)$ if and only if $s_i = t_i$ for all $1\leq i<  n$. If we set $\alpha = (1,...,n)$ and $B = \{s_\alpha\}$, then we exactly get that $C_{B,\alpha} = B_{2^{-n}}(s) = B_r(s)$.
			\item Let $D = \{s \in S ~\mid~ s_i = 1 ~\text{ for only finitely many  } ~ i\in \bN\}$. I claim this is a countable dense subset of $S$. To see that it is dense, choose any point $s \in S$. Let $t^n \in S$ be the element such that
				\begin{align*}
					t^n_i = 
					\begin{cases}
						s_i & 1\leq i\leq n \\
						-1 & \text{otherwise}
					\end{cases}.
				\end{align*}
			Then $t^n \in D$ and $d(t^n,s) \leq 2^{-n}$ so $\lim t^n = s$. This proves that $D$ is dense in $S$.

			To see it is countable, we construct an injection into the rational numbers $\bQ$. We send $t \in D$ to a rational number $r$ such that $0 \leq r < 2$ and whose $n$th digit is $0$ if $t_n = -1$ and $1$ if $t_n = 1$. More precisely, we define a map $f:D\to \bQ$ by
			\begin{align*}
				f(s) = \sum_{i=0}^\infty \frac{1}{10^i} \frac{1}{2}(1 + s_i).
			\end{align*}
			The element $f(s)$ is indeed rational since it has a finite decimal expansion, and $f$ is injective since $f(s) = f(t)$ if and only if $s_i = t_i$ for all $i\in \bN$. Hence, $D$ is a countable dense subset of $S$.

			\item We now argue that the Borel algebra on $S$ equipped with the metric topology induced by $d$ is equal to the product $\sigma$-algebra. Let $D$ be the countable dense subset defined in part (5).

				First, we show that that every open set $U \subseteq S$ is the countable union of measurable sets and is hence itself measurable. The intersection $U \cap D$ is countable and dense in $U$. By the openness of $U$, for each $t \in U\cap D$ we may find an open ball $B_{r_t}(t) \subseteq U$, and by the denseness of $D$,
				\begin{align*}
					U = \bigcup_{t \in U\cap D} B_{r_t}(t).
				\end{align*}
				Since open balls in $S$ are cylinders and cylinders are measurable, $U$ is also measurable. This implies $\cB(S) \subseteq \cS$.

				For the other inclusion, recall that $\cS$ is generated as a $\sigma$-algebra by cylinders $C_{B,\alpha} \subseteq S$. Each of these sets is open by part (3), so $C_{B,\alpha}$ is measurable in the Borel $\sigma$-algebra on $S$. This implies that $\cS \subseteq \cB(S)$, and we conclude that the product $\sigma$-algebra on $S$ is identical to the Borel sigma algebra induced by the metric topology.
		\end{enumerate}
	\end{prf}
=======
	\prob First problem
	\prob[\textsc{Exercise 10}] Another problem
>>>>>>> Stashed changes
\end{homework}
\end{document}
