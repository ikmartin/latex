\documentclass[hidelinks,11pt,dvipsnames]{article}
% xcolor commonly causes option clashes, this fixes that
\PassOptionsToPackage{dvipsnames,table}{xcolor}
\usepackage[tmargin=1in, bmargin=1in, lmargin=0.8in, rmargin=1in]{geometry}

%%%%%%%%%%%%%%%%%%%%%%%%%%%%%%%%%%%%%%%%%%%%%%%%%%%%%%%%%%%%%%%%%%%%
%%% For inkscape-figures
%%% Assumes the following directory structure:
%%% master.tex
%%% figures/
%%%     figure1.pdf_tex
%%%     figure1.svg
%%%     figure1.pdf
%%%%%%%%%%%%%%%%%%%%%%%%%%%%%%%%%%%%%%%%%%%%%%%%%%%%%%%%%%%%%%%%%%%%
%\usepackage{import}
\usepackage{pdfpages}
\usepackage{transparent}

\newcommand{\incfig}[2][1]{%
    \def\svgwidth{#1\columnwidth}
    \import{./figures/}{#2.pdf_tex}
}

\pdfsuppresswarningpagegroup=1

% enable synctex for inverse search, whatever synctex is
\synctex=1
\usepackage{float,macrosabound,homework,theorem-env}
\usepackage{microtype}


% font stuff
\usepackage{sectsty}
\allsectionsfont{\sffamily}
\linespread{1.1}

% bibtex stuff
\usepackage[backend=biber,style=alphabetic,sorting=anyt]{biblatex}
\addbibresource{main.bib}

% colored text shortcuts
\newcommand{\blue}[1]{\color{MidnightBlue}{#1}}
\newcommand{\red}[1]{\textcolor{Mahogany}{#1}}
\newcommand{\green}[1]{\textcolor{ForestGreen}{#1}}


% use mathptmx pkg while using default mathcal font
\DeclareMathAlphabet{\mathcal}{OMS}{cmsy}{m}{n}

% fixes the positioning of subscripts in $$ $$
\renewcommand{\det}{\operatorname{det}}

\usetikzlibrary{positioning, arrows.meta}
\newcommand{\here}[2]{\tikz[remember picture]{\node[inner sep=0](#2){#1}}}

%%%%%%%%%%%%%%%%%%%%%%%%%%%%%%%%%%%%%%%%%%%%%%%%%%%%%%%%%%%%%%%%%%%%%
%%% Entry Counter
%%%%%%%%%%%%%%%%%%%%%%%%%%%%%%%%%%%%%%%%%%%%%%%%%%%%%%%%%%%%%%%%%%%%%
\newcounter{entry-counter}
\newcommand{\entry}[1]
{
	\addtocounter{entry-counter}{1}
    \tchap{Entry \arabic{entry-counter}}
	%\addcontentsline{toc}{section}{Entry \arabic{entry-counter}: #1}
	\vspace{-1.5em}
    \begin{center}
		\small \emph{Written: #1}
    \end{center}
}

\usepackage{titling}
\renewcommand\maketitlehooka{\null\mbox{}\vfill}
\renewcommand\maketitlehookd{\vfill\null}

\usepackage{indentfirst}

% Title page stuff
\title{Notes for Probability Prelim \\ \vspace{0.5em}{\Large Fall 2022}\vspace{0.5em}\\ The University of Texas at Austin}
\date{Last Compiled: \today}
\author{Isaac Martin}

% start document
\begin{document}
\pagestyle{empty}
\maketitle
\newpage
\tableofcontents
\newpage
\entry{2022-Aug-22} 
Professor: Gordan Z
Email: gordanz@math.utexas.edu
Administrative stuff only today.
\begin{itemize}
  \item Exams will be take-home, there will be midterms. Dates will be decided upon later.
  \item Professor will suggestion books to supplement notes. Many books available. You can get by with the lecture notes, but not a recommended strategy.
  \item Real analysis is the big hurdle for the course.
\end{itemize}

\bigskip

\entry{2022-Aug-24}
\section{Measurable Spaces}
The \textbf{sample space} of a probability problem is the set of all possible outcomes of an experiment, typically denotes $\Omega$.

\begin{defn}\label{defn:algebra}
  Suppose we have a nonempty set $S$ and a set $\cS \subseteq 2^S$. We say that $\cS$ is an \textbf{algebra} if
  \begin{enumerate}[(1)]
    \item $\emptyset, S \in \cS$
    \item $A \in \cS\implies A^c \in \cS$
    \item $\cS$ is closed under finite unions, i.e. $A,B \in \cS\implies A\cup B \in \cS$.
  \end{enumerate}
\end{defn}

\begin{defn}\label{defn:sigma-algebra}
  Let $S$ be a set. A \textbf{sigma algebra} is a collection $\cS$ of subsets of $S$ which is an algebra and is additionally closed under countable unions: $A_n \in \cS$ for $n \in \bN$ implies $\bigcup A_n \in \cS$.
\end{defn}

\begin{example}\label{example:algebra-not-sigma}
  Let $S = \bZ$ and set $\cS$ to be the collection of open and closed sets under the Zariski topology. Then $\cS$ is an algebra but not a sigma algebra.
\end{example}
\begin{defn}\label{defn:pi-system}
  $\cS$ is a $\pi$-system if $A, B \in \cS$ implies that $A\cap B \in \cS$.
\end{defn}
\begin{defn}\label{defn:labmda-system}
  $\cS$ is a lambda system if
  \begin{enumerate}[(1)]
    \item $S \in \cS$
    \item if $A,B \in S$ and $A\subseteq B$ then $B \setminus A \in \cS$
    \item if $(A_n)_{n \in \bN}\in \cS$ and $A_n \nearrow A$ then $A\in \cS$
  \end{enumerate}
\end{defn}
Notice that every sigma algebra is a lambda set but not every lambda set is a sigma algebra. Simply any nested sequence of sets and their set differences inside an ambient space (NOT VERIFIED but I think this works).

\bigskip

\entry{2022-Aug-26} We would like to generate a $\sigma$-algebra. This is the approach to building a measure space, we start by defining a collection $\cA \subseteq 2^S$ and we'd like to generate a $\sigma$-algebra containing $\cA$. Morally, we're specifying a collection of sets which we know how to measure and then adding in everything else we should be able to measure given our methods for measuring simple sets.
\begin{defn}\label{defn:generate-sigma-alg}
  Let $S$ be a set and $\cA \subseteq 2^S$. The \textbf{sigma algebra generated by} $\cA$ is denoted $\sigma(\cA)$ and is the smallest $\sigma$-algebra containing $\cA$
  \begin{align*}
    \sigma(\cA) = \bigcap_{\substack{\cA \subseteq \cB \\ \cB \text{ is a sigma algebra}}} \cB.
  \end{align*}
\end{defn}
\begin{rmk}\label{rmk:need-transfinite-induction}
  One cannot simply take all unions and complements to generate a $\sigma$-algebra. To do such a thing algorithmically, one needs transfinite induction. Instead, simply define the $\sigma$-algebra generated by $\cA$ to be the intersection, and accept that we don't really understand the collection of everything in there.
\end{rmk}
\begin{defn}\label{defn:borel-sigma-algebra}
  Let $X$ be a topological space with topology $\cT$. The \textbf{Borel sigma algebra} of $X$ is denoted $\cB(X)$ or $\cB(\cT)$ and is the sigma algebra generated by the open sets of $X$. We say that $B \subseteq X$ is a \textbf{Borel set} if $B \in \cB$.
\end{defn}
\begin{rmk}\label{rmk:borel-set}
  No set you can construct in finite words in $\bR^n$ is not going to be a Borel set. To give an example of a set which is not Borel, one needs serious set theoretic machinery, which is scary and foreign to the author of these notes. However, $|\cB(\bR^n)| < |2^{\bR^n}|$. So it is that in one sense, $\cB(\bR^n)$ is almost all of $2^{\bR^n}$, and in another, it is almost none of it.
\end{rmk}

\begin{example}\label{example:various-generators-for-borel-R}
  The following are all valid generating sets of $\cB(\bR)$:
  \begin{itemize}
    \item $\cA_1 = \{\text{open subsets of} \bR\}$
    \item $\cA_2 = \{\text{open intervals}\}$
    \item $\cA_3 = \{\text{closed intervals}\}$
    \item $\cA_4 = \{(a,b) ~\mid~ a,b \in \bQ\}$.
  \end{itemize}
  All of these $\cA$ satisfy $\sigma(\cA) = \cB(\bR)$.
\end{example}

Analysts define sigma algebras in order to solve the Banach-Tarski paradox, in some sense. The goal is to strictly define which sets are measurable and which are not. In probability, we have another use for sigma algebras, we think of them abstractly as ways of organizing data/information, very much like topologies.

\begin{defn}\label{defn:measurable-function}
  Let $(S,\cS)$ and $(T,\cT)$ be $\sigma$-algebras. A function $f:S\to T$ is said to be \textbf{measurable} if $A \in \cT \implies f^{-1}(A) \in \cS$, if pullbacks of measurable sets are measurable.
\end{defn}
Measurable functions are the morphisms in the category of $\sigma$-algebras.


\entry{2022-Aug-29}
We ended last time by defining measurable functions $f:S\to T$ where $S$ and $T$ have accompanying $\sigma$-algebras $\cS$ and $\cT$ respectively. We also saw that measurable functions compose to give measurable functions, hence they serve as morphisms in the category of $\sigma$-algebras.

\begin{prop}\label{prop:measurability-criterion}
  Let $\cS$ and $\cT$ be $\sigma$-algebras on $S$ and $T$ respectively and let $f:S\to T$ be a function. Suppose that $\cC\subseteq 2^T$ generates $\cT$ as a $\sigma$-algebra, i.e. $\sigma(\cC) = \cT$, and that $f^{-1}(B) \in \cS$ for all $\cB \in \cC$. Then $f$ is measurable.
\end{prop}
\begin{prf}
  Set $\cD = \left\{\cB \in 2^T ~\midd~ f^{-1}(B) \in \cS\right\}$. We immediately have that $\cC\subseteq \cD$ by the properties of $f$. We'd like to show that $\cD$ is a $\sigma$-algebra, and hence $\cT = \cD$. You can do that pretty easily but Prof man skipped it.
\end{prf}

\begin{cor}\label{cor:continuous-implies-measurable}
  Suppose $f:(S,\cS) \to (T,\cT)$ is a function and that $\cS,\cT$ are Borel $\sigma$-algebras on topological spaces $S,T$. If $f$ is continuous then $f$ is measurable.
\end{cor}
\begin{prf}
  Open sets generate a Borel $\sigma$-algebra, and if $f$ is continuous then the pullback of open maps are open.
\end{prf}

\begin{defn}\label{defn:}
  Suppose we have a set $S$ and an indexed collection of $\sigma$-algebras $(T_\gamma, \cT_\gamma)_{\gamma \in \Gamma}$. Suppose further that $f_\gamma:S\to T_\gamma$ is a function for each $\gamma \in \Gamma$. We define the $\sigma$-algebra on $S$ generated by $(f_\gamma)_{\gamma \in \Gamma}$ by
  \begin{align*}
    \sigma\left(f_\gamma, \gamma \in \Gamma\right) := \sigma\left\{f^{-1}_\gamma(B) \midd B \in \cT_\gamma, \gamma \in \Gamma\right\}.
  \end{align*}
  That is, we take all possible inverse image of all measurable sets and generate a $\sigma$-algebra.
\end{defn}

We'll use this construction in a sec, but we'll break really quick and discuss a nice characterization of algebras on finite sets.

\begin{example}\label{example:algebras-are-partitions-finite}
  $ $
  \begin{enumerate}
    \item Suppose $S$ is finite. We have a one-to-one correspondence between partitions on $S$ and algebras on $S$. Given an algebra, the element of the partition containing a given elements $x \in S$ is the intersection of all measurable sets containing $x$. Given a partition, use the partition to generate a $\sigma$-algebra. This is the one-to-one correspondence.
    \item Now suppose you have a function $f:S\to \bR$ where $S$ is measurable. Then $f$ is measurable if and only if $f$ is constant on the partitions corresponding to the $\sigma$-algebra on $S$.
  \end{enumerate}
\end{example}

\entry{2022-Aug-31} Let $(S_\gamma)_{\gamma \in \Gamma}$ be sets $(\Gamma = \{1,2\}, \bN, \text{ etc})$
For us, 
\begin{align*}
  \prod_{\gamma \in \Gamma} S_\gamma = \left\{f:\Gamma \to \bigcup_{\gamma \in \Gamma} S_\gamma ~\midd~ f(\gamma) \in S_\gamma, \forall \gamma\right\}.
\end{align*}
These are called \textbf{choice functions}. The axiom choice says that one of these choice functions exists, that is all. This product comes along with projection functions $\pi_\gamma:\prod_{\gamma \in \Gamma} S_\gamma \to S_\gamma$ which just evaluates an element $f \in \prod_{\gamma \in \Gamma} S_\gamma$ at $\gamma$. If the $S_\gamma$ are all $\sigma$-algebra or topological spaces or whatever we use these \emph{canonical} projection maps to produce a structure of the same flavor on the product.

When $(S_\gamma,\cS_\gamma)$ are all measurable, we define
\begin{align*}
  \bigotimes_{\gamma \in \Gamma} \cS_\gamma := \sigma\left(\pi_\gamma: \gamma \in \Gamma\right)
\end{align*}
to be the $\sigma$-algebra of the product: it is the smallest $\sigma$-algebra for which all the $\pi_\gamma$ are measurable.

\begin{example}\label{example:basic-product}
  Set $\Gamma = \{1,2\}$, $(S_1,\cS_1)$ and $(S_2,\cS_2).$ If $S = S_1\times S_2$, $\cS = \sigma(\pi_1,\pi_2) = \sigma(\pi^{-1}_1(B_1),\pi_2^{-1}(B_2))$.
\end{example}

\begin{defn}\label{defn:product-cylinder}
  Let $\{S_\gamma,\cS_\gamma\}_{\gamma \in \Gamma}$ be a family of measurable spaces, and let $\left(\prod_{\gamma \in \Gamma}S_\gamma, \bigotimes_{\gamma \in \Gamma} \cS_\gamma\right)$ be its product. A \textbf{cylinder set} is a measurable subset $C \subseteq \prod_{\gamma \in \Gamma} S_\gamma$ such that there exist a finite set $\{\gamma_1,...,\gamma_n\}\subseteq \Gamma$ so that for some $B_i \in S_{\gamma_i}$ we have
  \begin{align*}
    C = \left\{s \in \prod_{\gamma \in \Gamma} S_\gamma ~\midd~ s(\gamma_i) \in B_i\right\}.
  \end{align*}
  That is, all of the $\gamma_i$ coordinates of elements in $C$ lie in our $B_i$ and $C$ is unrestricted in every other coordinate.
\end{defn}

\begin{example}\label{example:coin-toss-space}
  Let $\Gamma = \bN$ and $S_n = \{-1,1\}$. We define the coin toss space to be $\prod_{n \in \bN} S_n = \{f : \bN \to \{-1,1\}\}$.
\end{example}
\printbibliography
\end{document}
