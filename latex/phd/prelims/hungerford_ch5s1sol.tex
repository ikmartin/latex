\documentclass[hidelinks,11pt,dvipsnames]{article}
% xcolor commonly causes option clashes, this fixes that
\PassOptionsToPackage{dvipsnames,table}{xcolor}
\usepackage[tmargin=1in, bmargin=1in, lmargin=0.8in, rmargin=1in]{geometry}

%%%%%%%%%%%%%%%%%%%%%%%%%%%%%%%%%%%%%%%%%%%%%%%%%%%%%%%%%%%%%%%%%%%%
%%% For inkscape-figures
%%% Assumes the following directory structure:
%%% master.tex
%%% figures/
%%%     figure1.pdf_tex
%%%     figure1.svg
%%%     figure1.pdf
%%%%%%%%%%%%%%%%%%%%%%%%%%%%%%%%%%%%%%%%%%%%%%%%%%%%%%%%%%%%%%%%%%%%
%\usepackage{import}
\usepackage{pdfpages}
\usepackage{transparent}

\newcommand{\incfig}[2][1]{%
    \def\svgwidth{#1\columnwidth}
    \import{./figures/}{#2.pdf_tex}
}

\pdfsuppresswarningpagegroup=1

% enable synctex for inverse search, whatever synctex is
\synctex=1
\usepackage{float,macrosabound,homework,theorem-env}
\usepackage{microtype}


% font stuff
\usepackage{sectsty}
\allsectionsfont{\sffamily}
\linespread{1.1}

% bibtex stuff
\usepackage[backend=biber,style=alphabetic,sorting=anyt]{biblatex}
\addbibresource{main.bib}

% colored text shortcuts
\newcommand{\blue}[1]{\color{MidnightBlue}{#1}}
\newcommand{\red}[1]{\textcolor{Mahogany}{#1}}
\newcommand{\green}[1]{\textcolor{ForestGreen}{#1}}


% use mathptmx pkg while using default mathcal font
\DeclareMathAlphabet{\mathcal}{OMS}{cmsy}{m}{n}

% fixes the positioning of subscripts in $$ $$
\renewcommand{\det}{\operatorname{det}}

\usetikzlibrary{positioning, arrows.meta}
\newcommand{\here}[2]{\tikz[remember picture]{\node[inner sep=0](#2){#1}}}

%%%%%%%%%%%%%%%%%%%%%%%%%%%%%%%%%%%%%%%%%%%%%%%%%%%%%%%%%%%%%%%%%%%%%
%%% Entry Counter
%%%%%%%%%%%%%%%%%%%%%%%%%%%%%%%%%%%%%%%%%%%%%%%%%%%%%%%%%%%%%%%%%%%%%
\newcounter{entry-counter}
\newcommand{\entry}[1]
{
	\addtocounter{entry-counter}{1}
    \tchap{Entry \arabic{entry-counter}}
	%\addcontentsline{toc}{section}{Entry \arabic{entry-counter}: #1}
	\vspace{-1.5em}
    \begin{center}
		\small \emph{Written: #1}
    \end{center}
}

\usepackage{titling}
\renewcommand\maketitlehooka{\null\mbox{}\vfill}
\renewcommand\maketitlehookd{\vfill\null}

\usepackage{tikz}
\usetikzlibrary{positioning,calc,intersections,through,backgrounds, shapes.geometric, decorations.markings,arrows}

\begin{document}
% set section number to 1
% fixes theorem numbering without need to have a section title
\setcounter{section}{1}

\pagestyle{empty}
	\LARGE
\begin{center}
	Solutions to Hungerford Chapter 5.1 \\
	\Large
	Isaac Martin \\
    Last compiled \today
\end{center}
\normalsize
\vspace{-4mm}
\hru

\begin{homework}[e]
  %\prob[\textsc{Exercise 15}] 
    \prob $ $
    \begin{enumerate}[(a)]
        \item Show that $[F:K] = 1$ if and only if $F = K$.
        \item Show that if $[F:K]$ is prime, then there are no intermediate fields between $F$ and $K$.
        \item If $u\in F$ has degree $n$ over $K$, then $n$ divides $[F:K]$.
    \end{enumerate}
    \begin{prf}
        $ $
        \begin{enumerate}[(a)]
            \item If $[F:K] = 1$ then $F$ is a 1-dimensional vector space over $K$. Hence there exists some element $x \in F$ such that for all other $y \in F$ there is some $k \in K$ such that $y = kx$. Define a map $\varphi:K\to F$ by $\varphi(k) = kx$. This is surjective by the previous observation and is injective as it is a map of fields. Hence it is an isomorphic.
            \item If $E$ is a field such that $F/E/K$, then $[F:E][E:K] = [F:K]$. Since $[F:K]$ is prime, either $[E:K] = 1$ or $[F:E] = 1$. In the first case $E\cong K$ and in the second $E \cong F$ by part (a). Hence there are no proper intermediate field extensions.
            \item If $u\in F$ has degree $n$ over $K$, then the field extension $K(u)$ of $K$ is degree $n$. Since $K(u) \subseteq F$, $K(u)$ is an intermediate extension and hence $[F:K(u)][K(u):K] = [F:K] \implies [F:K(u)]n = [F:K]$, giving us the result.
        \end{enumerate}
    \end{prf}
\end{homework} 
\end{document}
