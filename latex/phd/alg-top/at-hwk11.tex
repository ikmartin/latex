\documentclass[hidelinks,11pt,dvipsnames]{article}
% xcolor commonly causes option clashes, this fixes that
\PassOptionsToPackage{dvipsnames,table}{xcolor}
\usepackage[tmargin=1in, bmargin=1in, lmargin=0.8in, rmargin=1in]{geometry}

%%%%%%%%%%%%%%%%%%%%%%%%%%%%%%%%%%%%%%%%%%%%%%%%%%%%%%%%%%%%%%%%%%%%
%%% For inkscape-figures
%%% Assumes the following directory structure:
%%% master.tex
%%% figures/
%%%     figure1.pdf_tex
%%%     figure1.svg
%%%     figure1.pdf
%%%%%%%%%%%%%%%%%%%%%%%%%%%%%%%%%%%%%%%%%%%%%%%%%%%%%%%%%%%%%%%%%%%%
%\usepackage{import}
\usepackage{pdfpages}
\usepackage{transparent}

\newcommand{\incfig}[2][1]{%
    \def\svgwidth{#1\columnwidth}
    \import{./figures/}{#2.pdf_tex}
}

\pdfsuppresswarningpagegroup=1

% enable synctex for inverse search, whatever synctex is
\synctex=1
\usepackage{float,macrosabound,homework,theorem-env}
\usepackage{microtype}


% font stuff
\usepackage{sectsty}
\allsectionsfont{\sffamily}
\linespread{1.1}

% bibtex stuff
\usepackage[backend=biber,style=alphabetic,sorting=anyt]{biblatex}
\addbibresource{main.bib}

% colored text shortcuts
\newcommand{\blue}[1]{\color{MidnightBlue}{#1}}
\newcommand{\red}[1]{\textcolor{Mahogany}{#1}}
\newcommand{\green}[1]{\textcolor{ForestGreen}{#1}}


% use mathptmx pkg while using default mathcal font
\DeclareMathAlphabet{\mathcal}{OMS}{cmsy}{m}{n}

% fixes the positioning of subscripts in $$ $$
\renewcommand{\det}{\operatorname{det}}

\usetikzlibrary{positioning, arrows.meta}
\newcommand{\here}[2]{\tikz[remember picture]{\node[inner sep=0](#2){#1}}}

%%%%%%%%%%%%%%%%%%%%%%%%%%%%%%%%%%%%%%%%%%%%%%%%%%%%%%%%%%%%%%%%%%%%%
%%% Entry Counter
%%%%%%%%%%%%%%%%%%%%%%%%%%%%%%%%%%%%%%%%%%%%%%%%%%%%%%%%%%%%%%%%%%%%%
\newcounter{entry-counter}
\newcommand{\entry}[1]
{
	\addtocounter{entry-counter}{1}
    \tchap{Entry \arabic{entry-counter}}
	%\addcontentsline{toc}{section}{Entry \arabic{entry-counter}: #1}
	\vspace{-1.5em}
    \begin{center}
		\small \emph{Written: #1}
    \end{center}
}

\usepackage{titling}
\renewcommand\maketitlehooka{\null\mbox{}\vfill}
\renewcommand\maketitlehookd{\vfill\null}


\usepackage{capt-of}
\usepackage{tikz}
\usetikzlibrary{positioning,calc,intersections,through,backgrounds, shapes.geometric, decorations.markings,arrows}

\def\sset{\subseteq}
\def\iso{\cong}
\def\gend#1{\langle #1\rangle}

\newcommand{\rightoverleftarrow}{%
  \mathrel{\vcenter{\mathsurround0pt
    \ialign{##\crcr
      \noalign{\nointerlineskip}$\longrightarrow$\crcr
      \noalign{\nointerlineskip}$\longleftarrow$\crcr
    }%
  }}%
}

\newcommand\makesphere{} % just for safety
\def\makesphere(#1)(#2)[#3][#4]{%
  % Synopsis
  % \makesphere[draw options](center)(initial angle:final angle:radius)
  \shade[ball color = #3, opacity = #4] #1 circle (#2);
  \draw #1 circle (#2);
  \draw ($#1 - (#2, 0)$) arc (180:360:#2 and 3*#2/10);
  \draw[dashed] ($#1 + (#2, 0)$) arc (0:180:#2 and 3*#2/10);
}
% same thing as make sphere but places white background behind
\newcommand\altmakesphere{} % just for safety
\def\altmakesphere(#1)(#2)(#3)[#4][#5]{%
  % Synopsis
  % \make sphere[draw options](center)(initial angle:final angle:radius)
  \draw [fill=white!30] #1 circle (#2);
  \shade[ball color = #4, opacity = #5] #1 circle (#2);
  \draw #1 circle (#2);
  \draw ($#1 - (#2, 0)$) arc (180:360:#2 and 3*#2/10);
  \draw[dashed] ($#1 + (#2, 0)$) arc (0:180:#2 and 3*#2/10);
  \node at #1 {#3};
}

\begin{document}
% set section number to 1
% fixes theorem numbering without need to have a section title
\setcounter{section}{1}

\pagestyle{empty}
	\LARGE
\begin{center}
	Algebraic Topology Homework 10 \\
	\Large
	Isaac Martin \\
    Last compiled \today
\end{center}
\normalsize
\vspace{-4mm}
\hru

\tchap{Problems from 2.1}
\begin{homework}[e]
  \prob[\textsc{Exercise 16.}]$ $
  \begin{enumerate}[(a)]
    \item Show that $H_0(X,A) = 0$ if and only if $A$ meets each path-component of $X$.
    \item Show that $H_1(X,A) = 0$ if and only if $H_1(A) \to H_1(X)$ is surjective and each path-component of $X$ contains at most one path-component of $A$.
  \end{enumerate}
  \begin{prf}$ $
    \begin{enumerate}[(a)]
      \item Suppose first that $A$ meets all path components of $X$. To show $H_0(X,A) = 0$, it suffices to show that all relative $0$-cycles are equivalent to $0$-cycles in $A$. First, note that a $0$-simplex $\sigma:\Delta_0\to X$ is entirely determined by the image of the single point constituting $\Delta_0$, that is, there is a one-to-one correspondence between points of $X$ and $0$-simplicies. Denote the $0$-simplex sending the point in $\Delta_0$ to $x \in X$ by $\sigma_x$. Given a point $x\in X$, there exists a path $\gamma:[0,1]\to X$ such that $\gamma(1) = x$ and $\gamma(0) = a$ for some point $a \in A$, by the assumption that $A$ meets every path component of $X$. Such a path can be realized as a $1$-simplex via composition with an isomorphism $\Delta_1\xrightarrow{\sim} [0,1]$. Then $\delta(\gamma) = \sigma_x - \sigma_a$. As an element in $C_0(X,A)$, this is equivalent to $\sigma_x$ since $-\sigma_a \in C_0(A)$. But then $\sigma_x \in \img(\delta_1)$, and hence represents the trivial class in $H_0(X,A)$. Since $x$ was chosen arbitrarily, every $0$-simplex represents the trivial class in $H_0(X,A)$, so $H_0(X,A) = 0$.

        Now suppose that $H_0(X,A) = 0$. To show this implies $A$ meets every path component of $X$, we consider the long exact sequence
        \begin{align*}
          ... \to H_1(X,A) \xrightarrow{\partial} H_0(A) \xrightarrow{i_*} \xrightarrow H_0(X) \xrightarrow{j_*} H_0(X,A) \to 0.
        \end{align*}
        Recall that $i_*$ acts by sending a class $[\alpha] \in H_0(A)$ represented by a cycle to $[i(\alpha)] \in H_0(X)$. Since $H_0(X,A)$ is $0$, the map $i_*$ induced by the inclusion $A\hookrightarrow X$ must be surjective. This means for each $[\beta]\in H_0(X)$ there is some $[\alpha]\in H_0(A)$ such that $i(\alpha) = \beta$. This implies that $A$ has nontrivial intersection with the cycle $\beta$ in $X$. Taking $\beta$ to lie in a particular path component of $X$ gives the result, for as we have seen, $H_0(X)$ can be generated by choosing one cycle for each path component of $X$.

      \item Note first that each path component of $X$ contains at most one path component of $A$ if and only if $H_0(A) \xrightarrow{i_*} H_0(X)$ is injective. Recall that the free abelian group $H_0(A)$ can be generated by choosing a single cycle in each path component of $A$. If two path components $A_1$ and $A_2$ of $A$ are both intersect a path component  $X_1$ of $X$ nontrivially, then we can choose two cycles $a_1$ and $a_2$ corresponding in $A_1$ and $A_2$ respectively such that $i(a_1),i(a_2) \in X_1$. However, this means these are both cycles in a single path component of $X$, and hence represent the same element in $H_0(X)$.

        If instead $i_*:H_0(A)\to H_0(X)$ is not injective. Then we can find some element $[a] \neq 0 \in H_0(A)$ such that $i_*([a]) = 0$. Let $I$ be an index set for the path components $A_k$ of $A$ and $a_k$ be a cycle contained in $A_k$. Then we can find some $c_k \in \bZ$ such that only finitely many are nonzero and
        \begin{align*}
          [a] = \sum_{k\in I} c_k [a_k].
        \end{align*}
        Now applying $i_*$ gives $\sum_{k\in I} c_k i_*([a_k]) = i_*([a]) = 0$. If each $i_*([a_k])$ belonged to a separate path component of $X$, then the $c_k$ would necessarily be zero, but as this is not the case by the assumption that $[a] \neq 0$, at least two of the $i_*([a_k])$ lie in the same path component.

        \bigskip

        We now consider the same exact sequence as in part (a), but this time a little farther up:
        \begin{align*}
          ...\to H_2(X,A) \xrightarrow{\partial} H_1(A) \xrightarrow{i_*} H_1(X) \xrightarrow{j_*} H_1(X,A) \to H_0(A) \to ...
        \end{align*}
        Suppose first that $H_1(X,A) = 0$. By the exactness of this sequence, $H_1(A) \xrightarrow{i_*} H_1(X)$ is necessarily surjective. Likewise, further down the sequence, $H_1(X,A) = 0$ implies that $H_0(A) \to H_0(X)$ is injective, and hence by what we have already shown, each path component of $X$ contains at most one path component of $A$.

        Now suppose that $H_1(A)\to H_1(X)$ is surjective and each path component of $X$ contains at most one path component of $A$. The latter condition tells us that $H_0(A)\to H_0(X)$ is injective, and hence the relevant exact sequence reads
        \begin{align*}
          ... H_2(X,A) \to H_1(A) \xrightarrow{i_*} H_1(X) \xrightarrow{j_*} H_1(X,A) \xrightarrow{\partial} H_0(A) \xrightarrow{i_*} H_0(X) \to ...
        \end{align*}
        and so by problem 2.1.15 we have that $H_1(X,A)$ is necessarily $0$. To repeat a portion of the argument in this problem, the surjectivity of the first $i_*$ implies that the kernel of $j_*$ is all of $H_1(X)$, and so $\img j_* = 0$ in $H_1(X,A)$ and hence $\ker \partial = 0$. However, the injectivity of the latter $i_*$ implies that $\img \partial = 0$, so in particular $\ker \partial = H_1(X,A)$. This then implies $H_1(X,A) = 0$.
    \end{enumerate}
  \end{prf}
  \prob[\textsc{Exercise 17.}]$ $
  \begin{enumerate}[(a)]
    \item Compute the homology groups $H_n(X,A)$ when $X$ is $S^2$ or $S^1\times S^1$ and $A$ is a finite set of points in $X$.
    \item Compute the groups $H_n(X,A)$ and $H_n(X,B)$ for $X$ a closed orientable surface of genus two with $A$ and $B$ the circles shown. [What are $X/A$ and $X/B$?]
  \end{enumerate}
  \begin{prf}$ $
    \begin{enumerate}[(a)]
    \item Recall that if $A$ is a finite set of points, say $|A| = m$ for instance, then
      \begin{align*}
        H_n(A) =
        \begin{cases}
          \bZ^{\oplus m} & n = 0 \\
          0 & n > 0
        \end{cases}.
      \end{align*}
      We also know that
      \begin{align*}
        H_n(S^2) =
        \begin{cases}
          \bZ &n = 0,2 \\
          0 & \text{else}
        \end{cases}
        \textbuff{2em}{and}
        H_n(S^1\times S^1) =
        \begin{cases}
          \bZ & n = 0,2 \\
          \bZ^{\oplus 2} & n = 1 \\
          0 & \text{else}
        \end{cases}.
      \end{align*}
      Set $X = S^2$ and let $A \subseteq X$ be a finite subset consisting of $m$ points. Then the long exact sequence of the pair $(X,A)$ reads
      \begin{align*}
        ... 0 \to H_2(A) \xrightarrow{i_*} H_2(X) \xrightarrow{j_*} H_2(X,A) \xrightarrow{\partial} H_1(A) \to ...
      \end{align*}
      in index $2$. In all higher indices we have that $H_k(A) = H_k(X) = 0$, so this same exact sequence tells us $H_k(X,A) = 0$ whenever $k \geq 3$. Since $H_2(A) = H_1(A) = 0$, we get an isomorphism $H_2(X) \cong H_2(X,A)$. In the index $1$ position we have
      \begin{align*}
        ... 0 \xrightarrow{i_*} H_1(X) \xrightarrow{j_*} H_1(X,A) \xrightarrow{\partial} H_0(A) \xrightarrow{i_*} H_0(X) \to ...
      \end{align*}
      consider the rightmost map above. The group $H_0(A)$ is the free abelian group on $m$ generators $g_1,...,g_m$ and are all mapped by $i_*$ to the single generator $g$ of $H_0(X) = \bZ$, hence
      \begin{align*}
        \ker i_* = \left\{\sum_{j=1}^{m-1} a_jg_j ~-~ g \sum_{j=1}^{m-1}a_j \midd a_1,...,a_j \in \bZ\right\} \cong \bZ^{m-1}.
      \end{align*}
      This means $\img \partial = \ker i_* \cong \bZ^{m-1}$. The long exact sequence gives us a short exact sequence
      \begin{align*}
        0 \to H_1(X) \to H_1(X,A) \to \img \partial = \ker i_* \to 0,
      \end{align*}
      and because $\ker i_*$ is free, this short exact sequence splits giving us
      \begin{align*}
        H_1(X,A) \cong H_1(X) \oplus \ker i_* = 0 \oplus \bZ^{m-1}.
      \end{align*}
      For the final homology group, we have
      \begin{align*}
        ... \to H_0(A) \xrightarrow{i_*} H_0(X)\xrightarrow{j_*} H_0(X,A) \to 0.
      \end{align*}
      We have already seen that $H_0(A)\xrightarrow{i_*}H_0(X)$ is surjective, so $\ker (H_0(X) \to H_0(X,A)) = \img i_* = H_0(X)$ and implies that $j_*$ is the trivial map. However, exactness at $H_0(X,A)$ means that $j_*$ is surjective, and hence $H_0(X,A) = 0$. To summarize,
      \begin{align*}
        H_0(S^2,A) =
        \begin{cases}
          H_2(S^2) \cong \bZ & n = 2 \\
          H_1(S^2) \oplus \bZ^{m-1} \cong \bZ^{m-1} & n = 1 \\
          0 & \text{else}
        \end{cases}.
      \end{align*}

      Notice that nothing in our above argument relied on the fact that $X = S^2$. The only space-specific properties we needed were triviality of $H_n(X)$ for all $n \geq 3$ to get $H_n(X,A) = 0$ and the path connectedness of $X$ in order to see $i_*:H_0(A) \to H_0(X)$ was path connected. Both of these properties still hold for $X = S^1\times S^1$, so by the same arguments as above,
      \begin{align*}
        H_0(S^1\times S^1,A) =
        \begin{cases}
          H_2(S^1\times S^1) \cong \bZ & n = 2 \\
          H_1(S^1\times S^1) \oplus \bZ^{m-1} \cong \bZ^{m+1} & n = 1 \\
          0 & \text{else}
        \end{cases}.
      \end{align*}

    \item 
    \end{enumerate}
  \end{prf}
\end{homework}
\tchap{Problems from 3.3}
\begin{homework}[e]
  \prob[\textsc{Exercise 3.}] Show that every covering space of an orientable manifolds is an orientable manifold.
  \prob[\textsc{Exercise 4.}] Given a covering space action of a group $G$ on an orientable manifold $M$ by orientation-preserving homeomorphisms, show that $M/G$ is also orientable.
  \prob[\textsc{Exercise 7.}] For a map $f:M\to N$ between connected closed orientable $n$-manifolds with fundamental classes $[M]$ $[N]$, the degree of $f$ is defined to be the integer $d$ such that $f_*([M]) = d[N]$, so the sign of the degree depends on the choice of fundamental classes. Show that for any connected closed orientable $n$-manifold $M$ there is a degree $1$ map $M\to S^n$.
  \prob[\textsc{Exercise 8.}] For a map $f:M\to N$ between connected closed orientable $n$-manifolds, suppose there is a ball $B \subseteq N$ such that $f^{-1}(B)$ is the disjoint union of balls $B_i$ each mapped homeomorphically by $f$ onto $B$. Show the degree of $f$ is $\sum_i \epsilon_i$ where $\epsilon_i$ is $+1$ or $-1$ according to whether $f:B_i\to B$ preserves or reverses local orientations induced from given fundamental classes $[M]$ and $[N]$.
\end{homework}
\tchap{Problems from 3.3}
\end{document}
