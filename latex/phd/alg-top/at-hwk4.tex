\documentclass[hidelinks,11pt,dvipsnames]{article}
% xcolor commonly causes option clashes, this fixes that
\PassOptionsToPackage{dvipsnames,table}{xcolor}
\usepackage[tmargin=1in, bmargin=1in, lmargin=0.8in, rmargin=1in]{geometry}

%%%%%%%%%%%%%%%%%%%%%%%%%%%%%%%%%%%%%%%%%%%%%%%%%%%%%%%%%%%%%%%%%%%%
%%% For inkscape-figures
%%% Assumes the following directory structure:
%%% master.tex
%%% figures/
%%%     figure1.pdf_tex
%%%     figure1.svg
%%%     figure1.pdf
%%%%%%%%%%%%%%%%%%%%%%%%%%%%%%%%%%%%%%%%%%%%%%%%%%%%%%%%%%%%%%%%%%%%
%\usepackage{import}
\usepackage{pdfpages}
\usepackage{transparent}

\newcommand{\incfig}[2][1]{%
    \def\svgwidth{#1\columnwidth}
    \import{./figures/}{#2.pdf_tex}
}

\pdfsuppresswarningpagegroup=1

% enable synctex for inverse search, whatever synctex is
\synctex=1
\usepackage{float,macrosabound,homework,theorem-env}
\usepackage{microtype}


% font stuff
\usepackage{sectsty}
\allsectionsfont{\sffamily}
\linespread{1.1}

% bibtex stuff
\usepackage[backend=biber,style=alphabetic,sorting=anyt]{biblatex}
\addbibresource{main.bib}

% colored text shortcuts
\newcommand{\blue}[1]{\color{MidnightBlue}{#1}}
\newcommand{\red}[1]{\textcolor{Mahogany}{#1}}
\newcommand{\green}[1]{\textcolor{ForestGreen}{#1}}


% use mathptmx pkg while using default mathcal font
\DeclareMathAlphabet{\mathcal}{OMS}{cmsy}{m}{n}

% fixes the positioning of subscripts in $$ $$
\renewcommand{\det}{\operatorname{det}}

\usetikzlibrary{positioning, arrows.meta}
\newcommand{\here}[2]{\tikz[remember picture]{\node[inner sep=0](#2){#1}}}

%%%%%%%%%%%%%%%%%%%%%%%%%%%%%%%%%%%%%%%%%%%%%%%%%%%%%%%%%%%%%%%%%%%%%
%%% Entry Counter
%%%%%%%%%%%%%%%%%%%%%%%%%%%%%%%%%%%%%%%%%%%%%%%%%%%%%%%%%%%%%%%%%%%%%
\newcounter{entry-counter}
\newcommand{\entry}[1]
{
	\addtocounter{entry-counter}{1}
    \tchap{Entry \arabic{entry-counter}}
	%\addcontentsline{toc}{section}{Entry \arabic{entry-counter}: #1}
	\vspace{-1.5em}
    \begin{center}
		\small \emph{Written: #1}
    \end{center}
}

\usepackage{titling}
\renewcommand\maketitlehooka{\null\mbox{}\vfill}
\renewcommand\maketitlehookd{\vfill\null}


\usepackage{caption}
\usepackage{tikz}
\usetikzlibrary{positioning,calc,intersections,through,backgrounds, shapes.geometric, decorations.markings,arrows}

\def\sset{\subseteq}
\def\iso{\cong}
\def\gend#1{\langle #1\rangle}

\newcommand{\rightoverleftarrow}{%
  \mathrel{\vcenter{\mathsurround0pt
    \ialign{##\crcr
      \noalign{\nointerlineskip}$\longrightarrow$\crcr
      \noalign{\nointerlineskip}$\longleftarrow$\crcr
    }%
  }}%
}

\newcommand\makesphere{} % just for safety
\def\makesphere(#1)(#2)[#3][#4]{%
  % Synopsis
  % \makesphere[draw options](center)(initial angle:final angle:radius)
  \shade[ball color = #3, opacity = #4] #1 circle (#2);
  \draw #1 circle (#2);
  \draw ($#1 - (#2, 0)$) arc (180:360:#2 and 3*#2/10);
  \draw[dashed] ($#1 + (#2, 0)$) arc (0:180:#2 and 3*#2/10);
}
% same thing as makesphere but places white background behind
\newcommand\altmakesphere{} % just for safety
\def\altmakesphere(#1)(#2)(#3)[#4][#5]{%
  % Synopsis
  % \makesphere[draw options](center)(initial angle:final angle:radius)
  \draw [fill=white!30] #1 circle (#2);
  \shade[ball color = #4, opacity = #5] #1 circle (#2);
  \draw #1 circle (#2);
  \draw ($#1 - (#2, 0)$) arc (180:360:#2 and 3*#2/10);
  \draw[dashed] ($#1 + (#2, 0)$) arc (0:180:#2 and 3*#2/10);
  \node at #1 {#3};
}

\begin{document}
\pagestyle{empty}
	\LARGE
\begin{center}
	Algebraic Topo logy Homework 2 \\
	\Large
	Isaac Martin \\
    Last compiled \today
\end{center}
\normalsize
\vspace{-4mm}
\hru

\tchap{Problems from 1.2}
\begin{homework}[e]
  \prob[\textsc{Exercise 1.9}]
  \prob[\textsc{Exercise 1.10}]
  \prob[\textsc{Exercise 1.12}]
  \prob[\textsc{Exercise 1.14}]
  \prob[\textsc{Exercise 1.21}] Show that the join $X*Y$ of two nonempty space $X$ and $Y$ is simply-connected if $X$ is path-connected.
\begin{proof}
    Let $(x,y,t),(u,v,s) \in X * Y$. We regard $X * Y$ as the space $X \times Y \times I / \sim$ where $(x,y_1,0) \sim (x,y_2,0)$ and $(x_1,y,1) \sim (x_2,y,1)$.
    
    We first show that $X * Y$ is path connected.  There exists a path from $(x,y,t)$ to $(x,y,0)$, call it $\gamma_1$. Because $X$ is path connected, there exists a path from $(x,y,0)$ to $(u,y,0)$ which is identified with $(u,v,0)$ in $X * Y$. We call the path from $(x,y,0)$ to $(u,v,0)$ $\gamma_2$. Finally, call the path from $(u,v,0)$ to $(u,v,s)$ $\gamma_3$. The path $\gamma_3 \gamma_2 \gamma_1$ starts at $(x,y,t)$ and ends at $(u,v,s)$. Therefore, there is path connecting any two points in $X*Y$, so $X*Y$ is path connected.  
    
    In order to apply Van Kampen's theorem, we must make a choice of open sets with which to express $X * Y$. These must be open sets whose intersection has a familiar fundamental group. Let $A = X \times Y \times [0,1) / \sim$ and $B = X \times Y \times (0,1] / \sim$. These sets are open, and their intersection is precisely $X \times Y \times (0,1)$ \emph{without} the equivalence relation. Since we can continuously retract the copies of $I$ extending from $X$ down onto $X$, we can deformation retract $A$ onto $X$. Similarly, we can deformation retract $B$ onto $Y$. This is similar to the process in problem 0.6(a) in Hatcher. Finally, by retracting either end of the intervals onto the point \{1/2\}, we can deformation retract $A \cap B$ to $X \times Y \times \{\frac{1}{2}\}$, which is homeomorphic to $X \times Y$. 
    
    This means $\pi_1(A \cap B) \approx \pi_1(X) \times \pi_1(Y)$, $\pi_1(A) \approx \pi_1(X)$, and $\pi_1(B) \approx \pi_1(Y)$. Van Kampen tells us that 
    \begin{equation*}
        \pi_1(X*Y) \approx \pi_1(A) * \pi_1(B) / N 
    \end{equation*}
    where $N$ is generated by elements of the form $\iota_A(\omega)\iota_B(\omega)^{-1}$ where $\iota_A: \pi_1(A \cap B) \hookrightarrow \pi_1(A)$ and 
    
    \noindent
    $\iota_B: \pi_1(A \cap B) \hookrightarrow \pi_1(B)$ are the homomorphisms induced by the inclusions. However, as we concluded above, $\pi_1(A \cap B) \approx \pi_1(X) \times \pi_1(Y)$. This means that $\iota_A$ and $\iota_B$ are actually surjective projections, and therefore the elements of the form $\iota_A(\omega)\iota_B(\omega)^{-1}$ are equivalently $ab^-1$, $a \in \pi_1(A)$ and $b \in \pi_1(B)$ . By modding out by $N$ we are thus actually identifying \emph{all} points. Thus, 
    
    \begin{equation*}
        \pi_1(X * Y) \approx \pi_1(A) * \pi_1(B) / N = 1
    \end{equation*}
\end{proof}
\end{homework}
\end{document}
