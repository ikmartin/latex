\documentclass[hidelinks,11pt,dvipsnames]{article}
% xcolor commonly causes option clashes, this fixes that
\PassOptionsToPackage{dvipsnames,table}{xcolor}
\usepackage[tmargin=1in, bmargin=1in, lmargin=0.8in, rmargin=1in]{geometry}

%%%%%%%%%%%%%%%%%%%%%%%%%%%%%%%%%%%%%%%%%%%%%%%%%%%%%%%%%%%%%%%%%%%%
%%% For inkscape-figures
%%% Assumes the following directory structure:
%%% master.tex
%%% figures/
%%%     figure1.pdf_tex
%%%     figure1.svg
%%%     figure1.pdf
%%%%%%%%%%%%%%%%%%%%%%%%%%%%%%%%%%%%%%%%%%%%%%%%%%%%%%%%%%%%%%%%%%%%
%\usepackage{import}
\usepackage{pdfpages}
\usepackage{transparent}

\newcommand{\incfig}[2][1]{%
    \def\svgwidth{#1\columnwidth}
    \import{./figures/}{#2.pdf_tex}
}

\pdfsuppresswarningpagegroup=1

% enable synctex for inverse search, whatever synctex is
\synctex=1
\usepackage{float,macrosabound,homework,theorem-env}
\usepackage{microtype}


% font stuff
\usepackage{sectsty}
\allsectionsfont{\sffamily}
\linespread{1.1}

% bibtex stuff
\usepackage[backend=biber,style=alphabetic,sorting=anyt]{biblatex}
\addbibresource{main.bib}

% colored text shortcuts
\newcommand{\blue}[1]{\color{MidnightBlue}{#1}}
\newcommand{\red}[1]{\textcolor{Mahogany}{#1}}
\newcommand{\green}[1]{\textcolor{ForestGreen}{#1}}


% use mathptmx pkg while using default mathcal font
\DeclareMathAlphabet{\mathcal}{OMS}{cmsy}{m}{n}

% fixes the positioning of subscripts in $$ $$
\renewcommand{\det}{\operatorname{det}}

\usetikzlibrary{positioning, arrows.meta}
\newcommand{\here}[2]{\tikz[remember picture]{\node[inner sep=0](#2){#1}}}

%%%%%%%%%%%%%%%%%%%%%%%%%%%%%%%%%%%%%%%%%%%%%%%%%%%%%%%%%%%%%%%%%%%%%
%%% Entry Counter
%%%%%%%%%%%%%%%%%%%%%%%%%%%%%%%%%%%%%%%%%%%%%%%%%%%%%%%%%%%%%%%%%%%%%
\newcounter{entry-counter}
\newcommand{\entry}[1]
{
	\addtocounter{entry-counter}{1}
    \tchap{Entry \arabic{entry-counter}}
	%\addcontentsline{toc}{section}{Entry \arabic{entry-counter}: #1}
	\vspace{-1.5em}
    \begin{center}
		\small \emph{Written: #1}
    \end{center}
}

\usepackage{titling}
\renewcommand\maketitlehooka{\null\mbox{}\vfill}
\renewcommand\maketitlehookd{\vfill\null}


\usepackage{capt-of}
\usepackage{tikz}
\usetikzlibrary{positioning,calc,intersections,through,backgrounds, shapes.geometric, decorations.markings,arrows}

\def\sset{\subseteq}
\def\iso{\cong}
\def\gend#1{\langle #1\rangle}

\newcommand{\rightoverleftarrow}{%
  \mathrel{\vcenter{\mathsurround0pt
    \ialign{##\crcr
      \noalign{\nointerlineskip}$\longrightarrow$\crcr
      \noalign{\nointerlineskip}$\longleftarrow$\crcr
    }%
  }}%
}

\newcommand\makesphere{} % just for safety
\def\makesphere(#1)(#2)[#3][#4]{%
  % Synopsis
  % \makesphere[draw options](center)(initial angle:final angle:radius)
  \shade[ball color = #3, opacity = #4] #1 circle (#2);
  \draw #1 circle (#2);
  \draw ($#1 - (#2, 0)$) arc (180:360:#2 and 3*#2/10);
  \draw[dashed] ($#1 + (#2, 0)$) arc (0:180:#2 and 3*#2/10);
}
% same thing as makesphere but places white background behind
\newcommand\altmakesphere{} % just for safety
\def\altmakesphere(#1)(#2)(#3)[#4][#5]{%
  % Synopsis
  % \make sphere[draw options](center)(initial angle:final angle:radius)
  \draw [fill=white!30] #1 circle (#2);
  \shade[ball color = #4, opacity = #5] #1 circle (#2);
  \draw #1 circle (#2);
  \draw ($#1 - (#2, 0)$) arc (180:360:#2 and 3*#2/10);
  \draw[dashed] ($#1 + (#2, 0)$) arc (0:180:#2 and 3*#2/10);
  \node at #1 {#3};
}

\begin{document}
% set section number to 1
% fixes theorem numbering without need to have a section title
\setcounter{section}{1}

\pagestyle{empty}
	\LARGE
\begin{center}
	Algebraic Topology Homework 9 \\
	\Large
	Isaac Martin \\
    Last compiled \today
\end{center}
\normalsize
\vspace{-4mm}
\hru

\tchap{Problems from 2.1}

\begin{homework}[e]
  \prob[\textsc{Exercise 22.}] Prove by induction on dimension the following facts about the homology of finite-dimensional CW complex $X$, using the observation that $X^n/X^{n-1}$ is a wedge sum of $n$-spheres:
  \begin{enumerate}[(a)]
    \item If $X$ has dimension $n$ then $H_i(X) = 0$ for $i > n$ and $H_n(X)$ is free.
    \item $H_n(X)$ is free with basis in bijective correspondence with the $n$-cells if there are no cells of dimension $n-1$ or $n+1$
    \item If $X$ has $k$ $n$-cells, then $H_n(X)$ is generated by at most $k$-elements.
  \end{enumerate}
\end{homework}
\tchap{Problems from 2.2}
\begin{homework}[e]
  \prob[\textsc{Exercise 5.}] Show that any two reflections of $S^n$ across different $n$-dimensional hyperplanes are homotopic, in fact homotopic through reflections. [The linear algebra formula for a reflection in terms of inner products may be helpful.]
  \begin{prf}
    The linear algebra formula Hatcher alludes to is reflection in the direction of $u$ given by $f_u(x) = x - 2u\cdot\frac{x\cdot u}{u^2}$. It is a map on $\bR^{n+1}$ which reflects a point $x$ across the hyper plane through the origin whose normal vector is $0 \neq u \in \bR^{n+1}$. Notice that for any vector $0 \neq u$, the reflection $f_u$ in the direction of $u$
    \begin{enumerate}[(1)]
      \item negates $u$
        \begin{align*}
          f_u(u) = u - 2u \frac{u\cdot u}{u^2} = u - 2u = -u,
        \end{align*}
      \item fixes the hyper plane $x\cdot u = 0$
        \begin{align*}
          x \cdot u = 0 \implies f_u(x) = x - 2u \frac{x \cdot u}{u^2} = x - 0 = x,
        \end{align*}
      \item is an involution
        \begin{align*}
          f_u\left(f_u(x)\right) &= \left(x - 2u \frac{x\cdot u}{u^2}\right) - 2u \frac{\left(x - 2u \frac{x\cdot u}{u^2}\right)\cdot u}{u^2} \\
            &= x - 2u \frac{x\cdot u}{u^2} - 2u \frac{x \cdot u}{u^2} + 2u\frac{2u^2\frac{x\cdot u}{u^2}}{u^2} \\
            &= x - 4u \frac{x\cdot u}{u^2} + 4 \frac{x\cdot u}{u^2}\\
            &= x
        \end{align*}
      \item and is a norm-preserving isometry (is ``norm preserving'' redundant?)
        \begin{align*}
          \|f_u(x)\|^2 &= \left(x - 2u \frac{x\cdot u}{u^2}\right)\cdot \left(x - 2u \frac{x\cdot u}{u^2}\right) \\
                       &= x^2 - 4x\cdot u \frac{x\cdot u}{u^2} + \frac{4(x\cdot u)^2}{u^2} \\
                       &= x^2 = \|x\|^2,
        \end{align*}
    \end{enumerate}
    which should be enough to convince us that this is indeed a reflection. Because $f_u$ is norm preserving, it is a continuous map which maps sends $S^n$ to $S^n$, and for notational convenience we will redefine $f_u$ to be the restriction $f_u|_{S^n}:S^n \to S^n$.

    Consider some other vector $0 \neq v \in \bR^{n+1}$ and suppose that the line between $v$ and $u$ does not contain the origin. Let $\gamma:I \to \bR^{n+1}$ be the linear interpolation from $u$ to $v$, i.e. the map $\gamma(t) = u\cdot t - (1 - t)\cdot v$. Then the map $F:S^n\times I \to S^n$ defined $F_t(x) = f_{\gamma(t)}(x)$ is continuous and satisfies $F_0(x) = f_v(x)$ and $F_1(x) = f_u(x)$; hence, it is a homotopy between $f_u$ and $f_v$ comprised itself entirely of reflection maps.

    If the line between $v$ and $u$ does contain the origin, then choose some other nonzero point $w \in \bR^{n+1}$ which is not on the linear subspace spanned by $u$ and $v$. By what we have already shown, $f_u \simeq f_w$ and $f_v \simeq f_w$, and since homotopy equivalence is an equivalence relation, $f_u \simeq f_v$.
  \end{prf}
  \prob[\textsc{Exercise 7.}] For an invertible linear transformation $f:\bR^n\to \bR^n$ show that the induced map on $H_n (\bR^n, \bR^n \setminus \{0\}) \cong \tilH_{n-1}(\bR^n \setminus \{0\}) \cong \bZ$ is $\id$ or $-\id$ according to whether the determinant of $f$ is positive or negative. [Use Gaussian elimination to show that the matrix of $f$ can be joined by a path of invertible matrices to a diagonal matrix with $\pm_1$'s on the diagonal.]
  \prob[\textsc{Exercise 8.}] A polynomial $f(z)$ with complex coefficients, viewed as a map $\bC\to \bC$, can always be extended to a continuous map of one-point compactifications $\hatf:S^2 \to S^2$. Show that the degree of $\hatf$ equals the degree of $f$ as a  polynomial. Show also that the local degree of $\hatf$ at a root of $f$ is the multiplicity of the root.
  \prob[\textsc{Exercise 12.}] Show that the quotient map $S^1\times S^1 \to S^2$ collapsing the subspace $S^1 \vee S^1$ to a point is not nullhomotopic by showing that it induces an isomorphism on $H_2$. On the other hand, show via covering spaces that any map $S^2 \to S^1\vee S^1$ is nullhomotopic.
\end{homework}
\tchap{Problems from 2.B}
\begin{homework}[e]
  \prob[\textsc{Exercise 1.}]
  \prob[\textsc{Exercise 2.}]
\end{homework}
\end{document}
