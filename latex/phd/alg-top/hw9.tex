\documentclass{amsart}
\usepackage{amsmath}
\usepackage{tikz}

\theoremstyle{remark}
\newtheorem{problem}{Problem}
\newtheorem*{remarks}{Remarks}
\raggedbottom

\def\R{\mathbb{R}}
\def\Z{\mathbb{Z}}
\let\iso\cong

\begin{document}

{\bf Homework 9}: due Friday October 27, 2022

Algebraic Topology, M382C, U.T. Austin, Daniel Allcock

\bigskip
Hatcher section 2.1 (p. 131):
20, 21, 23, 29

(on 29, compute the homology groups using singular
homology, eg via Mayer-Vietoris.)

Hatcher section 2.2 (p. 155):
28, 31, 36

(on 36, you can use Mayer-Vietoris, rather than following
the relative homology argument that Hatcher sketches for you.
Also, some people prefer to 
understand the statement as
``$H_*(X\times S^n)\iso H_*(X)\oplus H_{*-n}(X)$''.
This is not very different, indeed it is 
literally the statement of the problem with $*$ in place
of $i$.  But it communicates more emphatically that to get the
homology of $X\times S^n$, you take sum two copies of $H_*(X)$,
with one of them ``slid up'' by increasing subscripts by~$n$.
This problem, and its generalization to $H_*(X\times Y)$, called the
K\"unneth formula,
cannot be stated cleanly in terms of reduced homology. 
Since suspensions are so much cleaner using reduced homology,
we keep both theories.)



\end{document}
