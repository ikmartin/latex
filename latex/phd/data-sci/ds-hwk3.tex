\documentclass[hidelinks,11pt,dvipsnames]{article}
% xcolor commonly causes option clashes, this fixes that
\PassOptionsToPackage{dvipsnames,table}{xcolor}
\usepackage[tmargin=1in, bmargin=1in, lmargin=0.8in, rmargin=1in]{geometry}

%%%%%%%%%%%%%%%%%%%%%%%%%%%%%%%%%%%%%%%%%%%%%%%%%%%%%%%%%%%%%%%%%%%%
%%% For inkscape-figures
%%% Assumes the following directory structure:
%%% master.tex
%%% figures/
%%%     figure1.pdf_tex
%%%     figure1.svg
%%%     figure1.pdf
%%%%%%%%%%%%%%%%%%%%%%%%%%%%%%%%%%%%%%%%%%%%%%%%%%%%%%%%%%%%%%%%%%%%
%\usepackage{import}
\usepackage{pdfpages}
\usepackage{transparent}

\newcommand{\incfig}[2][1]{%
    \def\svgwidth{#1\columnwidth}
    \import{./figures/}{#2.pdf_tex}
}

\pdfsuppresswarningpagegroup=1

% enable synctex for inverse search, whatever synctex is
\synctex=1
\usepackage{float,macrosabound,homework,theorem-env}
\usepackage{microtype}


% font stuff
\usepackage{sectsty}
\allsectionsfont{\sffamily}
\linespread{1.1}

% bibtex stuff
\usepackage[backend=biber,style=alphabetic,sorting=anyt]{biblatex}
\addbibresource{main.bib}

% colored text shortcuts
\newcommand{\blue}[1]{\color{MidnightBlue}{#1}}
\newcommand{\red}[1]{\textcolor{Mahogany}{#1}}
\newcommand{\green}[1]{\textcolor{ForestGreen}{#1}}


% use mathptmx pkg while using default mathcal font
\DeclareMathAlphabet{\mathcal}{OMS}{cmsy}{m}{n}

% fixes the positioning of subscripts in $$ $$
\renewcommand{\det}{\operatorname{det}}

\usetikzlibrary{positioning, arrows.meta}
\newcommand{\here}[2]{\tikz[remember picture]{\node[inner sep=0](#2){#1}}}

%%%%%%%%%%%%%%%%%%%%%%%%%%%%%%%%%%%%%%%%%%%%%%%%%%%%%%%%%%%%%%%%%%%%%
%%% Entry Counter
%%%%%%%%%%%%%%%%%%%%%%%%%%%%%%%%%%%%%%%%%%%%%%%%%%%%%%%%%%%%%%%%%%%%%
\newcounter{entry-counter}
\newcommand{\entry}[1]
{
	\addtocounter{entry-counter}{1}
    \tchap{Entry \arabic{entry-counter}}
	%\addcontentsline{toc}{section}{Entry \arabic{entry-counter}: #1}
	\vspace{-1.5em}
    \begin{center}
		\small \emph{Written: #1}
    \end{center}
}

\usepackage{titling}
\renewcommand\maketitlehooka{\null\mbox{}\vfill}
\renewcommand\maketitlehookd{\vfill\null}


\usepackage{capt-of}
\usepackage{tikz}
\usepackage{listings}
\usetikzlibrary{positioning,calc,intersections,through,backgrounds, shapes.geometric, decorations.markings,arrows}

\def\sset{\subseteq}
\def\iso{\cong}
\def\gend#1{\langle #1\rangle}

\newcommand{\rightoverleftarrow}{%
  \mathrel{\vcenter{\mathsurround0pt
    \ialign{##\crcr
      \noalign{\nointerlineskip}$\longrightarrow$\crcr
      \noalign{\nointerlineskip}$\longleftarrow$\crcr
    }%
  }}%
}

\newcommand\makesphere{} % just for safety
\def\makesphere(#1)(#2)[#3][#4]{%
  % Synopsis
  % \makesphere[draw options](center)(initial angle:final angle:radius)
  \shade[ball color = #3, opacity = #4] #1 circle (#2);
  \draw #1 circle (#2);
  \draw ($#1 - (#2, 0)$) arc (180:360:#2 and 3*#2/10);
  \draw[dashed] ($#1 + (#2, 0)$) arc (0:180:#2 and 3*#2/10);
}
% same thing as makesphere but places white background behind
\newcommand\altmakesphere{} % just for safety
\def\altmakesphere(#1)(#2)(#3)[#4][#5]{%
  % Synopsis
  % \make sphere[draw options](center)(initial angle:final angle:radius)
  \draw [fill=white!30] #1 circle (#2);
  \shade[ball color = #4, opacity = #5] #1 circle (#2);
  \draw #1 circle (#2);
  \draw ($#1 - (#2, 0)$) arc (180:360:#2 and 3*#2/10);
  \draw[dashed] ($#1 + (#2, 0)$) arc (0:180:#2 and 3*#2/10);
  \node at #1 {#3};
}

\begin{document}
% set section number to 1
% fixes theorem numbering without need to have a section title
\setcounter{section}{1}

\pagestyle{empty}
	\LARGE
\begin{center}
	Foundations of Data Science and Machine Learning -- \emph{Homework 3}\\
	\Large
	Isaac Martin \\
    Last compiled \today
\end{center}
\normalsize
\vspace{-4mm}
\hru

\begin{homework}[e]
  \prob[\textsc{Exercise 2.}] Let $\bfX\in \bR^{n\times d}$ be a matrix whose $n$ rows are the data points $\bfx_1,...,\bfx_n \in \bR^d$, and let $\chi = \{\bfx_1,...,\bfx_n\}$. Consider the $k$-means optimization problem: find a partition $C_1,...,C_k$ which minimizes, among all partitions of $[n]$ into $k$ subsets,
  \begin{align*}
    \operatorname{cost}_{\chi}(C_1,...,C_k):= \sum_{j=1}^k\sum_{i \in C_j} \left\|\bfx_i - \frac{1}{|C_j|}\sum_{i\in C_j} \bfx_i\right\|^2_2.
  \end{align*}
  \begin{enumerate}[(a)]
    \item Suppose that $\Phi:\bR^d \to \bR^r$ with $r = \cO(\log(n)/\epsilon^2)$ is a random i.i.d. spherical Gaussian projection matrix and thus satisfies the JL lemma. Consider the projected points $\bfy_j = \Phi\bfx_j\in \bR^r$ and suppose $\tilC_1,...,\tilC_k$ are an optimal set of $k$-means clusters for the data points $\cY = \{\bfy_1,....,\bfy_n\}$. That is,
    \begin{align*}
      \operatorname{cost}_{\cY}(\tilC_1,...,\tilC_n) = \sum_{j=1}^k\sum_{i\in \tilC_j}
    \end{align*}
  \end{enumerate}
  \prob[\textsc{Exercise 3.}] Find the mapping $\varphi(\bfx)$ that gives rise to the polynomial kernel \begin{align*}
    K(\bfx, \bfy) = (x_1x_2 + y_1y_2)^2.
  \end{align*}
  \begin{prf}
    Consider the map $\varphi:\bR^2 \to \bR^3$ defined $\varphi(x_1,x_2) = (x_1^2, x_2^2, \sqrt{2}x_1x_2)$. Interestingly, this is similar to the map one considers from a polynomial ring $R[x_1,x_2]$ to its $2^{\text{nd}}$ Veronese subring $R[x_1^2,x_1x_2,x_2^2]$. We then have that
    \begin{align*}
      \varphi(\bfx)\cdot \varphi(\bfy)^T
        &= (x_1^2,x_2^2,\sqrt{2}x_1x_2) \cdot (y_1^2,y_2^2, \sqrt{2}y_1y_2) \\
        &= x_1^2y_1^2 + x_2^2y_2^2 + 2x_1y_1x_2y_2 \\
        &= (x_1y_1 + x_2y_2)^2,
    \end{align*}
    hence $\varphi$ gives rise to the desired kernel.
  \end{prf}
\end{homework}
\end{document}
