\documentclass[hidelinks,11pt,dvipsnames]{article}
% xcolor commonly causes option clashes, this fixes that
\PassOptionsToPackage{dvipsnames,table}{xcolor}
\usepackage[tmargin=1in, bmargin=1in, lmargin=0.8in, rmargin=1in]{geometry}

%%%%%%%%%%%%%%%%%%%%%%%%%%%%%%%%%%%%%%%%%%%%%%%%%%%%%%%%%%%%%%%%%%%%
%%% For inkscape-figures
%%% Assumes the following directory structure:
%%% master.tex
%%% figures/
%%%     figure1.pdf_tex
%%%     figure1.svg
%%%     figure1.pdf
%%%%%%%%%%%%%%%%%%%%%%%%%%%%%%%%%%%%%%%%%%%%%%%%%%%%%%%%%%%%%%%%%%%%
%\usepackage{import}
\usepackage{pdfpages}
\usepackage{transparent}

\newcommand{\incfig}[2][1]{%
    \def\svgwidth{#1\columnwidth}
    \import{./figures/}{#2.pdf_tex}
}

\pdfsuppresswarningpagegroup=1

% enable synctex for inverse search, whatever synctex is
\synctex=1
\usepackage{float,macrosabound,homework,theorem-env}
\usepackage{microtype}


% font stuff
\usepackage{sectsty}
\allsectionsfont{\sffamily}
\linespread{1.1}

% bibtex stuff
\usepackage[backend=biber,style=alphabetic,sorting=anyt]{biblatex}
\addbibresource{main.bib}

% colored text shortcuts
\newcommand{\blue}[1]{\color{MidnightBlue}{#1}}
\newcommand{\red}[1]{\textcolor{Mahogany}{#1}}
\newcommand{\green}[1]{\textcolor{ForestGreen}{#1}}


% use mathptmx pkg while using default mathcal font
\DeclareMathAlphabet{\mathcal}{OMS}{cmsy}{m}{n}

% fixes the positioning of subscripts in $$ $$
\renewcommand{\det}{\operatorname{det}}

\usetikzlibrary{positioning, arrows.meta}
\newcommand{\here}[2]{\tikz[remember picture]{\node[inner sep=0](#2){#1}}}

%%%%%%%%%%%%%%%%%%%%%%%%%%%%%%%%%%%%%%%%%%%%%%%%%%%%%%%%%%%%%%%%%%%%%
%%% Entry Counter
%%%%%%%%%%%%%%%%%%%%%%%%%%%%%%%%%%%%%%%%%%%%%%%%%%%%%%%%%%%%%%%%%%%%%
\newcounter{entry-counter}
\newcommand{\entry}[1]
{
	\addtocounter{entry-counter}{1}
    \tchap{Entry \arabic{entry-counter}}
	%\addcontentsline{toc}{section}{Entry \arabic{entry-counter}: #1}
	\vspace{-1.5em}
    \begin{center}
		\small \emph{Written: #1}
    \end{center}
}

\usepackage{titling}
\renewcommand\maketitlehooka{\null\mbox{}\vfill}
\renewcommand\maketitlehookd{\vfill\null}


\usepackage{capt-of}
\usepackage{tikz}
\usepackage{listings}
\usetikzlibrary{positioning,calc,intersections,through,backgrounds, shapes.geometric, decorations.markings,arrows}

\def\sset{\subseteq}
\def\iso{\cong}
\def\gend#1{\langle #1\rangle}

\newcommand{\rightoverleftarrow}{%
  \mathrel{\vcenter{\mathsurround0pt
    \ialign{##\crcr
      \noalign{\nointerlineskip}$\longrightarrow$\crcr
      \noalign{\nointerlineskip}$\longleftarrow$\crcr
    }%
  }}%
}

\newcommand\makesphere{} % just for safety
\def\makesphere(#1)(#2)[#3][#4]{%
  % Synopsis
  % \makesphere[draw options](center)(initial angle:final angle:radius)
  \shade[ball color = #3, opacity = #4] #1 circle (#2);
  \draw #1 circle (#2);
  \draw ($#1 - (#2, 0)$) arc (180:360:#2 and 3*#2/10);
  \draw[dashed] ($#1 + (#2, 0)$) arc (0:180:#2 and 3*#2/10);
}
% same thing as makesphere but places white background behind
\newcommand\altmakesphere{} % just for safety
\def\altmakesphere(#1)(#2)(#3)[#4][#5]{%
  % Synopsis
  % \make sphere[draw options](center)(initial angle:final angle:radius)
  \draw [fill=white!30] #1 circle (#2);
  \shade[ball color = #4, opacity = #5] #1 circle (#2);
  \draw #1 circle (#2);
  \draw ($#1 - (#2, 0)$) arc (180:360:#2 and 3*#2/10);
  \draw[dashed] ($#1 + (#2, 0)$) arc (0:180:#2 and 3*#2/10);
  \node at #1 {#3};
}

\begin{document}
% set section number to 1
% fixes theorem numbering without need to have a section title
\setcounter{section}{1}

\pagestyle{empty}
	\LARGE
\begin{center}
	Foundations of Data Science and Machine Learning -- \emph{Homework 5}\\
	\Large
	Isaac Martin \\
    Last compiled \today
\end{center}
\normalsize
\vspace{-4mm}
\hru

\begin{homework}[e]
  \prob Suppose $\bfA$ is a $n\times d$ full-rank matrix, with $n < d$, and fix $\bfb\in \bR^n$. Consider minimizing the least squares objective $F(\bfx) = \|\bfA\bfx - \bfb\|^2_2$ Note that in this setting, the solution space $\cS = \{x : \bfA\bfx = \bfb\}$ is an affine subspace of $\bR^d$. We use gradient descent with constant step-size:
  \begin{align*}
    \bfx = \bfx_{k-1} - \eta \nabla F(\bfx_{k-1}).
  \end{align*}
  \begin{enumerate}[(a)]
    \item Give an upper bound for the step-size $\eta$ such that gradient descent is guaranteed to converge for $\eta$ below this threshold.
    \item Suppose that gradient descent is initialized at $\bfx_0 = 0$. Show that when gradient descent converges, it must converge to the least-norm solution $\bfx^* = \operatorname{argmin}_{\bfx \in \cS} \|\bfx\|^2_2$.
  \end{enumerate}
  \begin{prf}$ $
    \begin{enumerate}[(a)]
      \item In class, we showed that if $\nabla F$ is Lipschitz with Lipschitz constant $L$, then choosing $\eta = 1/L$ guarantees the convergence of gradient descent. In particular, any $\eta \leq 1/L$ will guarantee the convergence of gradient descent, so we need only find $L$. We have
        \begin{align*}
        \|\nabla F(\bfx) - \nabla F(\bfy)\|_2 
          &= \|2\bfA^\top(\bfA \bfx - \bfb) - 2\bfA^\top(\bfA \bfy - \bfb)\|_2 \\
          &= \|2\bfA^\top \bfA(\bfx - \bfy)\|_2 \\
          &\leq 2\|\bfA^\top \bfA\|\cdot \|\bfx - \bfy\|_2
        \end{align*}
        where $\|\cdot\|$ denotes the operator norm. Hence choosing $\eta \leq (2\|\bfA^\top\bfA\|)^{-1}$ will guarantee the convergence of gradient descent for any initialization.

      \item Let us first prove the hint, namely, that if $\bfx \in \img\bfA^\top$ (i.e. if $\bfx$ is in the rowspan of $\bfA$) then so is $A\bfx - \eta \nabla F(\bfx)$. Suppose then that $\bfx = \bfA^\top \bfu$ for some $\bfu \in \bR^n$. Then
        \begin{align*}
          \bfy = \bfA \cdot \bfA^\top \bfu - \nabla F(\bfA^\top \bfu) 
            &=\bfA \cdot \bfA^\top \bfu - \nabla 2A^\top(\bfA \bfA^\top \bfu - \bfu) \\
            &= \bfA^\top u - 2\eta \bfA^\top \bfA \bfA^\top \bfu - 2\bfA^\top \bfb \\
            &= \bfA^\top\left(\bfu - 2\eta \bfA\bfA^\top \bfu - 2\bfb\right) \implies \bfy \in \img \bfA^\top.
        \end{align*}
        Because the update rule is continuous and the image of affine linear transformations is closed, we can further conclude that an initialization $\bfx_0$ is in the rowspan of $\bfA$ if and only if the point $\bfx^*$ it converges to is in the rowspan of $\bfA$, provided $\eta$ is chosen small enough to guarantee convergence.

        \bigskip

        Now we prove that any two points initialized in the rowspan of $\bfA$ converge to the same point. Take $\bfx_0 = \bfA^\top \bfu_0$ to be an initialization for some $\bfu_0 \in \bR^n$. By what we have previously shown, $\bfx^* = \bfA^\top \bfu^*$ for some $\bfu^* \in \bR^n$, supposing we have chosen $\eta$ to be small enough. Since $\bfx^*$ is a stable point of the update rule, we get that $\nabla F(\bfx^*) = 0$ and hence
        \begin{align*}
          \nabla F(\bfA^\top \bfu^*) = 2\bfA^\top(\bfA\bfA^\top &\bfu^* - \bfb) = 0 \\
            &\implies \bfA^\top(\bfA\bfA^\top \bfu^* - \bfb) = 0\\
            &\implies \bfA^\top \bfu^* - \bfb = 0
        \end{align*}
        since $\bfA$ is full rank with $n < d$ (so $\ker A^\top = 0$). This means $\bfu^* = (\bfA\bfA^\top)^{-1}\bfb$, noting that the inverse $(\bfA\bfA^\top)^{-1}$ exists again because $\bfA$ is fully rank with $n < d$. Using this expression for $\bfu^*$ gives us that $\bfx^* = \bfA^\top(\bfA\bfA^\top)^{-1}\bfb$, which notably does not depend on the initialization, implying that any two points initialized in the rowspan of $\bfA$ converge to the same point.

        \bigskip

        Finally, consider two different initialization $\bfy_0 \in \bR^d \setminus \img(A^\top)$ and $\bfx_0 \in \img(A^\top)$. As before, $\bfx_0 = \bfA^\top \bfu$ for some $\bfu \in \bR^n$. Since $\bfy_0$ is not in the rowspan of $\bfA$, the stable point $\bfy^*$ of the update rule to which $\bfy_0$ converges is also not in $\img(\bfA^\top)$. Hence $\bfy^* = \bfA^\top \bfu^* + \bfv$ for some $\bfv \not\in \img\bfA^\top$, where $\bfu^*$ is as above. The stability condition $\nabla F(\bfy^*) = 0$ gives us $\bfA\bfA^\top \bfu^* + \bfA\bfv - \bfb = 0$ repeating the calculation from the last paragraph. But $\bfA\bfA^\top \bfu^* - \bfb = 0$, so $\bfA\bfv = 0$. This means
        \begin{align*}
          \|\bfy^*\|^2_2 
            &= (\bfA^\top \bfu^* + \bfv)^\top (\bfA^\top \bfu^* + \bfv) \\
            &= \bfu^\top \bfA^\top \bfA \bfu + \bfu^\top \bfA v + (\bfA\bfv)^\top \bfu + \bfv^\top \bfv \\
            &= \bfu^\top \bfA^\top \bfA \bfu + 0 + 0 + \bfv^\top \bfv \\
            &\geq \bfu^\top \bfA^\top \bfA \bfu \\
            &= \|\bfx^*\|^2_2.
        \end{align*}
        Thus, $\bfx^* = \arg\min_{\bfx\in \cS} \|\bfx\|^2_2$. Any point in the rowspan of $\bfA$ converges to $\bfx^*$ under the update rule; in particular, the initialization $\bfx_0 = 0$ converges to $\bfx^*$, proving the desired result.
    \end{enumerate}
  \end{prf}

  \prob Let $f:\bR^d \to \bR$ be a differentiable function. It satisfies the PL-inequality if there exists a constant $\mu > 0$ such that for all $w \in \bR^d$ it holds
  \begin{align*}
    \frac{1}{2}\|\nabla f(w)\|^2_2 \geq \mu(f(w) - f^*).
  \end{align*}
  By contrast we say $f$ is \emph{invex} if there exists a function $\eta:\bR^d\times \bR^d\to \bR^d$ such that for all $x, y\in \bR^d$ it holds
  \begin{align*}
    f(y) \geq f(x) + \nabla f(x)^\top \eta(x,y).
  \end{align*}
  \begin{enumerate}[(a)]
    \item Show that if $f$ satisfies the PL-inequality then $f$ is invex.
    \item Show that any stationary point of an invex function is a global minimizer.
  \end{enumerate}
  \begin{prf}$ $
    \begin{enumerate}[(a)]
      \item Since $f$ satisfies the PL-inequality, for some $\mu > 0$
        \begin{align*} 
          \frac{1}{2}\|\nabla f(w)\|^2_2 \geq \mu(f(w) - f^*)
        \end{align*}
        for all $w\in \bR^d$, where $f^*$ is the global minimum of $f$. Rearranging, we get
        \begin{align*}
          &\hspace{5em}\frac{1}{2}\|\nabla f(w)\|^2_2 \geq \mu(f(w) - f^*) \\
            &\implies \nabla f(w)^\top \nabla f(w) \geq 2\mu f(w) - 2\mu f^* \\
            &\implies - \nabla f(w)^\top \nabla f(w) \leq 2\mu f^* - 2\mu f(w) \leq f(w) \leq 2\mu f(u) - 2\mu f(w)
        \end{align*}
        for any $w, u\in \bR^d$, since $f^*$ is the global minimum of $f$. This in turn implies that
        \begin{align*}
          2\mu f(w) - \nabla f(w)^\top \nabla f(w) \leq 2\mu f(u),
        \end{align*}
        so if we set $\eta(x,y) = - \frac{1}{2\mu}\nabla f(x)$ then we get
        \begin{align*}
          f(x) = \nabla f(x)^\top \cdot \eta(x,y) \leq f(y)
        \end{align*}
        for all $x, y\in \bR^d$. This proves that $f$ is invex.

      \item A point $\bfx$ is a stationary point of $f$ if $\nabla f(x) = 0$. If $f$ is invex, then we get
        \begin{align*}
          f(y) \geq f(x) + \nabla f(x)^top \eta(x,y) = f(x)
        \end{align*}
        for all $y\in \bR^d$. Hence any stationary point of $f$ is a global minima. Combining with part (a) we see that any function which satisfies the PL-inequality is easily optimized.
    \end{enumerate}
  \end{prf}
  \prob In your favorite programming language, implement stochastic gradient-descent for the linear least squares loss $f(\bfw) = \frac{1}{2}\|\bfA\bfW - \bfb\|^2_2$. Provide convergence plots to validate the convergence guarantees for SGD discussed in class. Specifically, compare empirical and theoretical convergence rates when $\bfA \in \bR^{10,000\times 1,000}$ has iid $\cN(0, 1/\sqrt{1000})$ Gaussian entries and $\bfb = \bfA\mathbf{1} + \epsilon$ where $\mathbf{1}$ is the all-ones vector and $\epsilon$ has iid Gaussian antries with variance 1, then 0.1, then 0.01 and finally 0. Repeat the comparisons but now consider $\bfA \in \bR^{10000 \times 1000}$ whose $j$th row has iid $\cN(0,1/\sqrt{1000j})$ Gaussian entries. (\emph{Note your answer should include 8-plots, because there are two choices of $\bfA$ and four different choices of $\epsilon$.})
  \prob Consider a three-state Markov chain with stationary probabilities $\left(\frac{1}{2}, \frac{1}{3},\frac{1}{6}\right)$. consider the Metropolis-Hastings algorithm with $G$ the complete graph on these three vertices. For each edge and each direction, what is the expected probability that we would actually make a move along the edge?
  \begin{prf}
    Recall that the Metropolis transition probabilities are
    \begin{align*}
      p_{xy} = \frac{1}{r}\min \left(1, \frac{\pi(y)}{\pi(x)}\right) 
    \end{align*}
    if $x$ and $y$ are distinct but adjacent and
    \begin{align*}
      p_{xx} = 1 - \sum_{y\neq x} p_{xy}.
    \end{align*}
    Let $a,b$ and $c$ be the vertices of the graph. Then
    \begin{align*}
      p_{ab} &= \frac{1}{2}\frac{2}{1}\cdot \frac{1}{3} = \frac{1}{3} \\
      p_{ac} &= \frac{1}{2}\frac{2}{1}\cdot \frac{1}{6} = \frac{1}{6} \\
      p_{aa} &= 1 - \frac{1}{3} - \frac{1}{6} = \frac{1}{2}.
    \end{align*}
    The other transition probabilities are
    \begin{align*}
      p_{ba} = \frac{1}{2}, \hspace{1.5em} p_{bc} = \frac{1}{4}, \hspace{1.5em} p_{bb} = \frac{1}{4}
    \end{align*}
    and
    \begin{align*}
      p_{ca} = \frac{1}{2}, \hspace{1.5em} p_{cb} = \frac{1}{3}, \hspace{1.5em} p_{cc} = \frac{1}{6}.
    \end{align*}
  \end{prf}
  \prob Consider the probability distribution $p(\bfx)$ where $\bfx \in \{0,1\}^{100}$ such that $p(0) = \frac{1}{2}$ and $p(\bfx) = \frac{1/2}{2^{100}-1}$. How does Gibbs sampling behave here?
  \begin{prf}
    The Gibbs transition probabilities are given by
    \begin{align*}
      p_{xy} =
      \begin{cases}
        \frac{1}{d}\pi(y_i ~|~ x_1,...,\hatx_i,...,x_d) & \text{ if $\bfx$ and $\bfy$ differ only in $i$} \\
        0 & \text{ otherwise }
      \end{cases}.
    \end{align*}
    Let $\hate_i$ denote the element of $\{0,1\}^{100}$ whose $i$th component is 1 and is 0 elsewhere.
    We have three cases to examine.

    \bigskip 

    \noindent\emph{If we are currently at }$\mathbf{0}$, then
    \begin{itemize}
    \item there is a $\frac{1}{100}\frac{\frac{1/2}{2^{100}-1}}{\frac{1}{2}+\frac{1/2}{2^{100}-1}} \approx \frac{1}{100}\cdot \frac{1}{2^{100}-1}$ chance of moving to $\hate_i$ for any $i \in \{1,...,100\}$. Altogether, we have a $1$ in $2^{100} - 1$ chance of leaving $0$ at all.
    \item We have a $1 - \frac{1}{2^{100}-1}\approx 1$ chance of remaining at zero.
    \end{itemize}
    Hence, if we ever reach 0 then we will stay at zero, since $2^{100}-1$ is a huge number.

    \bigskip

    \noindent\emph{If we are currently at }$\hate_i$, then we
    \begin{itemize}
      \item have a $\frac{1}{100}\frac{1/2}{\frac{1}{2}+\frac{1/2}{2^{100}-1}}\approx \frac{1}{100}$ chance of moving to 0.
      \item have a $\frac{1}{100}\frac{\frac{1/2}{2^{100}-1}}{\frac{1/2}{2^{100}-1} + \frac{1/2}{2^{100}-1}}\approx \frac{1}{200}$ chance of moving to some other nonzero point, of which there are 99 adjacent to $\hate_i$ giving us an approximately $\frac{1}{2}$ chance point of moving to a point which is not $0$
      \item have an approximately $1 - \frac{1}{100} - \frac{1}{2} = \frac{1}{2}-\frac{1}{100}$ chance of remaining at $\hate_i$.
    \end{itemize}

    \bigskip

    \emph{If we are currently at } $\bfx\neq \hate_i, \mathbf{0}$, then we
    \begin{itemize}
      \item have a $\frac{1}{100}\frac{\frac{1/2}{2^{100}-1}}{\frac{1/2}{2^{100}-1}+\frac{1/2}{2^{100}-1}} \approx \frac{1}{200}$ chance of moving to any individual neighbor of $\bfx$, or altogether a $1$ in $2$ chance of leaving $\bfx$ to \emph{some} other point
      \item have an $1 - 100\cdot \frac{1}{200} \approx \frac{1}{2}$ chance of remaining at $\bfx$.
    \end{itemize}

    If we initialize a random walk on $G$ at $\mathbf{0}$ then we will remain there functionally forever. If we initialize it at any other point, then we have a $\frac{1}{2}$ chance to leave and a $\frac{1}{2}$ chance to remain. The situation is slightly different at a point neighboring $\mathbf{0}$, where we have twice the chance of transitioning to $\mathbf{0}$ than to any other point. Thus, a random walk on $G$ will visit a variety of points, transitioning to a new point every $2$ steps on average, unless it reaches $\mathbf{0}$, in which case it will remain there indefinitely. However, the chance of reaching $\mathbf{0}$ from a random initialization is just as small as the chance of leaving $\mathbf{0}$, since there are $2^{100}$ points in total.
  \end{prf}
\end{homework}
\newpage
\begin{verbatim}  
\end{verbatim}
\end{document}
