\documentclass[hidelinks,11pt,dvipsnames]{article}
% xcolor commonly causes option clashes, this fixes that
\PassOptionsToPackage{dvipsnames,table}{xcolor}
\usepackage[tmargin=1in, bmargin=1in, lmargin=0.8in, rmargin=1in]{geometry}

%%%%%%%%%%%%%%%%%%%%%%%%%%%%%%%%%%%%%%%%%%%%%%%%%%%%%%%%%%%%%%%%%%%%
%%% For inkscape-figures
%%% Assumes the following directory structure:
%%% master.tex
%%% figures/
%%%     figure1.pdf_tex
%%%     figure1.svg
%%%     figure1.pdf
%%%%%%%%%%%%%%%%%%%%%%%%%%%%%%%%%%%%%%%%%%%%%%%%%%%%%%%%%%%%%%%%%%%%
%\usepackage{import}
\usepackage{pdfpages}
\usepackage{transparent}

\newcommand{\incfig}[2][1]{%
    \def\svgwidth{#1\columnwidth}
    \import{./figures/}{#2.pdf_tex}
}

\pdfsuppresswarningpagegroup=1

% enable synctex for inverse search, whatever synctex is
\synctex=1
\usepackage{float,macrosabound,homework,theorem-env}
\usepackage{microtype}


% font stuff
\usepackage{sectsty}
\allsectionsfont{\sffamily}
\linespread{1.1}

% bibtex stuff
\usepackage[backend=biber,style=alphabetic,sorting=anyt]{biblatex}
\addbibresource{main.bib}

% colored text shortcuts
\newcommand{\blue}[1]{\color{MidnightBlue}{#1}}
\newcommand{\red}[1]{\textcolor{Mahogany}{#1}}
\newcommand{\green}[1]{\textcolor{ForestGreen}{#1}}


% use mathptmx pkg while using default mathcal font
\DeclareMathAlphabet{\mathcal}{OMS}{cmsy}{m}{n}

% fixes the positioning of subscripts in $$ $$
\renewcommand{\det}{\operatorname{det}}

\usetikzlibrary{positioning, arrows.meta}
\newcommand{\here}[2]{\tikz[remember picture]{\node[inner sep=0](#2){#1}}}

%%%%%%%%%%%%%%%%%%%%%%%%%%%%%%%%%%%%%%%%%%%%%%%%%%%%%%%%%%%%%%%%%%%%%
%%% Entry Counter
%%%%%%%%%%%%%%%%%%%%%%%%%%%%%%%%%%%%%%%%%%%%%%%%%%%%%%%%%%%%%%%%%%%%%
\newcounter{entry-counter}
\newcommand{\entry}[1]
{
	\addtocounter{entry-counter}{1}
    \tchap{Entry \arabic{entry-counter}}
	%\addcontentsline{toc}{section}{Entry \arabic{entry-counter}: #1}
	\vspace{-1.5em}
    \begin{center}
		\small \emph{Written: #1}
    \end{center}
}

\usepackage{titling}
\renewcommand\maketitlehooka{\null\mbox{}\vfill}
\renewcommand\maketitlehookd{\vfill\null}


\begin{document}
\pagestyle{empty}
	\LARGE
\begin{center}
	Toric Geometry: \\
	Theorems and Definitions \\
	
	\bigskip
	\Large
	Isaac Martin \\
	Lent 2022 \\
    Last compiled \today
\end{center}
\normalsize
\vspace{-2mm}
\hru

\tableofcontents
\newpage

\setcounter{section}{0}

\section{Dictionary}
Toric geometry is concerned with the construction of varieties and schemes given by specifying semigroups and fans and other combinatorial objects. It is therefore useful to fix certain symbols.
\begin{itemize}
	\item $N$: We define $N = \Hom_{\Grp}(\bC^*, (\bC^*)^n)$ and note that $N \cong \bZ^{n}$. 
	\item $M$ : We define $M$ to be the dual lattice of $N$, $M = \Hom_\bZ(N,\bZ) \cong \bZ^n$.
	\item $N_\bR$ and  $M_\bR$ : We define $N_\bR = N \otimes_\bZ \bR \cong \bR^n$ and $M_\bR = M \otimes_\bR \bR \cong \bR^n$.
\end{itemize}

\newpage
\section{What makes a toric variety?}

\subsection{Tori}
\subsection{Toric Varieties}
\subsection{Cones and Fans}
Throughout this section, let $T \cong \left(\bC^*\right)^n$ and $N = \Hom_{\Grp}(\bC^*, T) \cong \bZ^n$. Note that $N$ is the collection of 1-parameter subgroups of $T$, or the set of cocharacters if you prefer that terminology. In addition, every variety is an integral separated scheme of finite type over $\Spec \bC$ unless otherwise specified.

\begin{defn}\label{defn:rational-poly-cone}
	A \emph{rational polyhedral cone} $\sigma$ in $N$ is a set $\sigma \subseteq N_\bR = N \otimes_\bZ\bR \cong \bR^n$ given by the positive span of some finite subset of $N_\bR$, i.e. a set
	\begin{align*}
		\sigma = \cone(v_1,...,v_k) = \left\{\sum_{i=1}^{k}c_iv_i \midd c_i \in \bR_{\geq 0}\right\}.
	\end{align*}
	By rescaling the cone basis set, we may assume $v_i \in N$ for each $1\leq i\leq k$, and from now on will do so.
\end{defn}
\begin{defn}\label{defn:span-dim-of-cone}
	Let $\sigma = \cone\{v_1, ...,v_k\}$ be a rational polyhedral cone. The \emph{span} of $\sigma$ is the smallest vector subspace $V$ containing $\sigma$. We have that
	\begin{align*}
		V = \sigma + (-\sigma) = \span\{v_1,...,v_k\} = \span\{\sigma\}.
	\end{align*}
	The \emph{dimension} of $\sigma$ is the dimension of the span of $\sigma$. We say that $\sigma$ is \emph{full-dimensional} if $\dim \sigma = \dim N_\bR = n$. 
\end{defn}
\begin{defn}\label{defn:strictly-convex}
	A rational polyhedral cone is said to be \emph{strictly convex} if it doesn't contain a line, i.e. if it doesn't contain a one dimensional affine subspace of $N_\bR$.

	Unless otherwise specified, by ``cone'' we mean ``strictly convex rational polyhedral cone''.
\end{defn}
\begin{defn}\label{defn:dual-cone}
	Given a cone $\sigma \subseteq N_\bR$, the \emph{dual cone} $\sigma\vee \subseteq M_\bR$ is defined
	\begin{align*}
		\sigma^{\vee} = \left\{m \in M_\bR \midd \langle m,v\rangle \geq 0, ~\forall v \in \sigma\right\}.
	\end{align*}
	The pairing $\langle -,-\rangle: M_\bR \times N_\bR \to \bR$ is simply the evaluation map $\langle m,u\rangle = m(u)$.

	We further define the double dual $(\sigma^{\vee})^{\vee}$ by
	\begin{align*}
		(\sigma^{\vee})^{\vee} = \left\{v \in N_\bR \midd \langle m,v\rangle \geq 0, ~ \forall m \in \sigma^{\vee}\right\} 
	\end{align*}
\end{defn}
The following are fundamental facts regarding $\sigma$ and $\sigma^{\vee}$.
\begin{prop}\label{prop:facts-about-cones-and-duals}
	Let $\sigma$ be a cone in $N$ and $\sigma^{\vee}$ be its dual.
	\begin{enumerate}[(a)]
		\item $\sigma^{\vee}$ is a rational polyhedral cone in $M$ (not necessarily strictly convex)
		\item $(\sigma^{\vee})^{\vee} = \sigma$
		\item $\sigma$ is full-dimensional if and only if $\sigma^{\vee}$ is strictly convex
	\end{enumerate}
\end{prop}
\begin{defn}\label{defn:fan}
	A \emph{fan} $\Sigma$ in $N$ is a collection of cones in $N$ such that
	\begin{enumerate}[(i)]
		\item if $\sigma \in \Sigma$ then every face of $\sigma$ belongs to $\Sigma$ 
		\item if $\sigma_1,\sigma_2\in \Sigma$ then $\sigma_1\cap \sigma_2$ is a face of both $\sigma_1$ and $\sigma_2$.
	\end{enumerate}
\end{defn}
We wish to construct varieties from cones and fans. Starting with a cone $\sigma$ in $N$, we will associate to it an affine variety $X_\sigma = \Spec R_\sigma$. Given a fan $\Sigma$, we will construct a variety $X_\Sigma$ by gluing together $X_{\sigma_1}$ and $X_{\sigma_2}$ along $X_{\sigma_1 \cap \sigma_2}$. We will first build affine toric varieties, and for that, we'll need affine semigroups.

\begin{defn}\label{defn:affine-semigroup}
	A \emph{semigroup} is a set $S$ together with an associative binary operation and an identity element. This is what some (most) people seem to call a monoid -- it's a category with a single point. To be an \emph{affine semigroup}, $S$ must additionally satisfy:
	\begin{itemize}
		\item $S$ is commutative. We will write the binary operation as $+$ and the identity element as  $0$ to reflect this. Note that this means a finite set $A \subseteq S$ therefore generates
			\begin{align*}
				\bN A = \left\{\sum_{m\in A} a_m m \midd a_m \in \bN\right\}.
			\end{align*}
		\item  $S$ is finitely generated, i.e. there is a finite set $A \subseteq S$ such that $\bN A = S$. 
		\item The semigroup can be embedded in a lattice $M$.
	\end{itemize}
\end{defn}

We focus first on building a variety $X_\sigma$ from a cone $\sigma$ in $N$. Here is our construction/definition.
\begin{construct}\label{construct:affine-toric-variety-from-monoid}
	Given a cone $\sigma \subseteq N_\bR$ and its dual cone $\sigma^{\vee} \subseteq M_\bR$, we define
	\begin{equation}\label{eqn:semigroup-assoc-to-cone}
		S_\sigma := \sigma \cap M
	\end{equation}
	to be the semigroup associated to $\sigma$. We then consider the group algebra over $\bC$ with basis $S_\sigma$:
	\begin{equation}\label{eqn:ring-assoc-to-cone}
		\bC[S_\sigma] = \left\{\sum_{i=1}^{r} c_i \cdot z^{m_i} \midd c_i \in \bC, ~m_i \in S_\sigma \subseteq M\right\}.
	\end{equation}
	The addition on $\bC[S_\sigma]$ is formal. The multiplication is defined $z^{m_i}\cdot z^{m_j} = z^{m_i + m_j}$ and is extended by distribution. E.g. we have that $\bC[\bN^n] = \bC[t_1,...,t_n]$ and $\bC[\bZ^n] = \bC[t_1^{\pm},...,t_n^{\pm}]$.

	Finally, we define
	\begin{equation}\label{eqn:affine-toric-variety}
	    X_\sigma = \Spec \bC[S_\sigma]
	\end{equation}
	to be the affine toric variety associated to $\sigma$. Note that because we will eventually build toric varieties from fans whose affine pieces are given by pieces of the above form, we sometimes denote $X_\sigma$ by $U_\sigma$ instead.
\end{construct}
It is still left to show that $X_\sigma$ constructed in this way is in fact a toric variety.

\begin{prop}(Cox-Little-Scheck)\label{prop:variety-assoc-to-cone-is-toric}
	If $\sigma$, $S_\sigma$, and $X_\sigma$ are as in Construction (\ref{construct:affine-toric-variety-from-monoid}) then $X_\sigma$ is an affine toric variety.
\end{prop}
\begin{proof}
	See page 31 of Cox-Little Scheck. Fill it in later.
\end{proof}
One might ask, ``Why do we define $S_\sigma$ as a subset of the dual lattice $M$ rather than the lattice $N$? Surely we could take $S_\sigma = \sigma \cap N$ and get an equally reasonable result.''

\begin{center}
	\textbf{COME BACK TO THE ABOVE QUESTION. CONSIDER REVERSE CONSTRUCTION -- GIVEN AFFINE TORIC VARIETY $T \subseteq X$ CONSTRUCT A SEMIGROUP (HOWEVER ONE DOESN'T ALWAYS GET A CONE)}
\end{center}

\newpage

\section{Properties of Affine Toric Varieties}

\begin{defn}\label{defn:saturated-semigroup}
	An affine semigroup $S \subseteq M$ is said to be  \emph{saturated} if for all $k \in \bN \setminus \{0\}$ and $m \in M$, $km \in S$ implies $m \in S$.
\end{defn}
An affine semigroup $S$ is saturated if and only if $S = S_\sigma = \sigma^{\vee} \cap M$ for some strongly convex rational polyhedral cone $\sigma \subseteq N$. In terms of toric varieties, this means the following:
\begin{prop}\label{thm:saturated-semigroup-and-cone}
	Let $V$ be an affine toric variety with torus $T_N$. Then the following are equivalent:
	\begin{enumerate}[(i)]
	    \item $V$ is normal (for us this means $V \cong \Spec R$ for some integrally closed domain $R$.)
		\item $V \cong \Spec(\bC[S])$, where $S \subseteq M$ is some saturated affine semigroup
		\item $V \cong \Spec \bC[S_\sigma] = X_\sigma$, where $S_\sigma = \sigma^{\vee} \cap M$ and $\sigma \subseteq N_\bR$ is a strongly convex rational polyhedral cone.
	\end{enumerate}
\end{prop}
\begin{prf}
	
\end{prf}
Notice that embedded in the equivalence $(b) \iff (c)$ from Theorem (\ref{thm:saturated-semigroup-and-cone}) is the fact that a semigroup is affine if and only if it is isomorphic to $S_\sigma$ for some strongly convex rational polyhedral cone $\sigma$. 

\newpage

\section{Smoothness of Affine Toric Varieties}

The main goal of this section is a classification of smooth affine toric varieties associated to cones $\sigma$. This is Theorem (\ref{thm:smooth-characterization}). Before we proceed, however, we prove several useful lemmas. Throughout this section $X_\sigma$ is an affine toric variety associated to a cone $\sigma \subseteq M_\bR$.
\begin{lem}\label{lem:Z-basis-smooth-lemma}
	Let $\sigma = \cone(v_1,...,v_k)\subseteq N_\bR$ be a cone. Suppose $\{v_1,...,v_k\}$ forms some part of a $\bZ$-basis for $N$. Then $X_\sigma \cong \bC^k\times (\bC^*)^{n-k}$ where $k = \dim \sigma \leq n$.
\end{lem}
\begin{proof}
	Choose a basis $e_1,...,e_n$ for $N$ such that $v_i = e_i$ for $1 \leq i\leq k$ (that $\{v_1,...,v_k\}$ is a $\bZ$-basis for $N$ exactly makes this possible). This implies that $S_\sigma = \sigma^{\vee}\cap M$ is generated by 
	\begin{align*}
		e_1^*,...,e_k^*, \pm e_{k+1}^*,...,\pm e_n^* \in M.
	\end{align*}
	To see this, it helps to note the $e_{i}^*$ for $k+1 \leq i\leq n$ are exactly the basis vectors of $M$ which are zero on $\sigma$. This means
	\begin{align*}
		\bC[S_\sigma] = \bC[t_1,...,t_k,t_{k+1}^{\pm}, ..., t_n^{*}] = \bC[t_1,...,t_n] \otimes_\bC \bC[t_{k+1}^\pm, ...,t_n^\pm]. 
	\end{align*}
\end{proof}

\begin{lem}\label{lem:closed-points-and-semigroup-hom}
	There exists a bijection correspondence
	\begin{align*}
		\left(\begin{array}{c}
			\text{closed points} \\
			\text{of } X_\sigma
		\end{array}\right) \leftrightarrow
		\left(\begin{array}{c}
			\text{semigroup} \\
			\text{homomorphisms} \\
			S_\sigma \to \bC
		\end{array}\right).
	\end{align*}
\end{lem}

\begin{proof}
	We have the following one-to-one correspondences:
	\begin{align*}
		\left\{\begin{array}{c}
			\text{closed points} \\
			\text{in X}
		\end{array}\right\} \leftrightarrow 
		\left\{\begin{array}{c}
			\text{maps of schemes} \\
			\text{over $\bC$} \\
			\Spec \bC \to \Spec \bC[S_\sigma]
		\end{array}\right\} \leftrightarrow^{(*)} 
		\left\{\begin{array}{c}
			\text{semigroup morphisms} \\
			S_\sigma \to \bC
		\end{array}\right\}.
	\end{align*}
	Only ($\ast$) is new. A morphism of schemes $\Spec \bC \to \Spec \bC[S_\sigma]$ over $\bC$ yields a morphism of $\bC$-algebras $\varphi:\Spec \bC[S_\sigma] \to \Spec \bC$, which in turn determines an affine semigroup homomorphism $S_\sigma \to \bC$ (with $\bC$ considered to be an affine semigroup under multiplication) since $\varphi(z^{t+s}) = \varphi(z^t)\varphi(z^s)$. Likewise, any homomorphism of affine semigroups $\psi:S_\sigma \to \bC$ can be extended to an algebra homomorphism $\varphi:\bC[S_\sigma]\to \bC$ by making a choice for the field automorphism $\varphi|_\bC$. Normally, we would have two choices for $\varphi|_\bC$, but in order for this to be compatible with the structure maps on $\Spec \bC$ and $\Spec \bC[S_\sigma]$, we have only \emph{one} choice. Therefore $\varphi$ is uniquely determined by the image of $S_\sigma$ in $\bC$, and since it is a $\bC$-algebra homomorphism it corresponds to a unique map of schemes over  $\bC$. 
\end{proof}

\begin{defn}\label{defn:perp-cone-distinguished-semigroup-hom}
	Define $x_\sigma \in X_\sigma$ to be the point corresponding to the semigroup map
	\begin{align*}
		S_\sigma \xrightarrow{x_\sigma} \bC, ~ m \mapsto 
		\begin{cases}
			1 & \text{if } m \in \sigma^\perp \\
			0 & \text{otherwise} 
		\end{cases},
	\end{align*}
	where
	\begin{align*}
		\sigma^\perp = \left\{m\in M_\bR \midd \langle u,m\rangle = 0, ~\forall u \in \sigma\right\} .
	\end{align*}
\end{defn}
We now proceed to Theorem (\ref{thm:smooth-characterization}).
\begin{thm}\label{thm:smooth-characterization}
	Let  $\sigma \subseteq N_\bR$ be a cone and $X_\sigma$ be the associated affine toric variety. The following are equivalent:
	\begin{enumerate}[(i)]
		\item $X_\sigma$ is smooth
		\item $\sigma$ is generated by a subset of a $\bZ$-basis for $N$ 
		\item $X_\sigma \cong \bC^k \times (\bC^*)^{n-k}$ where $k = \dim \sigma$.
	\end{enumerate}
\end{thm}
\begin{proof}[\ref{thm:smooth-characterization}]
	Lemma (\ref{lem:Z-basis-smooth-lemma}) gives us $(ii)\implies (iii)$. The fibre product of smooth schemes with smooth structure maps is again smooth, so $(iii)\implies (i)$ is clear. It is only left to prove $(i) \implies (ii)$. 
\end{proof}
\end{document}
