\documentclass[hidelinks,11pt,dvipsnames]{article}
% xcolor commonly causes option clashes, this fixes that
\PassOptionsToPackage{dvipsnames,table}{xcolor}
\usepackage[tmargin=1in, bmargin=1in, lmargin=0.8in, rmargin=1in]{geometry}

%%%%%%%%%%%%%%%%%%%%%%%%%%%%%%%%%%%%%%%%%%%%%%%%%%%%%%%%%%%%%%%%%%%%
%%% For inkscape-figures
%%% Assumes the following directory structure:
%%% master.tex
%%% figures/
%%%     figure1.pdf_tex
%%%     figure1.svg
%%%     figure1.pdf
%%%%%%%%%%%%%%%%%%%%%%%%%%%%%%%%%%%%%%%%%%%%%%%%%%%%%%%%%%%%%%%%%%%%
%\usepackage{import}
\usepackage{pdfpages}
\usepackage{transparent}

\newcommand{\incfig}[2][1]{%
    \def\svgwidth{#1\columnwidth}
    \import{./figures/}{#2.pdf_tex}
}

\pdfsuppresswarningpagegroup=1

% enable synctex for inverse search, whatever synctex is
\synctex=1
\usepackage{float,macrosabound,homework,theorem-env}
\usepackage{microtype}


% font stuff
\usepackage{sectsty}
\allsectionsfont{\sffamily}
\linespread{1.1}

% bibtex stuff
\usepackage[backend=biber,style=alphabetic,sorting=anyt]{biblatex}
\addbibresource{main.bib}

% colored text shortcuts
\newcommand{\blue}[1]{\color{MidnightBlue}{#1}}
\newcommand{\red}[1]{\textcolor{Mahogany}{#1}}
\newcommand{\green}[1]{\textcolor{ForestGreen}{#1}}


% use mathptmx pkg while using default mathcal font
\DeclareMathAlphabet{\mathcal}{OMS}{cmsy}{m}{n}

% fixes the positioning of subscripts in $$ $$
\renewcommand{\det}{\operatorname{det}}

\usetikzlibrary{positioning, arrows.meta}
\newcommand{\here}[2]{\tikz[remember picture]{\node[inner sep=0](#2){#1}}}

%%%%%%%%%%%%%%%%%%%%%%%%%%%%%%%%%%%%%%%%%%%%%%%%%%%%%%%%%%%%%%%%%%%%%
%%% Entry Counter
%%%%%%%%%%%%%%%%%%%%%%%%%%%%%%%%%%%%%%%%%%%%%%%%%%%%%%%%%%%%%%%%%%%%%
\newcounter{entry-counter}
\newcommand{\entry}[1]
{
	\addtocounter{entry-counter}{1}
    \tchap{Entry \arabic{entry-counter}}
	%\addcontentsline{toc}{section}{Entry \arabic{entry-counter}: #1}
	\vspace{-1.5em}
    \begin{center}
		\small \emph{Written: #1}
    \end{center}
}

\usepackage{titling}
\renewcommand\maketitlehooka{\null\mbox{}\vfill}
\renewcommand\maketitlehookd{\vfill\null}


\begin{document}
\pagestyle{empty}
	\LARGE
\begin{center}
	Definitions and Theorems from Infinite Groups (Lent '22) \\
	\Large
	Isaac Martin \\
    Last compiled \today
\end{center}
\normalsize
\vspace{-2mm}
\hru
\tchap{Free Groups and Presentations}
\begin{defn}[Directly Finite]\label{defn:directly-finite-ring}
	We say that a ring $R$ (not necessarily commutative) with unity is \emph{directly finite} (D.F.) if $\forall a,b \in R$, $ab = 1 \implies ba = 1$. 
\end{defn}

\begin{defn}[Group Algebra]\label{defn:group-alg}
	Let $G$ be a group, $K$ a field. The \emph{group algebra}, denoted $K[G]$, is a $K$-algebra. As a set it consists of all finite linear combinations of elements in $K$ and $G$ :
	\begin{align*}
		K[G] = \left\{\sum_{g\in G}\lambda_g \cdot g \midd \lambda_g \in K, \lambda_g \neq 0 \text{ for only finitely many $g$ }\right\}.
	\end{align*}
	Addition is defined pointwise: $\lambda g + \lambda' g = (\lambda+\lambda')g$. Multiplication is defined
	\begin{align*}
		(\lambda g)\cdot (\mu h) = (\lambda \mu)(gh)
	\end{align*}
	and extended by distribution.
\end{defn}

\begin{defn}label{defn:directly-finite-group}
	A group $G$ is said to be \emph{directly finite} (D.F.) if $K[G]$ is a directly finite ring for all fields $K$.
\end{defn}

\begin{example}\label{ex:directly-finite-group}$ $
	\begin{enumerate}[(i)]
		\item  if $G$ is abelian then $K[G]$ is commutative, and therefore directly finite.
		\item if $G$ is finite then it is also directly finite.
	\end{enumerate}
\end{example}

\begin{thm}[Kaplansky]\label{thm:grp-alg-over-C-DF}
	For any group $G$, the group algebra $\bC[G]$ is directly finite.
\end{thm}

\begin{thm}[Elek, Szako ('01')]\label{thm:sofic-groups-DF}
	Every sofic group is directly finite.
\end{thm}

\begin{prop}\label{prop:finite-group-injects-into-Sym}
	Let $G$ be a finite group. The map  $\rho:G\to \Sym(G)$ defined $\rho(g):h\mapsto gh$ is an injective homomorphism. Moreover, for all $e \neq g \in G$, $\rho(g)$ has no fixed points.
\end{prop}

\begin{defn}[Hamming Distance]\label{defn:hamming-dist}
	Suppose $\sigma, \tau \in \Sym(n)$ are two permutations. The \emph{Hamming distance} from $
\sigma$ to $\tau$ is
\begin{align*}
	d_n(\sigma, \tau) = 1 - \frac{1}{n}\left|\{1 \leq i \leq n ~ \mid ~ \sigma(i) = \tau(i)\}\right|.
\end{align*}
That is, it's a number between 0 and 1 and is equal to 0 if and only if $\sigma = \tau$.
\end{defn}

\begin{defn}[Sofic]\label{defn:sofic-group}
	$G$ is a sofic group if and only if $\forall A \subseteq G$, $\forall \epsilon > 0 $ there exists $n \in \bN$ and a function $\phi:A\to \Sym(n)$ such that
	\begin{enumerate}[(i)]
		\item for all $g, h \in A$, if $gh \in A$, then 
			\begin{align*}
				d_n(\phi(gh),\phi(g)\phi(h)) \leq \epsilon,
			\end{align*}
		i.e. the distance is ``small''
	\item for all $e \neq g \in G$,
		\begin{align*}
			d_n(id_n, \phi(g)) \geq 1 - \epsilon,
		\end{align*}
		i.e. the distance is ``large''.
	\end{enumerate}
	Such a function $\phi$ is a $(A, \epsilon)$-representation.
\end{defn}

\begin{example}\label{ex:finite-group-sofic}
	Every finite group is sofic.
\end{example}

\begin{thm}\label{thm:abelian-groups-sofic}
	Every abelian group is sofic.
\end{thm}

\begin{lem}\label{lem:int-is-sofic}
	$\bZ$ is sofic.
\end{lem}

\begin{lem}\label{lem:G-sofic-iff-all-fg-subgroup-sofic}
	A group $G$ is sofic if and only if every finitely generated subgroup of $G$ is sofic.
\end{lem}

\begin{lem}\label{lem:sofic-preserved-by-products}
	If $G$ and $H$ are sofic groups, then $G\times M$ is sofic.
\end{lem}

\begin{proof}[Theorem \ref{thm:abelian-groups-sofic}]
	Given an abelian group $G$, we may assume it is finitely generated by lemma \ref{lem:G-sofic-iff-all-fg-subgroup-sofic}. By the structure theorem, we have
	\begin{align*}
		G \cong \bZ^k \oplus \frac{\bZ}{p_1^{n_1}} \oplus ... \oplus \frac{\bZ}{p_i^{n_i}},
	\end{align*}
	hence $G$ is sofic by lemma \ref{lem:sofic-preserved-by-products}.
\end{proof}

\section{2022 - 01 - 24: Free Groups}
Throughout this section, $X$ is a set and $X^{-1} = \left\{x^{-1} \midd x \in X\right\}$ is the set of inverses of $X$. 

\begin{defn}\label{def:word}
	A \emph{word} in $X$ is a finite sequence of symbols:  $y_1y_2...y_m$ with $y_i \in X\cup X^{-1}$. The empty word is a valid word. We denote the set of all words in $X$ by $W(X)$. 
\end{defn}

\begin{defn}\label{def:concatenation}
	\emph{Concatenation of words} is a map $W(X)\times W(X) \to W(X)$ defined
	\begin{align*}
		(y_1...y_m, z_1...z_n) \mapsto y_1...y_mz_1...z_n.
	\end{align*}
	This map gives $W(X)$ the structure of a monoid where the empty word is the identity.
\end{defn}
\begin{defn}\label{def:expansion-contraction}
	Given two words $w,v \in W(X)$, we say that $w \sim v$ if it is possible to pass from one word to the other by means of a finite sequence of the following two operations:
	\begin{enumerate}[(a)]
		\item insertion of an $xx^{-1}$ or an $x^{-1}x$ for $x \in X$, as consecutive elements of a word;
		\item deletion of such an $xx^{-1}$ or $x^{-1}x$.
	\end{enumerate}
	The relation $\sim$ is an equivalence relation on $W(X)$ and we define the \emph{free group on $X$} to be $\frac{W(X)}{\sim}$. The group operation on $F(X)$ is induced by concatenation.
\end{defn}
\begin{defn}\label{def:reduced-word}
	A word $w \in W(X)$ is said to be \emph{reduced} if there is no word $v$ which can be obtained by operation (b) above; in other words, if $w$ is the shortest word in the equivalence class $[x]\in F(X)$. For any other word $v \in W(X)$, there exists a unique reduced $w \in W(X)$ such that $[v] = [w]$.
\end{defn}
\begin{thm}\label{thm:universal-property-of-free-group}
	Let $X$ be a set, $H$ a group, and $\phi:X\to H$ a function. Then there exists a unique group homomorphism $\Phi:F(X)\to H$ such that
	\begin{center}
		\begin{tikzcd}
			X \arrow[r,"\phi"] \arrow[hookrightarrow]{d}{i} &H \\
			F(X) \arrow[dashed]{ru}{\Phi} &
		\end{tikzcd}
	\end{center}
	commutes.
\end{thm}

\end{document}








