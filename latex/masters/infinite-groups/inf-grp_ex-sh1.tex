\documentclass[hidelinks,11pt,dvipsnames]{article}
% xcolor commonly causes option clashes, this fixes that
\PassOptionsToPackage{dvipsnames,table}{xcolor}
\usepackage[tmargin=1in, bmargin=1in, lmargin=0.8in, rmargin=1in]{geometry}

%%%%%%%%%%%%%%%%%%%%%%%%%%%%%%%%%%%%%%%%%%%%%%%%%%%%%%%%%%%%%%%%%%%%
%%% For inkscape-figures
%%% Assumes the following directory structure:
%%% master.tex
%%% figures/
%%%     figure1.pdf_tex
%%%     figure1.svg
%%%     figure1.pdf
%%%%%%%%%%%%%%%%%%%%%%%%%%%%%%%%%%%%%%%%%%%%%%%%%%%%%%%%%%%%%%%%%%%%
%\usepackage{import}
\usepackage{pdfpages}
\usepackage{transparent}

\newcommand{\incfig}[2][1]{%
    \def\svgwidth{#1\columnwidth}
    \import{./figures/}{#2.pdf_tex}
}

\pdfsuppresswarningpagegroup=1

% enable synctex for inverse search, whatever synctex is
\synctex=1
\usepackage{float,macrosabound,homework,theorem-env}
\usepackage{microtype}


% font stuff
\usepackage{sectsty}
\allsectionsfont{\sffamily}
\linespread{1.1}

% bibtex stuff
\usepackage[backend=biber,style=alphabetic,sorting=anyt]{biblatex}
\addbibresource{main.bib}

% colored text shortcuts
\newcommand{\blue}[1]{\color{MidnightBlue}{#1}}
\newcommand{\red}[1]{\textcolor{Mahogany}{#1}}
\newcommand{\green}[1]{\textcolor{ForestGreen}{#1}}


% use mathptmx pkg while using default mathcal font
\DeclareMathAlphabet{\mathcal}{OMS}{cmsy}{m}{n}

% fixes the positioning of subscripts in $$ $$
\renewcommand{\det}{\operatorname{det}}

\usetikzlibrary{positioning, arrows.meta}
\newcommand{\here}[2]{\tikz[remember picture]{\node[inner sep=0](#2){#1}}}

%%%%%%%%%%%%%%%%%%%%%%%%%%%%%%%%%%%%%%%%%%%%%%%%%%%%%%%%%%%%%%%%%%%%%
%%% Entry Counter
%%%%%%%%%%%%%%%%%%%%%%%%%%%%%%%%%%%%%%%%%%%%%%%%%%%%%%%%%%%%%%%%%%%%%
\newcounter{entry-counter}
\newcommand{\entry}[1]
{
	\addtocounter{entry-counter}{1}
    \tchap{Entry \arabic{entry-counter}}
	%\addcontentsline{toc}{section}{Entry \arabic{entry-counter}: #1}
	\vspace{-1.5em}
    \begin{center}
		\small \emph{Written: #1}
    \end{center}
}

\usepackage{titling}
\renewcommand\maketitlehooka{\null\mbox{}\vfill}
\renewcommand\maketitlehookd{\vfill\null}


\begin{document}
\pagestyle{empty}
	\LARGE
\begin{center}
	Infinite Groups Example Sheet 1 \\
	\Large
	Isaac Martin \\
    Last compiled \today
\end{center}
\normalsize
\vspace{-2mm}
\hru
\tchap{Chapter 1}
\begin{homework}[e]
	\prob Show directly that every finite group is directly finite (without using the fact that sofic groups are directly finite)
	\begin{prf}
		Note first that $G$ is a basis for $K[G]$ and therefore $K[G]$ has finite dimension exactly when $G$ is a finite group.

		Given an element $r \in K[G]$, define the map $\alpha_r:K[G]\to K[G]$ to be left multiplication by $r$, i.e. $\alpha_r(s) = r\cdot s$. This map is $K$-linear due to the distributive and associative properties on $K[G]$. 

		Suppose we have  $r,s\in K[G]$ such that $r\cdot s = 1$. Given any $x \in K[G]$ we have that $\alpha_r(\alpha_s(x)) = r(sx) = (rs)x = x$, hence the composition $\alpha_r\circ \alpha_s$ is the identity on $K[G]$, and is in particular both injective and surjective. Recall the following two general facts regarding compositions of functions:
		\begin{itemize}
			\item if $f\circ g$ is injective then $g$ is injective
			\item if $f\circ g$ is surjective then $f$ is surjective
		\end{itemize}
		Hence, $\alpha_r$ is surjective and $\alpha_s$ is injective. The rank-nullity theorem tells us that an endomorphism on a finite dimensional vector space is injective if and only if it is surjective, so both $\alpha_r$ and $\alpha_s$ are automorphisms of  $K[G]$. Since inverses are unique and $\alpha_r \circ \alpha_s = \id_{K[G]}$, we conclude that $\alpha_r$ and $\alpha_s$ are inverses. Applying these maps to the identity yields that
		\begin{align*}
			rs = \alpha_r\circ\alpha_s(1) = 1 = \alpha_s\circ\alpha_r(1) = sr,
		\end{align*}
		so we are done.
	\end{prf}

	\prob Let $X_1$ and $X_2$ be finite sets. Show that there is an injective homomorphism $\Phi:\Sym(X_1)\times \Sym(X_2)\to \Sym(X_1\times X_2)$, given by:
	\begin{align*}
		\Phi(\sigma_1, \sigma_2)(x_1,x_2) = \big(\sigma_1(x_1), \sigma_2(x_2)\big).
	\end{align*}
	Deduce that the direct product of two sofic groups is sofic.
	
	\prob Let $X$ and $Y$ be finite sets. Prove that $F(X) \cong F(Y)$ if and only if $|X|=|Y|$.
	\begin{prf}
		Suppose first that $|X| = |Y|$, which by definition means there is some bijection $f:X\to Y$. Let $\iota:X\to F(Y)$ be the function obtained by composing $f$ with the inclusion $Y \hookrightarrow F(Y)$. We show that the pair $(\iota, F(Y))$ satisfy the universal property of a free group over $X$, and therefore $F(Y) \cong F(X)$. 

		Let $G$ be an arbitrary group and fix a function $\phi:X\to G$. Define the map $\Phi:F(X)\to G$ by $y_1^{\epsilon_1}y_2^{\epsilon_2}...y_n^{\epsilon_2}\mapsto \phi(x_1)^{\epsilon_1}\phi(x_2)^{\epsilon_2}...\phi(x_n)^{\epsilon_n}$, where $x_i = f^{-1}(y_i)$ and $\epsilon_i \in \{\pm 1\}$. Note that for each $y \in Y$ there is a unique $x \in X$ such that $x = f^{-1}(y)$ since $f$ is bijective. 

		The map $\Phi$ is a group morphism:
		\begin{itemize}
			\item $\Phi(1) = \Phi(yy^{-1}) = \phi(f^{-1}(y))\cdot \phi(f^{-1}(y))^{-1} = 1_G$,
			\item $\Phi\left(y_1^{\epsilon_1}...y_m^{\epsilon_m}z_1^{\delta_1}...z_n^{\delta_n}\right) = \phi\left(f^{-1}(y_1)\right)^{\epsilon_1}...\phi\left(f^{-1}(y_m)\right)^{\epsilon_m}\phi\left(f^{-1}(z_1)\right)^{\delta_1}...\phi\left(f^{-1}(z_n)\right)^{\delta_n}$

				$\phantom{\Phi\left(y_1^{\epsilon_1}...y_m^{\epsilon_m}z_1^{\delta_1}...z_n^{\delta_n}\right)} = \Phi(y_1^{\epsilon_1}...y_m^{\epsilon_m}z_1^{\delta_1}...z_n^{\delta_n})$.	
		\end{itemize}

		Suppose we had another group homomorphism $\Psi:F(Y)\to G$ extending $\phi$. We then would have
		\begin{align*}
			\Psi(y^{-1}) = \Psi(y)^{-1} = \phi\left(f^{-1}(y)\right) = \Phi(y)^{-1} = \Phi(y^{-1})
		\end{align*}
		for each $y \in Y$, and since
		\begin{align*}
			\Psi\left(y_1^{\epsilon_1}...y_n^{\epsilon_n}\right) &= \Psi(y_1)^{\epsilon_1}\cdot ...\cdot \Psi(y_n)^{\epsilon_n} = \phi\left(f^{-1}(y_1)\right)^{\epsilon_1}...\phi\left(f^{-1}(y_n)\right)^{\epsilon_n} = \Phi\left(y_1^{\epsilon_1}...y_n^{\epsilon_n}\right),
		\end{align*}
		we get that $\Psi = \Phi$. Thus, $\Phi$ uniquely extends $\phi$ and therefore $F(Y)$ satisfies the universal property of a free group over $X$. This implies that $F(X) \cong F(Y)$. 

		Let us now consider the forward implication. One consequence of the universal property of free groups is that, for any group $G$, we have a one-to-one correspondence of functions $f:X\to G$ and group homomorphisms $\Phi:F(X)\to G$. Consider the case that $G \cong \bZ/2\bZ$. The map sending a function $f:X\to G$ to $f^{-1}(0)$ yields a bijection between the possible functons $X \to G$ and the subsets of $X$, so $\left|2^{|X|}\right| = \left|\Hom_{\Set}(X,\bZ/2\bZ)\right|$. These facts together with the assumption that $F(X) \cong F(Y)$ give us
		\begin{align*}
			\left|2^{|X|}\right| &= \left|\Hom_{\Set}(X,\bZ/2\bZ)\right| = \left|\Hom_{\Grp}(F(X),\bZ/2\bZ)\right| \\
								 &= \left|\Hom_{\Grp}(Y,\bZ/2\bZ)\right| = \left|\Hom_{\Set}(Y,\bZ/2\bZ)\right| = \left|2^{|Y|}\right|.
		\end{align*}
		Up until now, we have not needed the finiteness of $|X|$ and $|Y|$. However, the statement "$\left|2^{|X|}\right| = \left|2^{|Y|}\right| \implies |X| = |Y|$" is true only when $|X| = |Y|$ (in fact, I think this statement is independent of ZFC when $X$ and $Y$ are infinite). In any case, by invoking the finiteness of $X$ and $Y$, we are done.
	\end{prf}

	\prob[\textsc{Exercise 10}] Another problem


\end{homework}
\end{document}
