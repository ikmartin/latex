\documentclass[hidelinks,11pt,dvipsnames]{article}
% xcolor commonly causes option clashes, this fixes that
\PassOptionsToPackage{dvipsnames,table}{xcolor}
\usepackage[tmargin=1in, bmargin=1in, lmargin=0.8in, rmargin=1in]{geometry}

%%%%%%%%%%%%%%%%%%%%%%%%%%%%%%%%%%%%%%%%%%%%%%%%%%%%%%%%%%%%%%%%%%%%
%%% For inkscape-figures
%%% Assumes the following directory structure:
%%% master.tex
%%% figures/
%%%     figure1.pdf_tex
%%%     figure1.svg
%%%     figure1.pdf
%%%%%%%%%%%%%%%%%%%%%%%%%%%%%%%%%%%%%%%%%%%%%%%%%%%%%%%%%%%%%%%%%%%%
%\usepackage{import}
\usepackage{pdfpages}
\usepackage{transparent}

\newcommand{\incfig}[2][1]{%
    \def\svgwidth{#1\columnwidth}
    \import{./figures/}{#2.pdf_tex}
}

\pdfsuppresswarningpagegroup=1

% enable synctex for inverse search, whatever synctex is
\synctex=1
\usepackage{float,macrosabound,homework,theorem-env}
\usepackage{microtype}


% font stuff
\usepackage{sectsty}
\allsectionsfont{\sffamily}
\linespread{1.1}

% bibtex stuff
\usepackage[backend=biber,style=alphabetic,sorting=anyt]{biblatex}
\addbibresource{main.bib}

% colored text shortcuts
\newcommand{\blue}[1]{\color{MidnightBlue}{#1}}
\newcommand{\red}[1]{\textcolor{Mahogany}{#1}}
\newcommand{\green}[1]{\textcolor{ForestGreen}{#1}}


% use mathptmx pkg while using default mathcal font
\DeclareMathAlphabet{\mathcal}{OMS}{cmsy}{m}{n}

% fixes the positioning of subscripts in $$ $$
\renewcommand{\det}{\operatorname{det}}

\usetikzlibrary{positioning, arrows.meta}
\newcommand{\here}[2]{\tikz[remember picture]{\node[inner sep=0](#2){#1}}}

%%%%%%%%%%%%%%%%%%%%%%%%%%%%%%%%%%%%%%%%%%%%%%%%%%%%%%%%%%%%%%%%%%%%%
%%% Entry Counter
%%%%%%%%%%%%%%%%%%%%%%%%%%%%%%%%%%%%%%%%%%%%%%%%%%%%%%%%%%%%%%%%%%%%%
\newcounter{entry-counter}
\newcommand{\entry}[1]
{
	\addtocounter{entry-counter}{1}
    \tchap{Entry \arabic{entry-counter}}
	%\addcontentsline{toc}{section}{Entry \arabic{entry-counter}: #1}
	\vspace{-1.5em}
    \begin{center}
		\small \emph{Written: #1}
    \end{center}
}

\usepackage{titling}
\renewcommand\maketitlehooka{\null\mbox{}\vfill}
\renewcommand\maketitlehookd{\vfill\null}


\begin{document}
\pagestyle{empty}
	\LARGE
\begin{center}
	Elliptic Curves Example Sheet 1\\
	\Large
	Isaac Martin \\
    Last compiled \today
\end{center}
\normalsize
\vspace{-2mm}
\hru

\begin{homework}[e]
	\prob[\textsc{Exercise 2}] Find rational parametrizations for the plane conic $x^{2} + xy + 3y^2 = 1$ and for the singular plane cubic $y^2 = x^2(x+1)$.
	\begin{prf}
		We first consider the plane conic $f(x,y) = x^2 + xy + 3y^2 - 1 = 0$ as a curve over $\bR$, and we illustrate the method by which the rational parametrization was found for the sake of the author who will revise these problems prior to the exam. The point $(-1,0)$ is a solution to $f(x,y) = 0$ and therefore the line $y = t(x+1)$ intersects the curve defined by $f$ at this point. We claim it intersects the ellipse $f$ at exactly one other point for all but a single value of $t \in \bR$. This point must satisfy the equation $f(x,t(x+1)) = 0$, hence
		\begin{align*}
			f(x,t(x+1)) &= x^2 + x(t(x+1)) + 3(t(x+1))^2 - 1 = 0 \\
			&\iff (x^2-1) + x(t(x+1)) + 3(t(x+1))^2 = 0 \\
			&\iff (x + 1)\left[(x-1)+xt+3t^2(x+1)\right] = 0 \\
			&\iff x = -1 \buffword{1em}{or} x = \frac{1 - 3t^2}{1 + t + 3t^2}.
		\end{align*}
		Using the latter expression to solve for $y$ in terms of $t$ gives us the potential parameterization
		\begin{align*}
			x_t = \frac{1 - 3t^2}{1 + t + 3t^2}, ~ y_t = \frac{2t + t^2}{1 + t + 3t^2}.
		\end{align*}
		The calculation above proves that $f(x_t, y_t) = 0$, so we need only show that $t \mapsto (x_t,y_t)$ is injective outside of a finite subset of $\bR$. To see this, consider the map $f(\bR^2)\setminus\{(-1,0),(-1,1/3)\} \to \bA^1$ defined $(x,y) \mapsto \frac{y}{x+1}$ is an inverse to $t \mapsto (x_t,y_t)$ outside except at $(-1,0)$ and  $(-1,1/3)$. This means $t \mapsto (x_t,y_t)$ is injective except at these two points, and is therefore a rational parameterization of the curve.

		\bigskip

		Now consider the plane conic $C: ~ y^2 = x^2(x+1)$, and let $x_t = t^2 - 1$ and $y_t = t(t^2 - 1)$. I claim that $t \mapsto (x_t,y_t)$ is a rational parameterization of $C$. The map $(x,y) \mapsto y/x$ is an inverse to $t\mapsto (x_t,y_t)$ everywhere except $(x,y) \in \{(\pm 1, 1), (\pm 1, -1), (0,0)\}$ since
		\begin{align*}
			\frac{t(t^2 - 1)}{(t^2 - 1)} = t \buffword{1em}{when} t \neq \pm 1,
		\end{align*}
		hence $t \mapsto (x_t,y_t)$ is injective outside a finite subset of $\bR$. Furthermore,
		\begin{align*}
			y_t^2 = t^2(t^2 - 1)^2 = (t^2-1)^2(t^2 - 1 + 1) = x_t^2(x_t + 1),
		\end{align*}
		so $t \mapsto (x_t, y_t)$ is indeed a rational parameterization of $C$.
	\end{prf}

	\prob[\textsc{Exercise 7}] Let $E$ be an elliptic curve over $\bQ$ with Weierstrass equation $y^2 = f(x)$.
	\begin{enumerate}[(i)]
		\item Put the curve $E_d: ~ dy^2 = f(x)$ in Weierstrass form.
		\item Show that if $j(E) \neq 0, 1728$ then every twist of $E$ is isomorphic to $E_d$ for some unique square-free integer $d$. [A \emph{twist} of $E$ is an elliptic curve $E'$ defined over $\bQ$ that is isomorphic to $E$ over $\ol{\bQ}.$]
	\end{enumerate}

	\prob[\textsc{Exercise 9}]$ $
	\begin{enumerate}[(i)]
		\item Find a formula for doubling a point on the elliptic curve $E:~y^2 = x^3 + ax + b.$ [You should fully expand the numerator of each rational function in your answer.]
		\item Find a polynomial in $x$ whose roots are the $x$-coordinates of the points $T$ with $3T = 0_E$. [Hint: Write $3T = 0_E$ as $2T = -T$.]
		\item Show that the polynomial found in \emph{(ii)} has distinct roots.
	\end{enumerate}
	\begin{prf}$ $
		\begin{enumerate}[(i)]
			\item 
		\end{enumerate}
	\end{prf}
	\prob[\textsc{Exercise 10}] Let $C$ be the plane cubix $aX^3 + bY^3 + cZ^3 = 0$ with $a,b,c \in \bQ^*$. Show that the image of the morphism $C \to \bP^3$; $(X^3: Y^3 : Z^3 : XYZ)$ is an elliptic curve $E$, and put  $E$ in Weierstrauss form. [You should try to give an answer that is symmetric under permuting $a,b$ and $c$.] What is the degree of the morphism from $C$ to $E$?
\end{homework}

\end{document}
