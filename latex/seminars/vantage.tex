\documentclass[hidelinks,11pt,dvipsnames]{article}
% xcolor commonly causes option clashes, this fixes that
\PassOptionsToPackage{dvipsnames,table}{xcolor}
\usepackage[tmargin=1in, bmargin=1in, lmargin=0.8in, rmargin=1in]{geometry}

%%%%%%%%%%%%%%%%%%%%%%%%%%%%%%%%%%%%%%%%%%%%%%%%%%%%%%%%%%%%%%%%%%%%
%%% For inkscape-figures
%%% Assumes the following directory structure:
%%% master.tex
%%% figures/
%%%     figure1.pdf_tex
%%%     figure1.svg
%%%     figure1.pdf
%%%%%%%%%%%%%%%%%%%%%%%%%%%%%%%%%%%%%%%%%%%%%%%%%%%%%%%%%%%%%%%%%%%%
%\usepackage{import}
\usepackage{pdfpages}
\usepackage{transparent}

\newcommand{\incfig}[2][1]{%
    \def\svgwidth{#1\columnwidth}
    \import{./figures/}{#2.pdf_tex}
}

\pdfsuppresswarningpagegroup=1

% enable synctex for inverse search, whatever synctex is
\synctex=1
\usepackage{float,macrosabound,homework,theorem-env}
\usepackage{microtype}


% font stuff
\usepackage{sectsty}
\allsectionsfont{\sffamily}
\linespread{1.1}

% bibtex stuff
\usepackage[backend=biber,style=alphabetic,sorting=anyt]{biblatex}
\addbibresource{main.bib}

% colored text shortcuts
\newcommand{\blue}[1]{\color{MidnightBlue}{#1}}
\newcommand{\red}[1]{\textcolor{Mahogany}{#1}}
\newcommand{\green}[1]{\textcolor{ForestGreen}{#1}}


% use mathptmx pkg while using default mathcal font
\DeclareMathAlphabet{\mathcal}{OMS}{cmsy}{m}{n}

% fixes the positioning of subscripts in $$ $$
\renewcommand{\det}{\operatorname{det}}

\usetikzlibrary{positioning, arrows.meta}
\newcommand{\here}[2]{\tikz[remember picture]{\node[inner sep=0](#2){#1}}}

%%%%%%%%%%%%%%%%%%%%%%%%%%%%%%%%%%%%%%%%%%%%%%%%%%%%%%%%%%%%%%%%%%%%%
%%% Entry Counter
%%%%%%%%%%%%%%%%%%%%%%%%%%%%%%%%%%%%%%%%%%%%%%%%%%%%%%%%%%%%%%%%%%%%%
\newcounter{entry-counter}
\newcommand{\entry}[1]
{
	\addtocounter{entry-counter}{1}
    \tchap{Entry \arabic{entry-counter}}
	%\addcontentsline{toc}{section}{Entry \arabic{entry-counter}: #1}
	\vspace{-1.5em}
    \begin{center}
		\small \emph{Written: #1}
    \end{center}
}

\usepackage{titling}
\renewcommand\maketitlehooka{\null\mbox{}\vfill}
\renewcommand\maketitlehookd{\vfill\null}


\begin{document}
\pagestyle{empty}
	\LARGE
\begin{center}
	Notes from the VaNTAGe Seminar \\
	\Large
	Isaac Martin \\
    Last compiled \today
\end{center}
\normalsize
\vspace{-2mm}
\hru

\tableofcontents
\newpage

\section{2022-02-01: Computing isomorphism classes of abelian varieties over finite fields}

We have a nice classification of abelian varieties $A/\bC$ of dimension $g$. For such a variety, $A(\bC)$ is a torus: $T:= \bC^g/\Lambda$ where $\Lambda \simeq_\bZ \bZ^{2g}$. In addition, $T$ admits a non-degenerate \emph{Riemann form} which in turn yields a polarization. The functor $A \mapsto A(\bC)$ therefore yields an equivalence of categories:
\begin{align*}
	\left\{\text{abelian varieties}/\bC\right\} \leftrightarrow
	\begin{array}{c}
		\bC^g/\lambda \text{ with } \Lambda \simeq \bZ^{2g} admitting \\
		a Riemann form
	\end{array}.
\end{align*}
In characteristic $p >0$, such an equivalence \emph{cannot} exit. There are supersingular elliptic curves with quaternionic endomorphism algebras. Nevertheless, over finite fields, we can obtain analogous results if we restrict ourselves to certain subcategories of abelian varieties.

Recall that an abelian variety $A/\bF_q$, with $q$ a power of a prime, comes equipped with a Frobenius endomorphism that induces an action
\begin{align*}
	\Frob_A:T_\ell A\to T_\ell A
\end{align*}
for any $\ell \neq p$ where $T_\ell(A) = \injlim A[\ell^n] \simeq \bZ^{2g}_\ell$. This is a $\bF_q$-linear map of vector spaces. The characteristic polynomial of this operator is denoted $h_A(x):= \fchar(\Frob_A)$, and it is a $q$-Weil polynomial and is isogeny invariant.

By Honda-Tate theory (that thing from the end of local fields) the association
\begin{center}
	isogeny class of $A ~ \leftrightarrow ~$ $h_A(x)$
\end{center}
is injective and allows us to enumerate all abelian varieties over $\bF_q$ up to isogeny. We also have that $h_A(x)$ is squarefree if and only if $\End(A)$ is commutative. 

Deligne came up with an equivalence of categories for \emph{ordinary} abelian varieties over $\bF_q$ analogous to that seen in abelian categories over $\bC$.

We say that $A/\bF_q$ is \emph{ordinary} if half of the $p$-adic roots of $h_A$ are units. 
\begin{thm}[Deligne '69']\label{thm:deligne-equivalence-abelian-char-p-varieties}
	Let $q = p^r$ with $p$ prime. There is an equivalence of categories between the category of ordinary abelian varieties over $\bF_q$ and the category of pairs $(T,F)$ with $T \simeq_\bZ \bZ^{2g}$ and $T \xrightarrow{F}T$ which satisfy the following:
	\begin{itemize}
		\item $F\otimes \bQ$ is semisimple
		\item the roots of $\fchar_{F\otimes \bQ}(x)$ have absolute value $\sqrt{q}$ and half of them are $p$-adic units
		\item There exists $V:T\to V$ such that $FV = VF = q$.
	\end{itemize}
	This is equivalence is given by the functor $A\mapsto (T(A),F(A))$.
\end{thm}

Let's consider the case that $h_A$ is square free.
\begin{itemize}
	\item Fix an \emph{ordinary squarefree} $q$-Weil polynomial $h:$ 
	\item Put $K := \bQ[x]/(h) = \bQ[F]$, an \'etale algebra = product of number fields.
	\item Put $V = q/F$. Deligne's equivalence induces:
	\begin{thm}\label{}
		$ \{\text{abelian varieties over $bF_q$ in $\cC_h$}\}/\sim$ ...
	\end{thm}
\end{itemize}

Didn't get anything else :/
\end{document}
