\documentclass[12pt]{amsart}
\usepackage{csquotes}
\usepackage{comment}
\usepackage{xcolor}
\usepackage[margin=1in]{geometry}
\usepackage{fancyhdr}
\setlength{\headheight}{14pt}
\setlength{\parindent}{2em}
\usepackage{lipsum}
\usepackage{fancyhdr}

\newcommand{\rcomment}[1]{\textcolor{red}{#1}}
\newcommand{\gcomment}[1]{\textcolor{green}{#1}}
\newcommand{\bcomment}[1]{\textcolor{Midnight Blue}{#1}}

\title{Request for funding to attend COMBS 2022}
\author{Isaac Martin}

\begin{document}
    \maketitle
	\section*{Context}
    Algebraic geometry is a robust, far-reaching field of mathematics, but its modern form is often quite inscrutable. There are nonetheless settings where one can get their hands dirty, compute examples, and see how the machine truly runs. One of these is toric geometry, which essentially serves as a playground for the rest of algebraic geometry where constructions are explicit, examples are plentiful, and everything (mostly) makes sense. This conference essentially explores various techniques used to extend the niceties of toric geometry to more general problems. One such technique is the idea of tropicalization, which essentially turns ``continuous'' problems into discrete optimization problems. This was used by Sam Payne to answer a difficult problem about moduli spaces of curves for instance, and is even providing new insight into Mirror Symmetry, a field of math closely related to string theory, via the work of Bernd Siebert and Mark Gross. 

	\section*{Statement of Goals}
	The COMBS 2022 conference is essentially a series of four mini courses on tropical geometry, its related fields, and other so called ``combinatorial methods'' in algebraic geometry which extend ideas from toric geometry in other ways. I don't know precisely what my eventual Ph.D. problem will be, but it will fall broadly into either the theory of moduli spaces, mirror symmetry, tropical geometry, or geometric representation theory, all of which involve many or all of these topics in some form or another. 

	The topics at COMBS also represent an unusually strong connection between the algebraic geometry departments at Cambridge and UT Austin, where I will pursue my Ph.D. Sam Payne, mentioned above, is a professor at Austin who advised the Ph.D. of my Part III Algebraic Geometry professor. Sam also frequently works with Navid Nabijou, my Part III Toric Geometry professor and the organizer of this conference. Bernd Siebert is another of my potential Ph.D. advisors who works on Mirror Symmetry through tropicalization with Mark Gross, who taught my Part III commutative algebra course. Besides the usual networking opportunities (of which there are plenty, in particular, one of the presenters is Farbod Shokrieh, whom I considered working with at the University of Washington) COMBS would give me an opportunity to learn more deeply about the research of my own professors at Cambridge and my potential Ph.D. advisors. To summarize, at this conference I hope to
	\begin{itemize}
		\item Learn more combinatorial techniques in algebraic geometry which will likely be vital to my Ph.D. research
		\item Familiarize myself more deeply with the research of my potential Ph.D. advisors at Austin and my professors at Cambridge
		\item Develop a broader understanding of current techniques in tropical geometry, its related fields, and how these techniques lend themselves to problems in other areas of algebraic geometry.
	\end{itemize}

	\section*{Estimate of Costs}
	The conference runs from the 5\textsuperscript{th} to the 9\textsuperscript{th} of September, so I would likely leave on the 4\textsuperscript{th} and arrive back in Austin on the 10\textsuperscript{th}.

	\begin{itemize}
		\item Round Trip Flight: \$1,876
		\item Round Trip Train from Heathrow: \$104
		\item Lodging: \$553 (\$93 per night)
	\end{itemize}

	Flights are currently quite high, and may go down in price before the conference. I could also book cheaper flights, perhaps by flying to Dublin and then using Ryan Air to get to Stansted. This would make for far more unpleasant travel, but would cut down on both train and plane costs. Conference lodging is arranged at Sidney Sussex College and is priced at \textsterling 76 per night.
\end{document}
