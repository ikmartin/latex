\documentclass[hidelinks,11pt,dvipsnames]{article}
% xcolor commonly causes option clashes, this fixes that
\PassOptionsToPackage{dvipsnames,table}{xcolor}
\usepackage[tmargin=1in, bmargin=1in, lmargin=0.8in, rmargin=1in]{geometry}

%%%%%%%%%%%%%%%%%%%%%%%%%%%%%%%%%%%%%%%%%%%%%%%%%%%%%%%%%%%%%%%%%%%%
%%% For inkscape-figures
%%% Assumes the following directory structure:
%%% master.tex
%%% figures/
%%%     figure1.pdf_tex
%%%     figure1.svg
%%%     figure1.pdf
%%%%%%%%%%%%%%%%%%%%%%%%%%%%%%%%%%%%%%%%%%%%%%%%%%%%%%%%%%%%%%%%%%%%
%\usepackage{import}
\usepackage{pdfpages}
\usepackage{transparent}

\newcommand{\incfig}[2][1]{%
    \def\svgwidth{#1\columnwidth}
    \import{./figures/}{#2.pdf_tex}
}

\pdfsuppresswarningpagegroup=1

% enable synctex for inverse search, whatever synctex is
\synctex=1
\usepackage{float,macrosabound,homework,theorem-env}
\usepackage{microtype}


% font stuff
\usepackage{sectsty}
\allsectionsfont{\sffamily}
\linespread{1.1}

% bibtex stuff
\usepackage[backend=biber,style=alphabetic,sorting=anyt]{biblatex}
\addbibresource{main.bib}

% colored text shortcuts
\newcommand{\blue}[1]{\color{MidnightBlue}{#1}}
\newcommand{\red}[1]{\textcolor{Mahogany}{#1}}
\newcommand{\green}[1]{\textcolor{ForestGreen}{#1}}


% use mathptmx pkg while using default mathcal font
\DeclareMathAlphabet{\mathcal}{OMS}{cmsy}{m}{n}

% fixes the positioning of subscripts in $$ $$
\renewcommand{\det}{\operatorname{det}}

\usetikzlibrary{positioning, arrows.meta}
\newcommand{\here}[2]{\tikz[remember picture]{\node[inner sep=0](#2){#1}}}

%%%%%%%%%%%%%%%%%%%%%%%%%%%%%%%%%%%%%%%%%%%%%%%%%%%%%%%%%%%%%%%%%%%%%
%%% Entry Counter
%%%%%%%%%%%%%%%%%%%%%%%%%%%%%%%%%%%%%%%%%%%%%%%%%%%%%%%%%%%%%%%%%%%%%
\newcounter{entry-counter}
\newcommand{\entry}[1]
{
	\addtocounter{entry-counter}{1}
    \tchap{Entry \arabic{entry-counter}}
	%\addcontentsline{toc}{section}{Entry \arabic{entry-counter}: #1}
	\vspace{-1.5em}
    \begin{center}
		\small \emph{Written: #1}
    \end{center}
}

\usepackage{titling}
\renewcommand\maketitlehooka{\null\mbox{}\vfill}
\renewcommand\maketitlehookd{\vfill\null}


\begin{document}
\begin{center}
	\LARGE
	Definitions and Theorems from Elliptic Curves\\
	\Large
	Isaac Martin \\
    Last compiled \today
\end{center}
\normalsize
\vspace{-2mm}
\hru

\tableofcontents
\newpage

\section{Fermat's Method of Infinite Descent}
\begin{defn}[Rational Triangle]\label{defn:triangle-definitions}
Let $a,b,c$ be the side lengths of a right triangle $\Delta$.
	\begin{enumerate}
		\item $\Delta$ is \emph{rational} if $a,b,c \in \bQ$.
		\item $\Delta$ is \emph{primitive} if $a,b,c \in \bZ$ and are pairwise coprime.
	\end{enumerate}
\end{defn}

\begin{lem}[Lemma 1.1]\label{lem:form-for-prim-triangles}
	Every primitive triangle has side lengths of the form $u^2+v^2$, $2uv,$ and $u^2 - v^2$ for some integers $u > v > 0$.
\end{lem}
\begin{defn}\label{defn:congruent-number}
	$D\in \bQ_{>0}$ is a \emph{congruent number} if there exists a rational triangle $\Delta$ with $\area(\Delta) = 0$. 
\end{defn}
N.B. it suffices to to consider $D$ a positive integer which is squarefree, e.g. $D = 5, 6$ are congruent.

\begin{lem}[Lemma 1.2]\label{lem:congruent-number-condition}
	$D \in \bQ_{>0}$ is congruent if and only if $Dy^{2}=x^3 - x$ for some $x,y \in \bQ$ with $y \neq 0$.
\end{lem}

\begin{thm}
	There is no solution to 
	\begin{align*}
		w^2 = uv(u+v)(u-v)
	\end{align*}
	for $u,v,w \in \bZ$ and $w \neq 0$.
\end{thm}
\begin{lem}\label{lem:fermat-uni-descent-polynomials}
	Let $u, v \in K[t]$ be coprime polynomials. If $\alpha u + \beta v$ is a square for four distinct choices of $(\alpha:\beta) \in \bP^1_K$ then $u,v \in K$. 
\end{lem}

\bigskip

\begin{cor}[1.6 in Lecture]\label{cor:points-on-E-same-for-K-and-functions-on-K}
	Let $E/K$ be an elliptic curve. Then $E(K(t)) = E(K)$.
\end{cor}
\begin{proof}
	Without loss of generality may assume $K = \ol{K}$. By a change of coordinates we may assume $E: ~ y^{2}=x(x-1)(x-\lambda)$ for some $\lambda \in K\setminus \left\{0,1\}\right.$ Suppose $(x,y) \in E(K(t))$. Write $x = \frac{u}{v}$ for coprime polynomials $u,v \in K[t]$. Then
	 \begin{align*}
		w^{2} = uv(u - v)(u - \lambda v)
	\end{align*}
	for some $w \in K[t]$. Because $K[t]$ is a UFD, we get that $u,v,u-v,$ and $u - \lambda v$ are all squares in $K[t]$ and then Lemma
\end{proof}

\bigskip

\section{Remarks on Algebraic Curves}
An algebraic curve is a projective variety of dimension 1. All affine curves are algebraic curves, simply take the equation cutting the variety out, homogenize it with a variable $Z$, and you've got a projective curve. The subset of the curve on which $Z = 1$ recovers the original affine curve.

Throughout these notes, $K$ is a field, $\ol{K}$ is a fixed algebraic closure of $K$, and $G_{\ol{K}/K}$ is the Galois group $\Gal(\ol{K}/K)$.

\subsection{Preliminaries}
When we say \emph{curve} in these notes, we always mean a projective variety of dimension one, and almost always we deal with curves that are smooth.

\begin{prop}\label{prop:local-ring-of-curve-is-dvr}
	Let $C$ be a curve and $P \in C$ be a smooth point. Then $\cO_{C,P} = \olK[C]_P$ is a DVR.
\end{prop}

\begin{defn}\label{defn:normalized-valuation}
	Let $C$ be a curve and $P \in C$ be a smooth point. Then the \emph{normalized valuation on} $\olK[C]_P$ is given by
	\begin{align*}
		\ord_P:\olK[C]_P\to \{0,1,2,3,...\} \cup \{\infty\}, \\
		\ord_P(f) = \sup\{d \in \bZ ~ \mid ~ f\in \frak^d_P\}.
	\end{align*}
	Here, $\frakm_P$ is the unique maximal ideal of $\olK[C]_P$. We extend this function to $\olK(C)\to \bZ\cup\{\infty\}$ by declaring $\ord_P(f/g) = \ord_P(f) - \ord_P(g)$.
\end{defn}
An alternative definition, or at least a slightly more explicit version of the same definition, involves fixing a uniformizer $\pi_P$ of $\olK[C]_P$ (note that the any element $f \in K[C]_P$ for which $\ord_P(f) = 1$ is a valid uniformizer) and declaring $\ord_P(f) = d$ where $d$ is the unique integer such that $f = u\cdot \pi^d$ for some unit $u \in \olK[C]_P^\times$. This is necessarily a nonnegative integer when $f \in K[C]_P$. 

There are two things to note about this definition. First, though $\olK(C) = \Frac(\olK[C]_P) = \Frac(\olK[C])$ regardless of which $P \in C$ we choose, $\ord_P$ does depend on the choice of $P$, quite clearly. Second, when we extend $\ord_P$ to $K(C)$, it is not necessary to additionally add the point $-\infty$ to the codomain. The only point in $K[C]_P$ which evaluates to $\infty$ under $\ord_P$ is $0$, which has no inverse in $K(C)$.

\begin{defn}\label{defn:order-zero-pole-regular}
	Let $C$ be a curve and $P$ a smooth point. The \emph{order of $f$ at $P$} is $\ord_P(f)$.
	\begin{itemize}
		\item If $\ord_P(f) > 0$ then $f$ has a \emph{zero} at $P$.
		\item If $\ord_P(f) < 0$ then $f$ has a \emph{pole} at $P$.
		\item If $\ord(f) \geq 0$ then $f$ is \emph{regular} at $P$ or alternatively is \emph{defined} at $P$. We can evaluate $f(P)$ in this case.
		\item If $\ord_P(f) < 0$, i.e. if $f$ has a pole at $P$, then we write $f(P) = \infty$.
	\end{itemize}
\end{defn}
All of this should be reminiscent of complex analysis, and indeed, all this is identical to that terminology in the case that $K = \bC$.

\begin{prop}\label{prop:finitely-many-points-where-f-has-zero-or-pole}
	Let $C$ be a smooth curve and $f \in \olK(C)$ with $f \neq 0$. Then there are only finitely many points $P \in C$ at which $f$ has a zero or pole. Furthermore, $f$ has no poles if and only if $f \in \olK$.
\end{prop}

The author of these notes is stupid and incessantly ignorant about matters regarding Galois, so we say a few things more about the Galois action. The Galois group  $G_{\olK/K}$ acts on $\bA^n_{\olK}$ by
\begin{align*}
	P^\sigma = (x_1^\sigma,...,x_n^\sigma),
\end{align*}
meaning that $\bA^n_K$ can be characterized by
\begin{align*}
	\bA^n_K = \bA^n(K) = \{P \in \bA^n_{\olK} ~\mid~ P^\sigma = P \text{ for all } \sigma \in G_{\olK/K}\}.
\end{align*}
When we write  $\bA^n$ without specifying the base field, it is implied that we mean $\bA^n_{\olK}$. Similarly, when we write $\bP^n$ we mean  $\left(\bA^{n+1}_{\olK}\setminus\{0\}\right)/\olK^*$, and we define the \emph{set of K-rational points in } $\bP^n$ to be 
\begin{equation}\label{eqn:K-rational-P-points}
	\bP^n(K) = \left\{[x_0,...,x_n] \in \bP^n ~\midd~ x_i \in K ~\text{ for all } ~0\leq i\leq n\right\}.
\end{equation}
\begin{defn}\label{defn:minimal-field-of-definition-for-P}
	Let $P = [x_0,...,x_n]\in \bP^n(\olK)$. The \textbf{minimal field of definition for $P$} is the field
	\begin{align*}
		K(P) = K(x_0/x_i,...,x_n/x_i)
	\end{align*}
	where $x_i$ is (one of) the nonzero coordinate(s) of $P$. Note that different valid choices of $x_i$ yield isomorphic fields when we adjoin elements.
\end{defn}
The Galois group acts on $\bP^n$ in the way one would hope. Given $\sigma \in G_{\olK/K}$,
\begin{align*}
	[x_0,...,x_n]^{\sigma} = [x_0^\sigma,...,x_n^\sigma].
\end{align*}
This action is well defined since
\begin{align*}
	[\lambda x_0, ...,\lambda x_n]^\sigma = \lambda^\sigma[x_0^\sigma,...,x_n^\sigma] = [x_0,...,x_n]^\sigma.
\end{align*}

We have a notion of the rationalization of a curve and a rational curve.
\begin{defn}\label{defn:plane-curve-rational}
	A plane curve $\{f(x,y) = 0 ~\mid~ (x,y) \in K = \ol{K}\} \subseteq \bA^2$ (with $f$ irreducible over $\ol{K}$) is said to be \textbf{rational} if it has a rational parameterization, i.e. $\exists \phi, \psi \in K(t)$ such that

	\begin{enumerate}[(i)]
		\item $\bA^1\to \bA^2$ defined $t \mapsto (\phi(t), \psi(t))$ is an injection on $\bA^1\setminus \{\text{finite set}\}$
		\item $f(\phi(t),\psi(t)) = 0$
	\end{enumerate}
\end{defn}

\subsection{Divisors}
The only codimension subschemes of a curve are the points on the curve. This makes the divisor class group of an algebraic curve $C$ particularly nice.
\begin{defn}\label{defn:divisor-group-of-curve}
	The \textbf{divisor class group} of a curve $C$ is the free abelian group generated by the points of $C$. More explicitly, a divisor $D \in \Div(C)$ is a formal sum
	\begin{align*}
		D = \sum_{P \in C} n_P(P)
	\end{align*}
	where only finitely many of the $n_P$ are nonzero. The \textbf{degree} of a divisor is defined by
	\begin{align*}
		\deg(D) = \sum_{P\in C} n_P.
	\end{align*}
	The \textbf{divisors of degree 0} form a subgroup of $\Div(C)$ which we denote by $\Div^0(C)$.
\end{defn}
The Galois action on divisors is exactly what you'd expect: given $\sigma \in G_{\olK/K}$ we define
\begin{align*}
	D^\sigma = \sum_{ P\in C } n_P(P^\sigma).
\end{align*}
We say that $D$ is \emph{defined over} $K$ if $D^\sigma = D$ for each $\sigma \in G_{\olK/K}$. This does \emph{not} mean that $D$ is defined over $K$ if and only if $P \in K$ for each $n_P \neq 0$ is the formal sum defining $D$, instead, the Galois action could simply permute the nonzero  $P$'s in some way.

\begin{defn}[Riemann-Roch Space]\label{defn:reim-roch-space}
	Let $C$ be a smooth projective curve. The Riemann-Roch space of a $D \in \Div(C)$ is
	\begin{align*}
		\cL(D) = \left\{f \in K(C)^* \midd \div(f) + D \geq 0\right\} \cup \left\{0\right\}
	\end{align*}
	i.e. the $K$-vector space of rational functions on $C$ with "poles no worse than specified by $D$."
\end{defn}
Here, the space $K(C)$ is $\Frac\left(K[x_1,...,x_n]/(F)\right)$.

\section{Geometry of Elliptic Curves}

\subsection{Weierstrass Equations}

An elliptic curve is a genus one curve in $\bP^2$ with a single specified base point on the line at infinity (remember that the line at infinity in $\bP^2$ is the set of points $[X:Y:0]$). After scaling $X$ and $Y$ appropriately an elliptic curve has an equation of the form
\begin{equation}\label{eqn:Weierstrass-form-hom}
	Y^2Z + a_1XYZ + a_3YZ^2 = X^3 + a_2X^2Z + a_4XZ^2+a_6Z^3.
\end{equation}
Here, $O = [0:1:0]$ is the base point and $a_1,...,a_6\in \ol{K}$, and equation (\ref{eqn:Weierstrass-form-hom}). We generally write an elliptic curve in non-homogeneous coordinates $x = X/Z$ and $y = Y/Z$:
\begin{equation}\label{eqn:Weierstrass-form}
	E:~ y^2 + a_1xy+a_3y = x^3 + a_2x^2 + a_4x + a_6,
\end{equation}
and remember that we always have a single extra point at infinity given by $O = [0:1:0]$. If $ a_1,...,a_6 \in K$ then we say that $E$ is \textbf{defined over $K$}.

We can make some simplifications in the cases that $\fchar(\ol{K}) \neq 2,3$. If $\fchar(\ol{K}) \neq 2$ then we can complete the square:
\begin{align*}
	y \mapsto \frac{1}{2}(y - a_1x - a_3)
\end{align*}
to get an equation
\begin{align*}
	y^2 = 4x^3 + b_2x^2 + 2b_4x + b_6,
\end{align*}
where
\begin{align*}
	b_2 = a_1^2 + 4a_4,\hspace{1.5em} b^4 = 2a_4+a_1a_3,\hspace{1.5em} b_6 = a_3^2+4a_6.
\end{align*}
The following are useful quantities:
\begin{align*}
	b_8 &= a_1^2a_6+4a_2a_6-a_1a_3a_4+a_2a_3^2-a_4^2, & \\
	c_4 &= b_2^2 - 24b_4, & \\
	c_6 &= -b_2^3+36b_2b_4 - 216b_6, & \\
	\Delta &= -b_2^2b_8-8b_4^3-27b_6^2+9b_2b_4b_6, & \text{the \textbf{discriminant} of $E$} \\
	j &= c_4^3/\Delta, &\text{the \textbf{j-invariant} of $E$} \\
	\omega &= \frac{dx}{2y+a_1x+a_3} = \frac{dy}{3x^2+2a_2x+a_4-a_1y} &\text{the \textbf{invariant differential}}.
\end{align*}
In the case that $\fchar{\ol{K}} \neq 2,3$ we can make an additional substitution
\begin{align*}
	(x,y) \mapsto \left(\frac{x - 3b_2}{36}, \frac{y}{108}\right)
\end{align*}
to eliminate the $x^2$ term and obtain the simpler equation
\begin{align*}
	E:~y^2 = x^3 - 27c_4x-54c_6
\end{align*}
for the elliptic curve.

The last three terms in the above table of quantities are of particular interest in classifying elliptic curves up to isomorphism.








\end{document}
