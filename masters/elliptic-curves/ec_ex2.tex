\documentclass[hidelinks,11pt,dvipsnames]{article}
% xcolor commonly causes option clashes, this fixes that
\PassOptionsToPackage{dvipsnames,table}{xcolor}
\usepackage[tmargin=1in, bmargin=1in, lmargin=0.8in, rmargin=1in]{geometry}

%%%%%%%%%%%%%%%%%%%%%%%%%%%%%%%%%%%%%%%%%%%%%%%%%%%%%%%%%%%%%%%%%%%%
%%% For inkscape-figures
%%% Assumes the following directory structure:
%%% master.tex
%%% figures/
%%%     figure1.pdf_tex
%%%     figure1.svg
%%%     figure1.pdf
%%%%%%%%%%%%%%%%%%%%%%%%%%%%%%%%%%%%%%%%%%%%%%%%%%%%%%%%%%%%%%%%%%%%
%\usepackage{import}
\usepackage{pdfpages}
\usepackage{transparent}

\newcommand{\incfig}[2][1]{%
    \def\svgwidth{#1\columnwidth}
    \import{./figures/}{#2.pdf_tex}
}

\pdfsuppresswarningpagegroup=1

% enable synctex for inverse search, whatever synctex is
\synctex=1
\usepackage{float,macrosabound,homework,theorem-env}
\usepackage{microtype}


% font stuff
\usepackage{sectsty}
\allsectionsfont{\sffamily}
\linespread{1.1}

% bibtex stuff
\usepackage[backend=biber,style=alphabetic,sorting=anyt]{biblatex}
\addbibresource{main.bib}

% colored text shortcuts
\newcommand{\blue}[1]{\color{MidnightBlue}{#1}}
\newcommand{\red}[1]{\textcolor{Mahogany}{#1}}
\newcommand{\green}[1]{\textcolor{ForestGreen}{#1}}


% use mathptmx pkg while using default mathcal font
\DeclareMathAlphabet{\mathcal}{OMS}{cmsy}{m}{n}

% fixes the positioning of subscripts in $$ $$
\renewcommand{\det}{\operatorname{det}}

\usetikzlibrary{positioning, arrows.meta}
\newcommand{\here}[2]{\tikz[remember picture]{\node[inner sep=0](#2){#1}}}

%%%%%%%%%%%%%%%%%%%%%%%%%%%%%%%%%%%%%%%%%%%%%%%%%%%%%%%%%%%%%%%%%%%%%
%%% Entry Counter
%%%%%%%%%%%%%%%%%%%%%%%%%%%%%%%%%%%%%%%%%%%%%%%%%%%%%%%%%%%%%%%%%%%%%
\newcounter{entry-counter}
\newcommand{\entry}[1]
{
	\addtocounter{entry-counter}{1}
    \tchap{Entry \arabic{entry-counter}}
	%\addcontentsline{toc}{section}{Entry \arabic{entry-counter}: #1}
	\vspace{-1.5em}
    \begin{center}
		\small \emph{Written: #1}
    \end{center}
}

\usepackage{titling}
\renewcommand\maketitlehooka{\null\mbox{}\vfill}
\renewcommand\maketitlehookd{\vfill\null}


\begin{document}
\pagestyle{empty}
	\LARGE
\begin{center}
	Elliptic Curves Example Sheet 2\\
	\Large
	Isaac Martin \\
    Last compiled \today
\end{center}
\normalsize
\vspace{-2mm}
\hru

\begin{homework}[e]
	\prob Find all points defined over the field $\bF_{13}$ of 13 elements on the elliptic curve
	\begin{align*}
		y^2 = x^3 + x + 5
	\end{align*}
	and show that they form a cyclic group. Find an example of an elliptic curve over $\bF_{13}$ for which this group is not cyclic. Are there any examples where the group requires more than two generators?
	\begin{prf}
		By brute force (i.e. a Python script) one can check that
		\begin{align*}
			E(\bF_{13}) = \{O_E, (1, 10), (2, 7), (3, 3), (4, 12), (9, 12), (10, 3), (11, 7), (12, 10)\}.
		\end{align*}
		Counting, this means that $\#E(\bF_{13}) = 9$. There are two groups of order 9, $\bZ/9\bZ$ and $\bZ/3\bZ \times \bZ/3\bZ$. One can choose any non-identity point in $E(\bF_{13})$ and add it to itself three times to verify that it is not order 3, and must therefore be order 9. Hence this is cyclic.

		I didn't finish this question.
	\end{prf}
	\prob Let $A$ be an abelian group and let $q:A\to \bZ$ be a map satisfying
	\begin{align*}
		q(x+y)+q(x-y) = 2q(x)+2q(y).
	\end{align*}
	Prove that $A$ is a quadratic form.
	\begin{prf}
		Recall that to be a quadratic form, $q$ must satisfy
		\begin{enumerate}[(i)]
			\item $q(nx) = n^2q(x)$ for all $x \in A$ and $n \in \bZ$ 
			\item $\langle x,y\rangle = q(x+y)-q(x)-q(y)$ is a $\bZ$-bilinear pairing.
		\end{enumerate}
        We prove these properties by induction.
		
		\bigskip

        \noindent \textbf{(i)}~ Notice that $q(1\cdot x) = 1^2q(x)$ trivially, $q(0+0) + q(0-0) = 2q(0)+2q(0)$ so $q(0) = 0$, and $q(2x) = 2q(x) + 2q(x) - q(x - x) = 4q(x)$ for all $x\in A$; hence, (i) holds for $n = 0,1$ and $2$. Now suppose that (i) holds for all positive values $k$ with $n > k > 2$. By induction,
		\begin{align*}
			q(nx) &= 2q((n-1)x) + 2q(x) - q((n-2)x) \\
				  &= 2(n-1)^2q(x) + 2q(x) - (n-2)^2q(x) \\
				  &= (2n^2 - 4n + 2 + 2 - n^2 + 4n - 4)q(x) = n^2q(x),
		\end{align*}
		so (i) holds for all values $n \geq 0$. 

		Finally, if $n \geq 0$ then
		\begin{align*}
			q(-nx) &= q(x - (n+1)x) = 2q(x)+2q((n+1)x) - q(x+(n+1)x) \\
				   &= 2q(x) + 2(n+1)^2q(x) - (n+2)^2q(x) \\
				   &= (2 + 2n^2+4n+2 - n^2 - 4n - 4)q(x) = n^2 q(x).
		\end{align*}
		This means $q(nx) = n^2q(x)$ for all $n \in \bZ$ and $x \in A$.
		
		\bigskip

		\noindent \textbf{(ii)}~ Since the pairing $\langle x,y\rangle$ is invariant under the permutation $x \mapsto y$ and $y \mapsto x$, it suffices to prove that $\langle -,-\rangle$ is $\bZ$-linear in the first coordinate, i.e. that 
		\begin{enumerate}[(a)]
			\item $\langle nx,y\rangle = n\langle x,y\rangle$ for all $n \in \bZ$ and $x,y \in A$
			\item $\langle x+y,z\rangle = \langle x,y\rangle + \langle y,z\rangle$ for all $x,y,z \in A$.
		\end{enumerate}
		
		\bigskip

		\noindent\textbf{(a)}~ We first treat the case that $n \geq 0$. This induction argument requires that the statement hold true for $n-1,n-2$ and $n - 3$, so we need the cases that $n = 0, 1$ and $2$ before proceeding to the induction step.

		\noindent \hspace{1em}\underline{$n = 0$}:~ $\langle 0 \cdot x,y\rangle = q(0\cdot x + y) - q(0\cdot x) - q(y) = q(y) - q(y) = 0 = 0\cdot \langle x,y\rangle$.

		\noindent \hspace{1em}\underline{$n = 1$}:~ This is trivially satisfied.

		\noindent \hspace{1em}\underline{$n = 2$}:~ We invoke the equality $q(2x) = 4q(x)$ provided by (i) here.
		\begin{align*}
			\langle 2x,y\rangle &= q(2x + y) - q(2x) - q(y) \\
				&= q(x + (x+y)) - q(2x) - q(y) \\
				&= 2q(x) + 2q(x+y) - q(x - (x+y)) - 4q(x) - q(y) \\
				&= 2q(x+y) - 2q(x) - q(-y) - q(y) \\
				&= 2\big(q(x+y) -q(x) - q(y)\big) = 2\langle x,y\rangle.
		\end{align*}
        
		Assume now that $n > 2$ and that $\langle kx,y\rangle = k\langle x,y\rangle$ holds for $n > k\geq 0$. This means
		\begin{align*}
			\langle kx,y\rangle = q(kx+y) - q(kx) - q(y) = k\big(q(x+y) - q(x) - q(y)\big)
		\end{align*}
		for $0 \leq k < n$ and so
		\begin{align*}\tag{$\ast$}\label{eqn:problem2-(ii)}
			q(kx + y) &~=~ k\big(q(x+y) - q(x) - q(y)\big) + k^2q(x) + q(y) \\
					  &~=~ kq(x+y) + (k^2 - k)q(x) - (k-1)q(y).
		\end{align*}
		We can now prove the desired statement:
		\begin{align*}
			\langle nx,y\rangle &~=~ q(nx + y) - q(nx) - q(y) \\
			  &~=~ 2q(x) + 2q\big((n-1)x + y\big) - q\big((n-2)x + y\big) - q(nx) - q(y) \\
			  &~=~ 2q(x) + \overbrace{2(n-1)(q(x+y)+2(n-1)(n-2)q(x) - 2(n-2)q(y))}^{2q\big((n-1)x + y\big) \text{ by (\ref{eqn:problem2-(ii)}) }} \\
			  &\hspace{1.5em} - \overbrace{(n-2)q(x+y) - (n-2)(n-3)q(x) + (n-3)q(y)}^{-q\big((n-2)x+y\big) \text{ by (\ref{eqn:problem2-(ii)})}} - n^2q(x) - q(y) \\
			  &~=~ \big(2(n-1) -(n-2)\big)q(x+y) +\big(2 + 2(n-1)(n-2)-(n-2)(n-3)-n^2\big)q(x) \\
			  &\hspace{1.5em}+ \big(-2(n-2)+(n+3)-1\big)q(y) \\
			  &~=~ nq(x+y) - nq(x) - nq(y) \\
			  &~=~ n\langle x,y\rangle.
		\end{align*}
        
		We now must treat the case that $n < 0$. If $n = -1$ we get
		\begin{align*}
			\langle -x,y\rangle &~=~ q(-x + y) - q(-x) - q(y) \\
			  &~=~ 2q(x) + 2q(y) - q(x+y) - q(x) - q(y) = -\langle x,y\rangle
		\end{align*}
		without too much trouble. Using this together with the $n\geq 0$ case gives us
		 \begin{align*}
			\langle -nx,y\rangle = - \langle nx,y\rangle
		\end{align*}
	    for $n \geq 0$, so we conclude that $\langle nx,y\rangle = n\langle x,y\rangle$ for all $n \in \bZ$.

		\bigskip

		\noindent \textbf{(b)}~ Let $x,y,z \in A$ be arbitrary elements. By expanding $\langle -,- \rangle$ it can be seen that
		\begin{align*}
			\langle x+y,z\rangle - \langle x,z \rangle - \langle y,z \rangle = 0
		\end{align*}
		if and only if
		\begin{align*}
			q(x+y+z) = q(x+y) + q(x+z) + q(y+z) - q(x) - q(y) - q(z).
		\end{align*}
		We prove this later equality. We first examine what we obtain by considering $q(x+y+z)$ and swapping each "$+$" one at a time. By assumption, we have that
		\begin{equation}\label{eqn:prob2-1}
			q(x+y+z) + q(x+y-z) = 2q(x+y) + 2q(z),
		\end{equation}
		\begin{equation}\label{eqn:prob2-2}
			q(x+y-z)+q(x-y-z)) = 2q(x - z) + 2q(y)
		\end{equation}
		and
		\begin{equation}\label{eqn:prob2-3}
			q(x - y - z) + q(- x - y - z) = 2q(-y-z)+2q(x).
		\end{equation}
		Adding equations (\ref{eqn:prob2-1}) and (\ref{eqn:prob2-3}) together and subtracting equation (\ref{eqn:prob2-2}) gives us
		\begin{align*}
			q(x+y+z) + q(-x-y-z) = 2q(x+y) + 2q(z) - 2q(x-z) - 2q(y) + 2q(y+z) + 2q(x),
		\end{align*}
		while recalling that $q(-x) = q(x)$, dividing both sides by $2$ and performing some convenient rearranging gives us
		\begin{align*}
			q(x+y+z) = \big[q(x+y) q(y+z) - q(y)\big] + \big[q(x - z) + q(z) + q(x)\big].
		\end{align*}
		Finally, we have that $q(x - z) = 2q(x) + 2q(z) - q(x + z)$. Substituting this in for $q(x-z)$, combining like terms, and rearranging a final time yields
		\begin{align*}
			q(x+y+z) = q(x+y) + q(x+z) + q(y+z) - q(x) - q(y) - q(z)
		\end{align*}
		as desired.
	\end{prf}
	\prob Find a translation-invariant differential $\omega$ on the multiplicative group $\bG_m$. Show that if $[n]:\bG_m \to \bG_m$ is the endomorphism $x \mapsto x^n$ then $[n]^*\omega = n\omega$.
	\begin{prf}
		An invariant differential of a formal group law $F \in R\llbracket X,Y\rrbracket$ is a differential form
		\begin{align*}
			\omega = P(T)dT \in R\llbracket T\rrbracket dT
		\end{align*}
		which satisfies
		\begin{align*}
			\omega \circ F(&T,S) = \omega(T) \\
						   &\iff \\
			P(F(T,S))&F_X(T,S) = P(T)
		\end{align*}
		where $F_X(T,S)$ is the partial derivative of $F$ in the first variable. The formal group law of $\bG_m$ is $F(X,Y) = X + Y + XY = (1 + X)(1 + Y) -1$, and its partial derivative in $X$ is $F_X(X,Y) = 1 + Y$. We are therefore looking for some $P(T) \in R\llbracket T\rrbracket$ such that
		\begin{align*}
			P((1+T)(1+S)-1) \cdot (1+S) = P(T).
		\end{align*}
		It is fortunate that we discussed the element $\frac{1}{1 - X} = 1 + x + x^2 + ... \in R\llbracket T\rrbracket$ in class -- a slight modification, the power series $P(T) = \frac{1}{1 + T} = 1 - T + T^2 - T^3 + ... \in R\llbracket T\rrbracket$, will do the trick:
		\begin{align*}
			P((1+T)(1+S)-1)\cdot (1 + S) = \frac{1}{(1+T)(1+S) - 1 + 1}\cdot (1 + S) = \frac{1}{1 + T} = P(T).
		\end{align*}
		Hence the differential form $\omega = \frac{1}{1 + T}$ is an invariant differential of the multiplicative formal group law.
	\end{prf}

	\prob[\textsc{Exercise 6.}] Show that if $\phi\in \End(E)$ then there exists $\tr(\phi) \in \bZ$ such that 
	\begin{align*}
		\deg([n] + \phi) = n^2 + n\tr(\phi) + \deg(\phi)
	\end{align*}
	for all $n\in \bZ$. Establish the following properties:
	\begin{enumerate}[(i)]
		\item $\tr(\phi+\psi)$ = $\tr(\phi) + \tr(\psi)$,
		\item $\tr(\phi^2) = \tr(\phi)^2 - 2\deg(\phi)$,
        \item $\phi^2 - [\tr(\phi)]\phi + [\deg(\phi)] = 0$
	\end{enumerate}

	\prob[\textsc{Exercise 9.}] Let $E/\bF_q$ be an elliptic curve with $p$ an odd prime. Show that there exists an elliptic curve $E'/\bF_p$ with
	\begin{align*}
		\# E(\bF_p) + \#E'(\bF_p) = 2(p+1).
	\end{align*}
	Show further that the groups $E(\bF_p) \times E'(\bF_p)$ and $E(\bF_{p^2})$ have the same order, but need not be isomorphic.

	\prob[\textsc{Exercise 10.}] Let $E$ be an elliptic curve over $\bF_p$ with $p$ a prime and $\#E(\bF_p) = p + 1 - a$, and let $\phi:E\to E$ be the $p$-power Frobenius, i.e. $\phi:(x,y)\mapsto (x^p,y^p)$. Let $\psi = [a] - \phi$.
	\begin{enumerate}[(i)]
		\item Show that $\phi\circ \psi = \psi\circ \phi = [p]$.
		\item Show that if $\psi$ is separable then $E[p^r] \cong \bZ/p^r\bZ$ for all $r \geq 1$.
		\item Show that if $p \geq 5$ and $E[p] = 0$ then $\#E(\bF_p) = p + 1$.
	\end{enumerate}
	\begin{prf}$ $
		\begin{enumerate}[(i)]
			\item The map $\psi$ is the difference of isogenies and is therefore itself an isogeny. In particular, this means that $\psi$ is a rational map. Since the Frobenius endomorphism $F:x\mapsto x^p$ on $\ol{\bF_p} \to \ol{\bF_p}$ is a field homomorphism, it commutes with addition and multiplication on the level of field elements, and therefore it commutes with rational functions $g \in \ol{\bF_p}(E)$. This in turn implies that $\phi \circ \psi = \psi \circ \phi,$ since the Frobenius endomophism commutes with the rational functions that locally present $\psi$.

			Applying problem 6 part (iii) and recalling that $a = \tr(\phi)$, we have
			\begin{align*}
				[\tr(\phi)]\phi - \phi^2 - [\deg(\phi)] = ([a] - \phi)\circ\phi - [\deg(\phi)] = 0.
			\end{align*}
			In class, we used the fact that $\deg(\phi) = p$, so the above equation reduces to $\psi\circ \phi = [p]$.


			\item We have the following string of equalities:
				\begin{align*}
					\#E[p^r] &= \#\ker([p^r]) \\
							 &= \#\ker(\phi^r\circ\psi^r) \\
							 &= \#\ker(\psi^r) \\
							 &= \deg(\psi^r) = p^r.
				\end{align*}
				That $\#\ker(\phi^r\circ \psi^r) = \#\ker(\psi^r)$ follows from the fact that $\phi$ is injective. To see this, recall that the Frobenius map $x\mapsto x^p$ is injective as a map $\ol{\bF_p} \to \ol{\bF_p}$ because \emph{every} field homomorphism is injective. The map $(x,y) \mapsto (x^p,y^p)$ is therefore also injective.

				Now that we know the cardinality of $E[p^r]$, we know that $E[p] \cong \bZ/p\bZ$ since this is the only group of order $p$ up to isomorphism. Inducting on $r$, we have that $E[p^r] \cong \bZ/p^r\bZ$ or $E[p^r] \cong \bZ/p^{r-1}\bZ \times \bZ/p\bZ$ since $E[p^r]$ must both contain $E[p^{r-1}] \cong \bZ/p^{r-1}\bZ$ as a subgroup and have cardinality $p^r$. The latter option is impossible since all of its elements have order at most $p - 1$ and $E[p^{r-1}] \cong \bZ/p^{r-1}\bZ$ by the inductive hypothesis. It must then be the case that $E[p^r] \cong \bZ/p^r\bZ$ for all $r \geq 1$.

			\item Let $\omega = \frac{dx}{y}$,and note that it an invariant differential of $E$. We have that
				\begin{align*}
					\phi^*(\omega) = \frac{d(x^p)}{y^p} = \frac{px^{p-1}dx}{y^p} = 0,
				\end{align*}
				and by Lemma 6.3 from lecture,
				\begin{align*}
					\psi^*\omega = ([a] - \phi)^* \omega &= [a]^*\omega - \phi^*\omega = [a]^*\omega = a\cdot \omega
				\end{align*}
				where the final equality follows from problem 3. We know that $\psi$ is inseparable by part (ii) since $E[p] = 0$, hence the induced map $\psi^*:\Omega_E\to \Omega_E$ is zero by Lemma 6.4 from lecture. In particular, this means that $a\cdot \omega = 0$, and because $\omega \neq 0$ and $\Omega_E$ forms a $\bF_p$-vector space, $a = 0$ in $\bF_p$. Hasse's bound tells us that $a \leq 2\sqrt{p}$, so $a < p$ when $p \geq 5$ and therefore $a = 0$ in $\bZ$. Since $\#E(\bF_p) = p + 1 - a$, we conclude that $\#E(\bF_p) = p + 1$.
		\end{enumerate}
	\end{prf}
\end{homework}

\end{document}
