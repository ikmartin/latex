\documentclass[hidelinks,11pt,dvipsnames]{article}
% xcolor commonly causes option clashes, this fixes that
\PassOptionsToPackage{dvipsnames,table}{xcolor}
\usepackage[tmargin=1in, bmargin=1in, lmargin=0.8in, rmargin=1in]{geometry}

%%%%%%%%%%%%%%%%%%%%%%%%%%%%%%%%%%%%%%%%%%%%%%%%%%%%%%%%%%%%%%%%%%%%
%%% For inkscape-figures
%%% Assumes the following directory structure:
%%% master.tex
%%% figures/
%%%     figure1.pdf_tex
%%%     figure1.svg
%%%     figure1.pdf
%%%%%%%%%%%%%%%%%%%%%%%%%%%%%%%%%%%%%%%%%%%%%%%%%%%%%%%%%%%%%%%%%%%%
%\usepackage{import}
\usepackage{pdfpages}
\usepackage{transparent}

\newcommand{\incfig}[2][1]{%
    \def\svgwidth{#1\columnwidth}
    \import{./figures/}{#2.pdf_tex}
}

\pdfsuppresswarningpagegroup=1

% enable synctex for inverse search, whatever synctex is
\synctex=1
\usepackage{float,macrosabound,homework,theorem-env}
\usepackage{microtype}


% font stuff
\usepackage{sectsty}
\allsectionsfont{\sffamily}
\linespread{1.1}

% bibtex stuff
\usepackage[backend=biber,style=alphabetic,sorting=anyt]{biblatex}
\addbibresource{main.bib}

% colored text shortcuts
\newcommand{\blue}[1]{\color{MidnightBlue}{#1}}
\newcommand{\red}[1]{\textcolor{Mahogany}{#1}}
\newcommand{\green}[1]{\textcolor{ForestGreen}{#1}}


% use mathptmx pkg while using default mathcal font
\DeclareMathAlphabet{\mathcal}{OMS}{cmsy}{m}{n}

% fixes the positioning of subscripts in $$ $$
\renewcommand{\det}{\operatorname{det}}

\usetikzlibrary{positioning, arrows.meta}
\newcommand{\here}[2]{\tikz[remember picture]{\node[inner sep=0](#2){#1}}}

%%%%%%%%%%%%%%%%%%%%%%%%%%%%%%%%%%%%%%%%%%%%%%%%%%%%%%%%%%%%%%%%%%%%%
%%% Entry Counter
%%%%%%%%%%%%%%%%%%%%%%%%%%%%%%%%%%%%%%%%%%%%%%%%%%%%%%%%%%%%%%%%%%%%%
\newcounter{entry-counter}
\newcommand{\entry}[1]
{
	\addtocounter{entry-counter}{1}
    \tchap{Entry \arabic{entry-counter}}
	%\addcontentsline{toc}{section}{Entry \arabic{entry-counter}: #1}
	\vspace{-1.5em}
    \begin{center}
		\small \emph{Written: #1}
    \end{center}
}

\usepackage{titling}
\renewcommand\maketitlehooka{\null\mbox{}\vfill}
\renewcommand\maketitlehookd{\vfill\null}


\begin{document}
\pagestyle{empty}
	\LARGE
\begin{center}
	Toric Geometry: \\
	Theorems and Definitions \\
	
	\bigskip
	\Large
	Isaac Martin \\
	Lent 2022 \\
    Last compiled \today
\end{center}
\normalsize
\vspace{-2mm}
\hru

\tableofcontents
\newpage

\setcounter{section}{0}

\section{Notation and Preliminaries}
\begin{defn}\label{defn:basic-notation}$ $
	\begin{itemize}
		\item $\frakh = \left\{\tau \in \bC \midd \Im(\tau) > 0\right\} $ is the upper half plane in $\bC$ 
		\item $\GL_2(\bR)^+ = \left\{g \in \GL_2(\bR) \midd \det(g) > 0\right\} $ is the set of all automorphisms of $\bR^2$ with positive determinant
		\item $\SL_2(\bR) = \left\{g \in \GL_2(\bR) \midd \det (g) = 1\right\} $ is the standard special linear subgroup of $\GL_2(\bR)$
		\item $\SL_2(\bZ) = \left\{\begin{pmatrix} a & b \\ c & d \end{pmatrix} \in \SL_2(\bR) \midd a,b,c,d \in \bZ\right\} $ is the special linear subgroup of $\GL_2(\bZ)$.
	\end{itemize}
\end{defn}

\begin{lem}\label{lem:transitive-GL2-action-on-H}
	The group  $GL_2(\bR)^+$ acts transitively on $\frakh$ by M\"obius transformations (fractional linear transformations). That is, for any $g = \begin{pmatrix}a & b \\ c & d\end{pmatrix} \in \GL_2(\bR)$ and any $\tau = x + iy \in \frakh$,
	 \begin{align*}
		\begin{pmatrix}
			a & b \\
			c & d
		\end{pmatrix} \tau = \frac{a\tau + b}{c\tau + d} \in \frakh
	\end{align*}
	and 
	\begin{align*}
		\tau = x + iy = 
        \begin{pmatrix}
		    y & x \\
		    0 & 1
    	\end{pmatrix}i.
	\end{align*}
\end{lem}

\begin{lem}\label{lem:weight-k-action-on-mero-func}
	Suppose $k \in \bZ$, $f:\frakh \to \bC$ is a meromorphic function, and $g \in \GL_2(\bR)^+$. We define the \textbf{weight k action of $g$} on $f$ to be the right action $\left(f \mapsto f|_k[g]\right)$ where $f|_k[g]:\frakh \to \bC$ is a meromorphic function defined
	\begin{align*}
		f |_k[g](\tau) = \det(g) f(g\tau)j(g,\tau)^k,
	\end{align*}
	where if $g = \begin{pmatrix}a & b \\ c & d\end{pmatrix}$ then $j(g,\tau) = (c\tau + d)$.
\end{lem}

\begin{defn}\label{defn:weakly-mod-func-to-mod-forms}
	Let $k \in \bZ$ and let $\Gamma \leq SL_2(\bZ)$ be a subgroup of finite index. A \textbf{weakly modular function} of weight $k$ and level $\Gamma$ is a meromorphic function $f$ on $\frakh$ such that for all  $\gamma \in \Gamma$, $f|_k[\gamma] = f$. A \textbf{modular function} (of weight $k$ and level $\Gamma$) is a weakly modular function which is meromorphic at $\infty$. A \textbf{modular form} is a weakly modular function which is holomorphic in $\frakh$ and at $\infty$. A \textbf{cuspoidal modular form} is a modular form which vanishes at $\infty$. 
\end{defn}

\section{Modular Forms on \texorpdfstring{$SL_2(\bZ)$}{the Special Linear Group}}
There are two elements of $\SL_2(\bZ)$ in which we will be particularly interested. These are given below.
\begin{equation}\label{eqn:S-T_matrices}
	
\end{equation}
\begin{thm}\label{thm:fund-domain-SL2}
	We let $\rho = e^{i\frac{2pi}{3}}$ and define
	\begin{align*}
		\cF &= \left\{\tau \in \frakh \midd \Re(\tau) \leq \frac{1}{2}, |\tau| \geq 1\right\}.
	\end{align*}
	Then
	\begin{enumerate}[(a)]
		\item Every $\tau \in \frakh$ is in the $\ol{\Gamma(1)}$ orbit of an element of $\cF$
		\item If $\tau \in \frakh$, then $\Stab_{\ol{\Gamma(1)}}(\tau) = \{1\}$, except
			\begin{align*}
				\Stab_{\ol{\Gamma(1)}}(i) = \{1, S\}, \buffword{1em}{and} \Stab_{\ol{\Gamma(1)}}(\rho) = \{1, S, ST, (ST)^2\}.
			\end{align*}
		\item $S$, $T$ generate $\ol{\Gamma(1)}$.
	\end{enumerate}
\end{thm}
\begin{prf}$ $
	\begin{enumerate}[(a)]
		\item Suppose $\tau \in \frakh$. If $\gamma \in \ol{\Gamma(1)}$ with $\gamma = \begin{pmatrix} a & b \\ c & d \end{pmatrix}$ then
			\begin{align*}
				\Im(\gamma \tau) = \frac{\Im(\tau)}{|c\tau + d|^2}.
			\end{align*}
			 The set of numbers $c\tau + d$ form a subset of the lattice $\bZ \tau \oplus \bZ$ in $\bC$ and in particular form a discrete subset of $\bC$, so there must be a  $\gamma_0 \in \SL_2(\bZ)$ for which $|c\tau + d|$ is minimal. This then implies that $\Im(\gamma_0\tau) = \frac{\Im(\tau)}{|c\tau+d|^2}$. We shall use this later in the proof.

			 Focus now on $\Re(\gamma_0 \tau)$. By multiplying 
	\end{enumerate}
\end{prf}


\end{document}
