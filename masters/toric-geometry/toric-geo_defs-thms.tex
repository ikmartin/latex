\documentclass[hidelinks,11pt,dvipsnames]{article}
% xcolor commonly causes option clashes, this fixes that
\PassOptionsToPackage{dvipsnames,table}{xcolor}
\usepackage[tmargin=1in, bmargin=1in, lmargin=0.8in, rmargin=1in]{geometry}

%%%%%%%%%%%%%%%%%%%%%%%%%%%%%%%%%%%%%%%%%%%%%%%%%%%%%%%%%%%%%%%%%%%%
%%% For inkscape-figures
%%% Assumes the following directory structure:
%%% master.tex
%%% figures/
%%%     figure1.pdf_tex
%%%     figure1.svg
%%%     figure1.pdf
%%%%%%%%%%%%%%%%%%%%%%%%%%%%%%%%%%%%%%%%%%%%%%%%%%%%%%%%%%%%%%%%%%%%
%\usepackage{import}
\usepackage{pdfpages}
\usepackage{transparent}

\newcommand{\incfig}[2][1]{%
    \def\svgwidth{#1\columnwidth}
    \import{./figures/}{#2.pdf_tex}
}

\pdfsuppresswarningpagegroup=1

% enable synctex for inverse search, whatever synctex is
\synctex=1
\usepackage{float,macrosabound,homework,theorem-env}
\usepackage{microtype}


% font stuff
\usepackage{sectsty}
\allsectionsfont{\sffamily}
\linespread{1.1}

% bibtex stuff
\usepackage[backend=biber,style=alphabetic,sorting=anyt]{biblatex}
\addbibresource{main.bib}

% colored text shortcuts
\newcommand{\blue}[1]{\color{MidnightBlue}{#1}}
\newcommand{\red}[1]{\textcolor{Mahogany}{#1}}
\newcommand{\green}[1]{\textcolor{ForestGreen}{#1}}


% use mathptmx pkg while using default mathcal font
\DeclareMathAlphabet{\mathcal}{OMS}{cmsy}{m}{n}

% fixes the positioning of subscripts in $$ $$
\renewcommand{\det}{\operatorname{det}}

\usetikzlibrary{positioning, arrows.meta}
\newcommand{\here}[2]{\tikz[remember picture]{\node[inner sep=0](#2){#1}}}

%%%%%%%%%%%%%%%%%%%%%%%%%%%%%%%%%%%%%%%%%%%%%%%%%%%%%%%%%%%%%%%%%%%%%
%%% Entry Counter
%%%%%%%%%%%%%%%%%%%%%%%%%%%%%%%%%%%%%%%%%%%%%%%%%%%%%%%%%%%%%%%%%%%%%
\newcounter{entry-counter}
\newcommand{\entry}[1]
{
	\addtocounter{entry-counter}{1}
    \tchap{Entry \arabic{entry-counter}}
	%\addcontentsline{toc}{section}{Entry \arabic{entry-counter}: #1}
	\vspace{-1.5em}
    \begin{center}
		\small \emph{Written: #1}
    \end{center}
}

\usepackage{titling}
\renewcommand\maketitlehooka{\null\mbox{}\vfill}
\renewcommand\maketitlehookd{\vfill\null}


\begin{document}
\pagestyle{empty}
	\LARGE
\begin{center}
	Toric Geometry: \\
	Theorems and Definitions \\
	
	\bigskip
	\Large
	Isaac Martin \\
	Lent 2022 \\
    Last compiled \today
\end{center}
\normalsize
\vspace{-2mm}
\hru

\tableofcontents
\newpage

\setcounter{section}{0}

\section{Dictionary}
Toric geometry is concerned with the construction of varieties and schemes given by specifying semigroups and fans and other combinatorial objects. It is therefore useful to fix certain symbols.
\begin{itemize}
	\item $N$: We define $N = \Hom_{\Grp}(\bC^*, (\bC^*)^n)$ and note that $N \cong \bZ^{n}$. 
	\item $M$ : We define $M$ to be the dual lattice of $N$, $M = \Hom_\bZ(N,\bZ) \cong \bZ^n$.
	\item $N_\bR$ and  $M_\bR$ : We define $N_\bR = N \otimes_\bZ \bR \cong \bR^n$ and $M_\bR = M \otimes_\bR \bR \cong \bR^n$.
\end{itemize}

\newpage
\section{What makes a toric variety?}

\subsection{Tori}
\subsection{Toric Varieties}
\subsection{Cones and Fans}
Throughout this section, let $T \cong \left(\bC^*\right)^n$ and $N = \Hom_{\Grp}(\bC^*, T) \cong \bZ^n$. Note that $N$ is the collection of 1-parameter subgroups of $T$, or the set of cocharacters if you prefer that terminology. In addition, every variety is an integral separated scheme of finite type over $\Spec \bC$ unless otherwise specified.

\begin{defn}\label{defn:rational-poly-cone}
	A \emph{rational polyhedral cone} $\sigma$ in $N$ is a set $\sigma \subseteq N_\bR = N \otimes_\bZ\bR \cong \bR^n$ given by the positive span of some finite subset of $N_\bR$, i.e. a set
	\begin{align*}
		\sigma = \cone(v_1,...,v_k) = \left\{\sum_{i=1}^{k}c_iv_i \midd c_i \in \bR_{\geq 0}\right\}.
	\end{align*}
\end{defn}
\begin{defn}\label{defn:span-dim-of-cone}
	Let $\sigma = \cone\{v_1, ...,v_k\}$ be a rational polyhedral cone. The \emph{span} of $\sigma$ is the smallest vector subspace $V$ containing $\sigma$. We have that
	\begin{align*}
		V = \sigma + (-\sigma) = \span\{v_1,...,v_k\} = \span\{\sigma\}.
	\end{align*}
	The \emph{dimension} of $\sigma$ is the dimension of the span of $\sigma$. We say that $\sigma$ is \emph{full-dimensional} if $\dim \sigma = \dim N_\bR = n$. 
\end{defn}
\begin{defn}\label{defn:strictly-convex}
	A rational polyhedral cone is said to be \emph{strictly convex} if it doesn't contain a line, i.e. if it doesn't contain a one dimensional affine subspace of $N_\bR$.

	Unless otherwise specified, by ``cone'' we mean ``strictly convex rational polyhedral cone''.
\end{defn}

\begin{defn}\label{defn:fan}
	A \emph{fan} $\Sigma$ in $N$ is a collection of cones in $N$ such that
	\begin{enumerate}[(i)]
		\item if $\sigma \in \Sigma$ then every face of $\sigma$ belongs to $\Sigma$ 
		\item if $\sigma_1,\sigma_2\in \Sigma$ then $\sigma_1\cap \sigma_2$ is a face of both $\sigma_1$ and $\sigma_2$.
	\end{enumerate}
\end{defn}
\begin{defn}\label{defn:dual-cone}
	Given a cone $\sigma \subseteq N_\bR$, the \emph{dual cone} $\sigma\vee \subseteq M_\bR$ is defined
	\begin{align*}
		\sigma^{\vee} = \left\{m \in M_\bR \midd \langle m,v\rangle \geq 0, ~\forall v \in \sigma\right\}.
	\end{align*}
	The pairing $\langle -,-\rangle: M_\bR \times N_\bR \to \bR$ is simply the evaluation map $\langle m,u\rangle = m(u)$.

	We further define the double dual $(\sigma^{\vee})^{\vee}$ by
	\begin{align*}
		(\sigma^{\vee})^{\vee} = \left\{v \in N_\bR \midd \langle m,v\rangle \geq 0, ~ \forall m \in \sigma^{\vee}\right\} 
	\end{align*}
\end{defn}
The following are fundamental facts regarding $\sigma$ and $\sigma^{\vee}$.
\begin{prop}\label{prop:facts-about-cones-and-duals}
	Let $\sigma$ be a cone in $N$ and $\sigma^{\vee}$ be its dual.
	\begin{enumerate}[(a)]
		\item $\sigma^{\vee}$ is a rational polyhedral cone in $M$ (not necessarily strictly convex)
		\item $(\sigma^{\vee})^{\vee} = \sigma$
		\item $\sigma$ is full-dimensional if and only if $\sigma^{\vee}$ is strictly convex
	\end{enumerate}
\end{prop}
\end{document}
