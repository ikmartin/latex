\documentclass[hidelinks,11pt,dvipsnames]{article}
% xcolor commonly causes option clashes, this fixes that
\PassOptionsToPackage{dvipsnames,table}{xcolor}
\usepackage[tmargin=1in, bmargin=1in, lmargin=0.8in, rmargin=1in]{geometry}

%%%%%%%%%%%%%%%%%%%%%%%%%%%%%%%%%%%%%%%%%%%%%%%%%%%%%%%%%%%%%%%%%%%%
%%% For inkscape-figures
%%% Assumes the following directory structure:
%%% master.tex
%%% figures/
%%%     figure1.pdf_tex
%%%     figure1.svg
%%%     figure1.pdf
%%%%%%%%%%%%%%%%%%%%%%%%%%%%%%%%%%%%%%%%%%%%%%%%%%%%%%%%%%%%%%%%%%%%
%\usepackage{import}
\usepackage{pdfpages}
\usepackage{transparent}

\newcommand{\incfig}[2][1]{%
    \def\svgwidth{#1\columnwidth}
    \import{./figures/}{#2.pdf_tex}
}

\pdfsuppresswarningpagegroup=1

% enable synctex for inverse search, whatever synctex is
\synctex=1
\usepackage{float,macrosabound,homework,theorem-env}
\usepackage{microtype}


% font stuff
\usepackage{sectsty}
\allsectionsfont{\sffamily}
\linespread{1.1}

% bibtex stuff
\usepackage[backend=biber,style=alphabetic,sorting=anyt]{biblatex}
\addbibresource{main.bib}

% colored text shortcuts
\newcommand{\blue}[1]{\color{MidnightBlue}{#1}}
\newcommand{\red}[1]{\textcolor{Mahogany}{#1}}
\newcommand{\green}[1]{\textcolor{ForestGreen}{#1}}


% use mathptmx pkg while using default mathcal font
\DeclareMathAlphabet{\mathcal}{OMS}{cmsy}{m}{n}

% fixes the positioning of subscripts in $$ $$
\renewcommand{\det}{\operatorname{det}}

\usetikzlibrary{positioning, arrows.meta}
\newcommand{\here}[2]{\tikz[remember picture]{\node[inner sep=0](#2){#1}}}

%%%%%%%%%%%%%%%%%%%%%%%%%%%%%%%%%%%%%%%%%%%%%%%%%%%%%%%%%%%%%%%%%%%%%
%%% Entry Counter
%%%%%%%%%%%%%%%%%%%%%%%%%%%%%%%%%%%%%%%%%%%%%%%%%%%%%%%%%%%%%%%%%%%%%
\newcounter{entry-counter}
\newcommand{\entry}[1]
{
	\addtocounter{entry-counter}{1}
    \tchap{Entry \arabic{entry-counter}}
	%\addcontentsline{toc}{section}{Entry \arabic{entry-counter}: #1}
	\vspace{-1.5em}
    \begin{center}
		\small \emph{Written: #1}
    \end{center}
}

\usepackage{titling}
\renewcommand\maketitlehooka{\null\mbox{}\vfill}
\renewcommand\maketitlehookd{\vfill\null}


\begin{document}
\begin{center}
	\Large
	\begin{LARGE}
		Essay Draft 2 \\
	\end{LARGE}
	Isaac Martin \\
    Last compiled \today
\end{center}
\normalsize
\vspace{-2mm}
\hru

\tableofcontents
\newpage
\section*{Introduction}
Write this later.

\newpage
\section{D-Modules over a polynomial ring}

\subsection{Dimension of modules over the Weyl algebra}
The primary goal of this section is a proof of Bernstein's Inequality, a striking example of how the theory of $D$-modules can drastically differ from that of modules over commutative rings. To accomplish this, it is necessary to discuss several basic facts regarding the dimension of modules over the Weyl algebra, proofs of which we brazenly omit in eternal deference to Atiyah-Macdonald.
\subsubsection{Facts about Dimension}
\begin{thm}\label{thm:basic-dim-properties-A_n}
	Let $M$ be a finitely-generated left $A_n$-module and $N \subseteq M$ a submodule. Then
	\begin{enumerate}[(a)]
		\item $\dim(M) = \max\{d(N),d(M/N)\}$ 
		\item If $\dim(N) = \dim(M/N)$ then $m(M) = m(N) + m(M/N)$.
	\end{enumerate}
\end{thm}
\begin{prf}$ $
	\begin{enumerate}[(a)]
		\item Let us first see how the Hilbert polynomials of $M$, $N$ and $M/N$ related. Denote by $S_n$ the associated graded ring of $A_n$, and let $\Gamma$ be a good filtration of $M$ with respect to $\cB$. Let$\Gamma'$ and $\Gamma''$ be the induced filtrations for $N$ and $M/N$. We then obtain the following short exact sequence of associated graded $S_n$-modules:
		\begin{align*}
			0 \to \gr^{\Gamma'}N \to \gr^\Gamma M \to \gr^{\Gamma''}M/N \to 0.
		\end{align*}
		We know $\gr^\Gamma M$ is a finitely generated $S_n$-module since $\Gamma$ is good, hence $\gr^{\Gamma''}M/N$ is also finitely generated since it is isomorphic to a quotient of $\gr^\Gamma M$. Likewise, since $S_n$ is Noetherian and $\gr^{\Gamma'}N$ is isomorphic to a submodule of $\gr^\Gamma M$, $\gr^{\Gamma'}N$ is finitely generated. This tells us that $\Gamma'$ and $\Gamma''$ are both good filtrations.

		Now consider the short exact sequence of vector spaces
		\begin{align*}
			0 \to \Gamma'_k/\Gamma'_{k-1} \to \Gamma_k/\Gamma_{k-1} \to \Gamma''_{k}/\Gamma''_{k-1}\to 0
		\end{align*}
		for $0\leq k$. By the rank-nullity theorem, $\dim_K\Gamma_k/\Gamma_{k-1} = \dim_K\Gamma'_k/\Gamma'_{k-1} + \dim_K\Gamma''_{k}/\Gamma''_{k-1}$, so 
		\begin{align*}
			\sum_{k=0}^\infty\left(\dim_K\Gamma_k/\Gamma_{k-1}\right) = \sum_{k = 0}^\infty\left(\dim_K\Gamma'_k/\Gamma'_{k-1} + \dim_K\Gamma''_{k}/\Gamma''_{k-1}\right)
		\end{align*}
		and thus for $s >> 0$ we get
		\begin{align*}
			\chi(s,\Gamma,M) = \chi(s,\Gamma',M) + \chi(s,\Gamma'', N).
		\end{align*}

		As all of the above are polynomials with positive leading coefficients by \red{CITE THEOREM}, we get that $\deg\left(\chi(s,\Gamma',M) + \chi(s,\Gamma'', N)\right) = \deg\left(\chi(s,\Gamma',M)\right) + \left(\chi(s,\Gamma'', N)\right)$ and hence
		\begin{align*}
			\dim(M) = \max\left\{\dim(N), \dim(M)\right\}.
		\end{align*}
		\item If $\dim(M/N) = \dim(N)$ then the polynomials $\chi(s,\Gamma,M), \chi(s,\Gamma',M)$ and $\chi(s,\Gamma'', N)$ all have the same degree. This then implies that the leading term of $\chi(s,\Gamma, M)$ is equal to the sum of the leading terms of $\chi(s,\Gamma',N)$ and $\chi(s,\Gamma'',M/N)$.
	\end{enumerate}
\end{prf}

\subsubsection{Bernstein's Inequality}

\begin{thm}\label{thm:bern-inequality}
	If $M$ is a finitely-generated left $A_n(K)$-module, then either $n \leq \dim(M)\leq 2n$ or $M = 0$.
\end{thm}
\begin{prf}
	TO DO!!!!!!!!!
\end{prf}

\subsection{Holonomic Modules}

The Bernstein inequality tells us that a nonzero finitely-generated left $A_n(K)$-module $M$ must have dimension at least $n$ and at most $2n$. Those modules of minimal dimension are nice enough that we give them their very own name; we call them \emph{holonomic modules}.
\begin{defn}\label{defn:holonomic-modules}
	A finitely generated left $A_n(K)$-module $M$ is said to be \emph{holonomic} if either $M = 0$ or $\dim(M) = n$.
\end{defn}
Examples are easy to identify thanks to Bernstein. We know that $R = K[x_1,...,x_n]$ is holonomic since $\dim K[x_1,...,x_n] = n$, and furthermore, both $I$ and $R/I$ are holonomic when $I$ is any proper ideal of $R$. As another example, in the case that $n = 1$, for any nonzero ideal $I \subseteq A_1$ we have that $\dim(A_1/I) \leq 1$ by 
\subsubsection{Lemma on B-Functions}
Let $f$ be a polynomial in $K[x_1,...,x_n]$ and let $s$ be a new variable. We will consider the Weyl algebra $A_n(K(s))$ over the field of rational functions in $s$ and the $A_n(K(s))$-module generated by the formal symbol $f^s$, upon which a rational function $p \in K(s)$ acts in the obvious way and the operator $\partial_i$ acts by the formula
\begin{equation}\label{eqn:formula-preceeding-b-func}
	\partial_j\left(f^s\right) = \frac{s}{f}\cdot \frac{\partial f}{\partial x_i}.
\end{equation}
When $s$ is an integer and $f^s$ is treated not as a formal symbol but as a power, this action of course agrees with the existing action of $\partial_j$. The above formula means that $A_n(K(s))p^s$ is an $A_n(K(s))$-submodule of $K(s)[x_1,...,x_n,f^{-1}]$.

\begin{thm}\label{thm:b-functions-Weyl}
	Fix $f \in K[x_1,...,x_n]$. There exists a polynomial $B(s) \in K[s]$ and a differential operator $D(s)\in A_n(K)[s]$ such that
	\begin{align*}
		B(s)f^s = D(s) f^{s+1}.
	\end{align*}
	The set of all such $B(s)$ form an ideal in $K[s]$, the monic generator of which is called the \emph{Bernstein polynomial} of $f$ and is denoted by $b_f(s)$.
\end{thm}
\begin{example}\label{example:explicit-b-function1}
	Let $f = x_1^2+...+x_n^2$. Notice that
	\begin{align*}
		\partial_i^2 f^{s+1} = 4x_i^2(s+1)sf^{s-1} + 2(s+1)f^s.
	\end{align*}
	Letting $D = \partial_1^2 + ... + \partial_n^2$, we get that
	\begin{align*}
		D(f^{s+1})
		  &= \sum_{i=0}^n \left(4x_i^2(s+1)sf^{s-1} ~+~ 2(s+1)f^s\right) \\
		  &= 4(s+1)s(x_1^2+...+x_n^2)f^{s-1} + 2n(s+1)f^s \\
		  &= 2(s+1)(2s+ n)f^s,
	\end{align*}
	hence $b_f(s) = 2(s+1)(2s+n)f^s$.
\end{example}

\subsection{Inverse Image and Direct Image of Modules over the Weyl Algebra}
Here we introduce two operations on $D$-modules, inverse images and direct images. We discuss the former first. As should be expected by now, some algebraic constructions are brushed under the rug, the tensor product of bimodules perhaps chief among them. The reader may wish to visit \cite{d-mod-primer}, \cite{d-mod_ps_rt} or \cite{ginzburg_d-mod} if this is unfamiliar.

\subsubsection{Inverse Images}
Let $R = K[x_1,...,x_n]$, $S = K[y_1,...,y_m]$ and $M$ be a left $A_n$-module. We wish to build a left $A_m$-module from $M$ in a meaningful way. Suppose further that we are given a polynomial map $F:K^m\to K^n$ whose coordinate polynomials are $F_1,...,F_n\in K[y_1,...,y_m]$. Composition with $F$ gives a map of rings $\varphi:R\to S$ defined $f\mapsto f\circ F$, and in particular, we may view $S$ as an $R$-module via the $R$-action given by $\varphi$. The tensor product $S\otimes_R M$ is then an $S$-module denoted $F^*M$, and we want to turn it into an $A_m$-module. The action of polynomials on $F^*M$ is known, and we define the action of $\partial_{y_i}$ by
\begin{equation}\label{eqn:action-inv-img}\tag{$\ast$}
	\partial_{y_i}(q\otimes u) = \partial_{y_i}(q) \otimes u ~ + ~ \sum_{j=1}^n q\partial_{y_i}(F_j) \otimes \partial_{x_j} u.
\end{equation}
for $q\in S$ and $u\in M$ and extend by linearity. As the $y_i$'s and $\partial_{y_i}$'s generate $A_m$, to check that this does indeed give us a left $A_m$-action on $F^*M$ we need only check that (\ref{eqn:action-inv-img}) is compatible with the relations
\begin{align*}
	[\partial_{y_i}, y_j] &= \delta_{ij} \\
	[y_i,y_j] &= [\partial_{y_i},\partial_{y_j}] = 0
\end{align*}
where $\delta_{ij}$ is the Dirac delta function.

Let us check the first relation on an arbitrary generator $q\otimes u \in F^*M$, referring to the image of $y_j \in S$ in $A_m$ by $\hat{y}_j$ to avoid confusion:

% Possible confusion due to the fact that $[\partial_{y_i}, \hat{y}_j] = \delta_{ij}$ in $A_m$ whereas $\partial_{y_i}(y_j) = \delta_{ij}$ in $S$.
\begin{align*}
	\partial_{y_i}\hat{y}_j(q\otimes u)
	  &= \partial_{y_i}(y_j q \otimes u) ~+~ \sum_{k=1}^n y_jq\partial_{y_i}(F_k) \otimes \partial_{x_k}u \\
	  &= \big(\partial_{y_i}(y_j q)\big)\otimes u ~+~ y_j \sum_{k=1}^n q \partial_{y_i}(F_k) \otimes \partial_{x_k}u \\
	  &= q(\partial_{y_i}y_j)\otimes u ~+~ y_j(\partial_{y_i}(q))\otimes u ~+~ y_j \sum_{k=1}^n q \partial_{y_i}(F_k) \otimes \partial_{x_k}u \\
	  &= q\delta_{ij} \otimes u ~+~ y_j\left(\partial_{y_j}(q) \otimes u ~+~ \sum_{k=1}^n q\partial_{y_i}(F_k) \otimes \partial_{x_k}u\right) \\
	  &= \delta_{ij}(q\otimes u) ~+~ y_j\partial_{y_i}(q\otimes u) \\
	  &= \delta_{ij}(q\otimes u) ~+~ \hat{y}_j\big(\partial_{y_i}(q\otimes u)\big).
\end{align*}
Since the relation $[\partial_{y_i},\hat{y}_j]v = \delta_{ij}v$ holds when $v$ is of the form $q\otimes u \in F^*M$, it holds for arbitrary elements $v\in F^*M$ once we extend by linearity. The other two relations follow from an easy check and a few applications of Leibniz's rule, respectively. \red{In reality, I simply don't want to write out the relation $[\partial_{y_i},\partial_{y_j}]v = 0$, but perhaps it would be helpful to include in a future draft.} This means the following definition makes sense:

\begin{defn}\label{defn:inv-img}
	Let $F:K^m\to K^n$ be a polynomial map with components $F_1,...,F_n \in K[y_1,...,y_m]$ and let $M$ be a left $A_n$-module. Then $F^*M$ endowed with the $A_m$-action defined in (\ref{eqn:action-inv-img}) is a left $A_m$-module called the \emph{inverse image} of $M$ by the polynomial map $F$.
\end{defn}

\begin{rmk}\label{rmk:inv-img}
	The origin of the term ``inverse image'' here is clearer if we rephrase things in terms of geometry. If $X$ is a variety and $D_X$ is its sheaf of differential operators, then a sheaf $M$ over $X$ is said to be a left $D_X$-module if $M$ is a sheaf of left modules over $D_X$. Given a morphism $f:X\to Y$ of smooth algebraic varieties and a left $D_Y$-module $M$, then $f^*M$ is simply the sheaf-theoretic inverse image defined in \cite{hartshorne}.
\end{rmk}

\subsubsection{Direct Images}

\newpage
\printbibliography
\end{document}
