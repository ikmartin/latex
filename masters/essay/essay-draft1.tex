\documentclass[hidelinks,11pt,dvipsnames]{article}
% xcolor commonly causes option clashes, this fixes that
\PassOptionsToPackage{dvipsnames,table}{xcolor}
\usepackage[tmargin=1in, bmargin=1in, lmargin=0.8in, rmargin=1in]{geometry}

%%%%%%%%%%%%%%%%%%%%%%%%%%%%%%%%%%%%%%%%%%%%%%%%%%%%%%%%%%%%%%%%%%%%
%%% For inkscape-figures
%%% Assumes the following directory structure:
%%% master.tex
%%% figures/
%%%     figure1.pdf_tex
%%%     figure1.svg
%%%     figure1.pdf
%%%%%%%%%%%%%%%%%%%%%%%%%%%%%%%%%%%%%%%%%%%%%%%%%%%%%%%%%%%%%%%%%%%%
%\usepackage{import}
\usepackage{pdfpages}
\usepackage{transparent}

\newcommand{\incfig}[2][1]{%
    \def\svgwidth{#1\columnwidth}
    \import{./figures/}{#2.pdf_tex}
}

\pdfsuppresswarningpagegroup=1

% enable synctex for inverse search, whatever synctex is
\synctex=1
\usepackage{float,macrosabound,homework,theorem-env}
\usepackage{microtype}


% font stuff
\usepackage{sectsty}
\allsectionsfont{\sffamily}
\linespread{1.1}

% bibtex stuff
\usepackage[backend=biber,style=alphabetic,sorting=anyt]{biblatex}
\addbibresource{main.bib}

% colored text shortcuts
\newcommand{\blue}[1]{\color{MidnightBlue}{#1}}
\newcommand{\red}[1]{\textcolor{Mahogany}{#1}}
\newcommand{\green}[1]{\textcolor{ForestGreen}{#1}}


% use mathptmx pkg while using default mathcal font
\DeclareMathAlphabet{\mathcal}{OMS}{cmsy}{m}{n}

% fixes the positioning of subscripts in $$ $$
\renewcommand{\det}{\operatorname{det}}

\usetikzlibrary{positioning, arrows.meta}
\newcommand{\here}[2]{\tikz[remember picture]{\node[inner sep=0](#2){#1}}}

%%%%%%%%%%%%%%%%%%%%%%%%%%%%%%%%%%%%%%%%%%%%%%%%%%%%%%%%%%%%%%%%%%%%%
%%% Entry Counter
%%%%%%%%%%%%%%%%%%%%%%%%%%%%%%%%%%%%%%%%%%%%%%%%%%%%%%%%%%%%%%%%%%%%%
\newcounter{entry-counter}
\newcommand{\entry}[1]
{
	\addtocounter{entry-counter}{1}
    \tchap{Entry \arabic{entry-counter}}
	%\addcontentsline{toc}{section}{Entry \arabic{entry-counter}: #1}
	\vspace{-1.5em}
    \begin{center}
		\small \emph{Written: #1}
    \end{center}
}

\usepackage{titling}
\renewcommand\maketitlehooka{\null\mbox{}\vfill}
\renewcommand\maketitlehookd{\vfill\null}


\begin{document}
\begin{center}
	\Large
	\begin{LARGE}
		D-Modules, Unit \textit{F}-Crystals, and Hodge Theory \\
	\end{LARGE}
	Definitions, Theorems, Remarks, and Notable Examples \\
	Isaac Martin \\
    Last compiled \today
\end{center}
\normalsize
\vspace{-2mm}
\hru

\tableofcontents
\newpage
\section{Basics}
Here we cover basic definitions and theorems in the theory of $D$-modules with a heavy emphasis on examples.

\subsection{Weyl Algebra}
Let $K$ be a field of characteristic $0$. We construct the Weyl algebra in two ways and prove that these constructions produce isomorphic rings.

\begin{defn}\label{defn:Weyl-algebra-construction-1}
	Let $K$ be a field of characteristic $0$ and let $K[X] = K[x_1,...,x_n] = \Gamma(X,\cO_X)$ be the polynomial ring over $K$ in $n$ variables, and let $X = \bA^n_K = \bA^n$. Consider the algebra of $K$-linear operators $\End_K(K[X])$ and more specifically the operators $\hat{x_i}, \partial_j \in \End_K(K[X])$ for $1\leq, i,j \leq n$. These are defined
	\begin{align*}
		\hat{x_i}: K[X]\to K[X], ~ f \mapsto x_i \cdot f
	\end{align*}
	and 
	\begin{align*}
		\partial_j:K[X]\to K[X], ~ f\mapsto \frac{\partial f}{\partial x_j}.
	\end{align*}
	These are both linear operators, and they satisfy the relation
	\begin{align*}
		[\partial_j, \hat{x_i}] = \partial_j \hat{x_i} - \hat{x_i}\partial_j = \delta_{ij}
	\end{align*}
	where $\delta_{ij} = 1$ if $i = j$ and is otherwise 0. 
\end{defn}
Since $K[\hat{x}] \cong K[x]$ as rings, we typically drop the hat notation and simply write $x_i$ for $\hat{x_i}$. For any two operators $A,B \in \End(R)$ we write  $[A,B] = AB - BA$. The commutator is a $K$-bilinear map on $\End(R)$.

We can also write the Weyl algebra down as a quotient of a free algebra in $2n$ generators over $K$. 

\begin{defn}\label{defn:Weyl-algebra-construction-2}
	The free algebra $K\{x_1,...,x_{2n}\}$ in $2n$ generators is the set of $K$-linear combinations of words in $x_1,...,x_{2n}$. Multiplication is given by concatenation on monomials and then extended to arbitrary elements by the distributive property. We have a homomorphism
	\begin{align*}
		\phi:K\{x_1,...,x_{2n}\} \to A_n
	\end{align*}
	given by $x_i \mapsto x_i$ and $x_{i+n} \mapsto \partial_i$ for $1\leq i\leq n$. Let $J$ be the two-sided ideal of  $K\{x_1,...,x_{2n}\}$ generated by $[x_{i+n},x_i] - 1$ for $1\leq i\leq n$. Each of these generators is mapped to zero in $A_n$ by the relations in Definition (\ref{defn:Weyl-algebra-construction-1}), so $J \subseteq \ker \phi$. We therefore obtain a map $\hat{\phi}:K{x_1,...,x_{2n}}/J\to A_n$ induced by $\phi$.
\end{defn}

\begin{thm}\label{thm:Weyl-constructions-iso}
	The map $\hat{\phi}$ is an isormophism.
\end{thm}

To summarize, in $A_n$,
\begin{itemize}
	\item $x_i$ and $x_j$ commute
	\item $\partial_i$ and $\partial_j$ commute
	\item $[\partial_i,x_j] = \delta_{ij}$, that is, $\partial_i$ and $x_j$ commute unless $i = j$.
\end{itemize}
\begin{example}\label{example:commutator-as-derivative-one-dim}
    Given a polynomial $f \in K[x]$, we can think of $f$ as an operator in $\End_K(K[x])$ by the map $x \mapsto \hat{x}$, and the operator $f$ is simply given by multiplication by $f$.
\end{example}

$ \bA, \bB, \bC, \bD, \bR, \bZ, \bC, \bN$
\end{document}
