\documentclass[hidelinks,11pt,dvipsnames]{article}
% xcolor commonly causes option clashes, this fixes that
\PassOptionsToPackage{dvipsnames,table}{xcolor}
\usepackage[tmargin=1in, bmargin=1in, lmargin=0.8in, rmargin=1in]{geometry}

%%%%%%%%%%%%%%%%%%%%%%%%%%%%%%%%%%%%%%%%%%%%%%%%%%%%%%%%%%%%%%%%%%%%
%%% For inkscape-figures
%%% Assumes the following directory structure:
%%% master.tex
%%% figures/
%%%     figure1.pdf_tex
%%%     figure1.svg
%%%     figure1.pdf
%%%%%%%%%%%%%%%%%%%%%%%%%%%%%%%%%%%%%%%%%%%%%%%%%%%%%%%%%%%%%%%%%%%%
%\usepackage{import}
\usepackage{pdfpages}
\usepackage{transparent}

\newcommand{\incfig}[2][1]{%
    \def\svgwidth{#1\columnwidth}
    \import{./figures/}{#2.pdf_tex}
}

\pdfsuppresswarningpagegroup=1

% enable synctex for inverse search, whatever synctex is
\synctex=1
\usepackage{float,macrosabound,homework,theorem-env}
\usepackage{microtype}


% font stuff
\usepackage{sectsty}
\allsectionsfont{\sffamily}
\linespread{1.1}

% bibtex stuff
\usepackage[backend=biber,style=alphabetic,sorting=anyt]{biblatex}
\addbibresource{main.bib}

% colored text shortcuts
\newcommand{\blue}[1]{\color{MidnightBlue}{#1}}
\newcommand{\red}[1]{\textcolor{Mahogany}{#1}}
\newcommand{\green}[1]{\textcolor{ForestGreen}{#1}}


% use mathptmx pkg while using default mathcal font
\DeclareMathAlphabet{\mathcal}{OMS}{cmsy}{m}{n}

% fixes the positioning of subscripts in $$ $$
\renewcommand{\det}{\operatorname{det}}

\usetikzlibrary{positioning, arrows.meta}
\newcommand{\here}[2]{\tikz[remember picture]{\node[inner sep=0](#2){#1}}}

%%%%%%%%%%%%%%%%%%%%%%%%%%%%%%%%%%%%%%%%%%%%%%%%%%%%%%%%%%%%%%%%%%%%%
%%% Entry Counter
%%%%%%%%%%%%%%%%%%%%%%%%%%%%%%%%%%%%%%%%%%%%%%%%%%%%%%%%%%%%%%%%%%%%%
\newcounter{entry-counter}
\newcommand{\entry}[1]
{
	\addtocounter{entry-counter}{1}
    \tchap{Entry \arabic{entry-counter}}
	%\addcontentsline{toc}{section}{Entry \arabic{entry-counter}: #1}
	\vspace{-1.5em}
    \begin{center}
		\small \emph{Written: #1}
    \end{center}
}

\usepackage{titling}
\renewcommand\maketitlehooka{\null\mbox{}\vfill}
\renewcommand\maketitlehookd{\vfill\null}


\begin{document}
\begin{center}
	\Large
	\begin{LARGE}
		D-Modules Draft 1 \\
	\end{LARGE}
	Definitions, Theorems, Remarks, and Notable Examples \\
	Isaac Martin \\
    Last compiled \today
\end{center}
\normalsize
\vspace{-2mm}
\hru

\tableofcontents
\newpage
\section{D-modules over Affine n-space}
Here we cover the basic theory of modules over the Weyl algebra, or in other words, the theory of $D$-modules in the case where $X = \bA^n_K$.

\subsection{Weyl Algebra}
Let $K$ be a field of characteristic $0$. We construct the Weyl algebra in two ways and prove that these constructions produce isomorphic rings.

\begin{defn}\label{defn:Weyl-algebra-construction-1}
	Let $K$ be a field of characteristic $0$ and let $K[X] = K[x_1,...,x_n] = \Gamma(X,\cO_X)$ be the polynomial ring over $K$ in $n$ variables, and let $X = \bA^n_K = \bA^n$. Consider the algebra of $K$-linear operators $\End_K(K[X])$ and more specifically the operators $\hat{x_i}, \partial_j \in \End_K(K[X])$ for $1\leq, i,j \leq n$. These are defined
	\begin{align*}
		\hat{x_i}: K[X]\to K[X], ~ f \mapsto x_i \cdot f
	\end{align*}
	and 
	\begin{align*}
		\partial_j:K[X]\to K[X], ~ f\mapsto \frac{\partial f}{\partial x_j}.
	\end{align*}
	These are both linear operators, and they satisfy the relation
	\begin{align*}
		[\partial_j, \hat{x_i}] = \partial_j \hat{x_i} - \hat{x_i}\partial_j = \delta_{ij}
	\end{align*}
	where $\delta_{ij} = 1$ if $i = j$ and is otherwise 0. 
\end{defn}
Since $K[\hat{x}] \cong K[x]$ as rings, we typically drop the hat notation and simply write $x_i$ for $\hat{x_i}$. For any two operators $A,B \in \End(R)$ we write  $[A,B] = AB - BA$. The commutator is a $K$-bilinear map on $\End(R)$.

We can also write the Weyl algebra down as a quotient of a free algebra in $2n$ generators over $K$.

\begin{defn}\label{defn:Weyl-algebra-construction-2}
	The free algebra $K\{x_1,...,x_{2n}\}$ in $2n$ generators is the set of $K$-linear combinations of words in $x_1,...,x_{2n}$. Multiplication is given by concatenation on monomials and then extended to arbitrary elements by the distributive property. We have a homomorphism
	\begin{align*}
		\phi:K\{x_1,...,x_{2n}\} \to A_n
	\end{align*}
	given by $x_i \mapsto x_i$ and $x_{i+n} \mapsto \partial_i$ for $1\leq i\leq n$. Let $J$ be the two-sided ideal of  $K\{x_1,...,x_{2n}\}$ generated by $[x_{i+n},x_i] - 1$ for $1\leq i\leq n$. Each of these generators is mapped to zero in $A_n$ by the relations in Definition (\ref{defn:Weyl-algebra-construction-1}), so $J \subseteq \ker \phi$. We therefore obtain a map $\hat{\phi}:K{x_1,...,x_{2n}}/J\to A_n$ induced by $\phi$.
\end{defn}

\begin{thm}\label{thm:Weyl-constructions-iso}
	The map $\hat{\phi}$ is an isormophism.
\end{thm}

\noindent To summarize, in $A_n$,
\begin{itemize}
	\item $x_i$ and $x_j$ commute
	\item $\partial_i$ and $\partial_j$ commute
	\item $[\partial_i,x_j] = \delta_{ij}$, that is, $\partial_i$ and $x_j$ commute unless $i = j$.
\end{itemize}
\begin{example}\label{example:commutator-as-derivative-one-dim}
    Given a polynomial $f \in K[x]$, we can think of $f$ as an operator in $\End_K(K[x])$ by the map $x \mapsto \hat{x}$, and the operator $f$ is simply given by multiplication by $f$. I claim that the commutator of $f$ with $\partial$ satisfies the following relation: $[\partial, f] = f'$ where $f'$ is the derivative of $f$. To see this, it suffices to show that $[\partial, x^n] = nx^{n-1}$ for $n \in \bZ_{\geq 0}$, since $[-,-]$ is $K$-bilinear.

	We show this by induction. The commutator relation $[\partial,x] = 1$ serves as the base case, so suppose $[\partial, x^k] = kx^{k-1}$ for $1\leq k < n$. Then
	\begin{align*}
		\partial x^n
		&= (\partial x)x^{n-1} \\
		&= (1 - x\partial)x^{n-1} \\
		&= x^{n-1} - x\partial x^{n-1} \\
		&= x^{n-1} - x\cdot (n-1) x^{n-2} = n\cdot x^{n-1},
	\end{align*}
	giving us the result.
\end{example}
It is useful to fix a basis for the Weyl algebra, but for arbitrary $n$, the notation becomes cumbersome. To remedy this, we use multi-indices. For $\alpha = (\alpha_1,...,\alpha_n), \beta = (\beta_1,...,\beta_n)\in \bN^n$, we denote by $x^\alpha\partial^\beta$ the element $x_1^{\alpha_1}...x_n^{\alpha_n}\partial_1^{\beta_1}...\partial_n^{\beta_n} \in A_n$. As it turns out, the set of all elements of this form is a $K$-basis for $A_n$
\begin{prop}\label{prop:canonical-basis}
	The set $\bfB = \{x^\alpha\partial^\beta ~ \mid ~ \alpha,\beta\in \bN^n\}$ is a basis of $A_n$ as a vector space over $K$. This is called the \emph{canonical basis} of $A_n$ and an operator $D \in A_n$ written as a linear combination of elements in $\bfB$ is said to be in \emph{canonical form}.
\end{prop}

\subsection{Basic Properties of the Weyl Algebra}
Despite the noncommutative of $A_n$, one might be tempted to draw comparisons between the Weyl algebra and a ring of polynomials, especially given that $A_n$ admits such a nice basis. In particular, one might wonder if $A_n$ admits any meaningful graded structure. The answer turns out to be ``sort of''. The goal of this section is primarily to define and examine this approximation of a graded structure. To do this, we first define the \emph{degree} of an element in $A_n$. We then notice that this fails to define a $K[X]$-grading for $A_n$ and provide a workaround.

We also say some words about the ideal structure of $A_n$ and the case in which $\fchar K = p > 0$.

\begin{defn}\label{defn:degree-of-Weyl-operator}
	The \emph{length} of a multindex $\alpha \in \bN^n$ is denoted $|\alpha|$ and is defined
	\begin{align*}
		|\alpha| = |\alpha_1| + ... + |\alpha_n|.
	\end{align*}
	Let $D$ be an operator $A_n$. The \emph{degree} of $D$, $\deg(D)$, is the largest length of the multi-indices $(\alpha, \beta) \in \bN^n \times \bN^n$ for which $x^\alpha\partial^\beta$ appears with non-zero coefficient in the canonical form of $D$. We define $\deg(0) = -\infty$.
\end{defn}
It is important to remember that the definition of degree depends on the canonical basis for $A_n$. The degree of $x\partial$ can simply be read off as 2, but $\partial x$ must first be written $x\partial + 1$ in order to see that $\deg(\partial x) = 2$.

The function $\deg: A_n\to \bN$ reproduces the multiplicative and additive structure of $\deg:K[X]\to \bN$:
\begin{thm}\label{thm:degree-properties}
    Let $D,D' \in A_n$.
	\begin{enumerate}[(1)]
		\item $\deg(DD') = \deg(D) + \deg(D')$
		\item $\deg(D+D') \leq \max\{\deg(D), \deg(D')\}$
		\item $\deg[D,D']\leq \deg(D) + \deg(D') - 2$.
	\end{enumerate}
\end{thm}
Note that equality holds in (2) when $\deg(D) \neq \deg(D')$ but that there is risk of cancellation otherwise, as is the case with rings of polynomials.
\begin{prf}
	We refer to \cite{Gieseker75} for the proof of (1) and (3) \red{COMPLETE PROOF OF (3) MANUALLY}.
\end{prf}

One might expect a graded structure to naturally fall from this definition of degree. The issue, as always, is one of noncommutativity. The element $x_1\partial_1$ ought to homogeneous, but it is the difference $\partial x - 1$ of two elements with non-equal degree. There is no way to define a collection of pairwise disjoint $K[X]$-submodules of $A_n$ whose direct sum recovers $A_n$. Nonetheless, we can still find a collection of $A_n$ submodules which resemble a grading on $A_n$. We will call this a \emph{filtration} of $A_n$, and it turns out that this filtration will come with a natural associated graded $K[X]$-module whose properties will yield new information about $A_n$.

We first define arbitrary filtered rings and modules.

\begin{defn}\label{defn:filtered-ring}
	Let $R$ be a $K$-algebra and $M$ an $R$-module. A collection $\cF = \{F_i\}_{i \geq 0}$ of $K$-vector spaces is said to be a \emph{filtration} of $R$ if
	\begin{enumerate}[(i)]
		\item $F_0\subset F_1 \subset F_2 \subset ... \subset R$
		\item $R = \bigcup_{i\geq 0} F_i$
		\item $F_i\cdot F_j \subseteq F_{i+j}$.
	\end{enumerate}
	If $R$ has a filtration it is called a \emph{filtered algebra}. Similarly, if $R$ is a filtered algebra, then a \emph{filtration of $M$} is a family $\Gamma = \{\Gamma_0\}_{i\geq 0}$ of $K$-vector spaces satisfying
	\begin{enumerate}[(i)]
		\item $\Gamma_0 \subseteq \Gamma_1 \subseteq \Gamma_2 \subseteq ... \subseteq M$,
		\item $M = \bigcup_{i\geq 0} \Gamma_i$
		\item $B_i\Gamma_j \subseteq \Gamma_{i+j}$.
	\end{enumerate}
\end{defn}
	Such a module is said to be \emph{filtered}. In this section, we additionally adopt the convention that
	\begin{enumerate}
		\item[(4)] $\Gamma_i$ is a finite-dimensional $K$-algebra for each $i \geq 0$,
	\end{enumerate}
	which will become important in our discussion of dimension.

\newpage
\printbibliography
\end{document}
