\documentclass[hidelinks,11pt,dvipsnames]{article}
% xcolor commonly causes option clashes, this fixes that
\PassOptionsToPackage{dvipsnames,table}{xcolor}
\usepackage[tmargin=1in, bmargin=1in, lmargin=0.8in, rmargin=1in]{geometry}

%%%%%%%%%%%%%%%%%%%%%%%%%%%%%%%%%%%%%%%%%%%%%%%%%%%%%%%%%%%%%%%%%%%%
%%% For inkscape-figures
%%% Assumes the following directory structure:
%%% master.tex
%%% figures/
%%%     figure1.pdf_tex
%%%     figure1.svg
%%%     figure1.pdf
%%%%%%%%%%%%%%%%%%%%%%%%%%%%%%%%%%%%%%%%%%%%%%%%%%%%%%%%%%%%%%%%%%%%
%\usepackage{import}
\usepackage{pdfpages}
\usepackage{transparent}

\newcommand{\incfig}[2][1]{%
    \def\svgwidth{#1\columnwidth}
    \import{./figures/}{#2.pdf_tex}
}

\pdfsuppresswarningpagegroup=1

% enable synctex for inverse search, whatever synctex is
\synctex=1
\usepackage{float,macrosabound,homework,theorem-env}
\usepackage{microtype}


% font stuff
\usepackage{sectsty}
\allsectionsfont{\sffamily}
\linespread{1.1}

% bibtex stuff
\usepackage[backend=biber,style=alphabetic,sorting=anyt]{biblatex}
\addbibresource{main.bib}

% colored text shortcuts
\newcommand{\blue}[1]{\color{MidnightBlue}{#1}}
\newcommand{\red}[1]{\textcolor{Mahogany}{#1}}
\newcommand{\green}[1]{\textcolor{ForestGreen}{#1}}


% use mathptmx pkg while using default mathcal font
\DeclareMathAlphabet{\mathcal}{OMS}{cmsy}{m}{n}

% fixes the positioning of subscripts in $$ $$
\renewcommand{\det}{\operatorname{det}}

\usetikzlibrary{positioning, arrows.meta}
\newcommand{\here}[2]{\tikz[remember picture]{\node[inner sep=0](#2){#1}}}

%%%%%%%%%%%%%%%%%%%%%%%%%%%%%%%%%%%%%%%%%%%%%%%%%%%%%%%%%%%%%%%%%%%%%
%%% Entry Counter
%%%%%%%%%%%%%%%%%%%%%%%%%%%%%%%%%%%%%%%%%%%%%%%%%%%%%%%%%%%%%%%%%%%%%
\newcounter{entry-counter}
\newcommand{\entry}[1]
{
	\addtocounter{entry-counter}{1}
    \tchap{Entry \arabic{entry-counter}}
	%\addcontentsline{toc}{section}{Entry \arabic{entry-counter}: #1}
	\vspace{-1.5em}
    \begin{center}
		\small \emph{Written: #1}
    \end{center}
}

\usepackage{titling}
\renewcommand\maketitlehooka{\null\mbox{}\vfill}
\renewcommand\maketitlehookd{\vfill\null}


\begin{document}
\begin{center}
	\Large
	\begin{LARGE}
		D-Modules Draft 1 \\
	\end{LARGE}
	Definitions, Theorems, Remarks, and Notable Examples \\
	Isaac Martin \\
    Last compiled \today
\end{center}
\normalsize
\vspace{-2mm}
\hru

\tableofcontents
\newpage
\noindent \textbf{Notes/Thoughts}
\begin{enumerate}
	\item Lots of numbered definitions, all of which are standard in the literature. Would it be a better stylistic choice to simply integrate these into the text and drop the numbering?
	\item Current algorithm for deciding whether a piece of mathematics is included in this document is "is Isaac comfortable with this content? If yes, don't include it. If no, include it." This is good for a document intended only for the author, but is bad for consistency and a standardized assumed background. Why should a discussion of Noetherian rings and modules be omitted but basic filtration definitions be given an entire subsection? For the second full draft, consider scrapping the more elementary pieces of the theory.
	\item At the time of this writing, induced filtrations only appear in the proof of Theorem (\ref{thm:noeth-assoc-graded-module})
	\item Likely a good idea to move Definition (\ref{defn:diff-op-rings}) to a section about $D_{R/A}$ in general. Has nothing to do with the characteristic of a field.
\end{enumerate}
\newpage

\section{Left Modules Over the Weyl Algebra}
Here we cover the basic theory of modules over the Weyl algebra, or in other words, the theory of $D$-modules in the case where $X = \bA^n_K$. This section is essentially an abbreviation of the book \emph{A Primer on Algebraic $D$-modules} (\cite{d-mod-primer}) with hints of more general theory included. We treat the generalized affine case in later sections.

\subsection{Weyl Algebra}\label{subsec:Weyl-algebra}
Let $K$ be a field of characteristic $0$. We construct the Weyl algebra in two ways and prove that these constructions produce isomorphic rings.

\begin{defn}\label{defn:Weyl-algebra-construction-1}
	Let $K$ be a field of characteristic $0$ and let $K[X] = K[x_1,...,x_n] = \Gamma(X,\cO_X)$ be the polynomial ring over $K$ in $n$ variables, and let $X = \bA^n_K = \bA^n$. Consider the algebra of $K$-linear operators $\End_K(K[X])$ and more specifically the operators $\hat{x_i}, \partial_j \in \End_K(K[X])$ for $1\leq, i,j \leq n$. These are defined
	\begin{align*}
		\hat{x_i}: K[X]\to K[X], ~ f \mapsto x_i \cdot f
	\end{align*}
	and 
	\begin{align*}
		\partial_j:K[X]\to K[X], ~ f\mapsto \frac{\partial f}{\partial x_j}.
	\end{align*}
	These are both linear operators, and they satisfy the relation
	\begin{align*}
		[\partial_j, \hat{x_i}] = \partial_j \hat{x_i} - \hat{x_i}\partial_j = \delta_{ij}
	\end{align*}
	where $\delta_{ij} = 1$ if $i = j$ and is otherwise 0. 
\end{defn}
Since $K[\hat{x}] \cong K[x]$ as rings, we typically drop the hat notation and simply write $x_i$ for $\hat{x_i}$. For any two operators $A,B \in \End(R)$ we write  $[A,B] = AB - BA$. The commutator is a $K$-bilinear map on $\End(R)$.

We can also write the Weyl algebra down as a quotient of a free algebra in $2n$ generators over $K$.

\begin{defn}\label{defn:Weyl-algebra-construction-2}
	The free algebra $K\{x_1,...,x_{2n}\}$ in $2n$ generators is the set of $K$-linear combinations of words in $x_1,...,x_{2n}$. Multiplication is given by concatenation on monomials and then extended to arbitrary elements by the distributive property. We have a homomorphism
	\begin{align*}
		\phi:K\{x_1,...,x_{2n}\} \to A_n
	\end{align*}
	given by $x_i \mapsto x_i$ and $x_{i+n} \mapsto \partial_i$ for $1\leq i\leq n$. Let $J$ be the two-sided ideal of  $K\{x_1,...,x_{2n}\}$ generated by $[x_{i+n},x_i] - 1$ for $1\leq i\leq n$. Each of these generators is mapped to zero in $A_n$ by the relations in Definition (\ref{defn:Weyl-algebra-construction-1}), so $J \subseteq \ker \phi$. We therefore obtain a map $\hat{\phi}:K{x_1,...,x_{2n}}/J\to A_n$ induced by $\phi$.
\end{defn}

\begin{thm}\label{thm:Weyl-constructions-iso}
	The map $\hat{\phi}$ is an isormophism.
\end{thm}

\noindent To summarize, in $A_n$,
\begin{itemize}
	\item $x_i$ and $x_j$ commute
	\item $\partial_i$ and $\partial_j$ commute
	\item $[\partial_i,x_j] = \delta_{ij}$, that is, $\partial_i$ and $x_j$ commute unless $i = j$.
\end{itemize}
\begin{example}\label{ex:commutator-as-derivative-one-dim}
    Given a polynomial $f \in K[x]$, we can think of $f$ as an operator in $\End_K(K[x])$ by the map $x \mapsto \hat{x}$, and the operator $f$ is simply given by multiplication by $f$. I claim that the commutator of $f$ with $\partial$ satisfies the following relation: $[\partial, f] = f'$ where $f'$ is the derivative of $f$. To see this, it suffices to show that $[\partial, x^n] = nx^{n-1}$ for $n \in \bZ_{\geq 0}$, since $[-,-]$ is $K$-bilinear.

	We show this by induction. The commutator relation $[\partial,x] = 1$ serves as the base case, so suppose $[\partial, x^k] = kx^{k-1}$ for $1\leq k < n$. Then
	\begin{align*}
		\partial x^n
		&= (\partial x)x^{n-1} \\
		&= (1 + x\partial)x^{n-1} \\
		&= x^{n-1} + x\partial x^{n-1} \\
		&= x^{n-1} + x\big[(n-1)x^{n-2}+x^{n-1}\partial\big] \\
		&= nx^{n-1} + x^n\partial,
	\end{align*}
	giving us the result.
\end{example}
It is useful to fix a basis for the Weyl algebra, but for arbitrary $n$, the notation becomes cumbersome. To remedy this, we use multi-indices. For $\alpha = (\alpha_1,...,\alpha_n), \beta = (\beta_1,...,\beta_n)\in \bN^n$, we denote by $x^\alpha\partial^\beta$ the element $x_1^{\alpha_1}...x_n^{\alpha_n}\partial_1^{\beta_1}...\partial_n^{\beta_n} \in A_n$. As it turns out, the set of all elements of this form is a $K$-basis for $A_n$
\begin{prop}\label{prop:canonical-basis}
	The set $\bfB = \{x^\alpha\partial^\beta ~ \mid ~ \alpha,\beta\in \bN^n\}$ is a basis of $A_n$ as a vector space over $K$. This is called the \emph{canonical basis} of $A_n$ and an operator $D \in A_n$ written as a linear combination of elements in $\bfB$ is said to be in \emph{canonical form}.
\end{prop}
The niceness of the Weyl algebra can be largely contributed to this canonical basis. Later, we replace $A_n$ with the ring of differential operators over an arbitrary ring $R$, and this object is far more nebulous when $R$ is not a polynomial ring.

\begin{defn}\label{defn:degree-of-Weyl-operator}
	The \emph{length} of a multindex $\alpha \in \bN^n$ is denoted $|\alpha|$ and is defined
	\begin{align*}
		|\alpha| = |\alpha_1| + ... + |\alpha_n|.
	\end{align*}
	Let $D$ be an operator $A_n$. The \emph{degree} of $D$, $\deg(D)$, is the largest length of the multi-indices $(\alpha, \beta) \in \bN^n \times \bN^n$ for which $x^\alpha\partial^\beta$ appears with non-zero coefficient in the canonical form of $D$. We define $\deg(0) = -\infty$.
\end{defn}
It is important to remember that the definition of degree depends on the canonical basis for $A_n$. The degree of $x\partial$ can simply be read off as 2, but $\partial x$ must first be written $x\partial + 1$ in order to see that $\deg(\partial x) = 2$.

The function $\deg: A_n\to \bN$ reproduces the multiplicative and additive structure of $\deg:K[X]\to \bN$:
\begin{thm}[{\cite[Theorem 2.1.1.]{d-mod-primer}}]\label{thm:degree-properties}
    Let $D,D' \in A_n$.
	\begin{enumerate}[(1)]
		\item $\deg(DD') = \deg(D) + \deg(D')$
		\item $\deg(D+D') \leq \max\{\deg(D), \deg(D')\}$
		\item $\deg[D,D']\leq \deg(D) + \deg(D') - 2$.
	\end{enumerate}
\end{thm}
Note that equality holds in (2) when $\deg(D) \neq \deg(D')$ but that there is risk of cancellation otherwise, as is the case with rings of polynomials.
\begin{prf}
	We refer to \cite{d-mod-primer} for the proof of (1) and (3) \red{COMPLETE PROOF OF (2) MANUALLY}.
\end{prf}

As an immediate corollary of Theorem (\ref{thm:degree-properties} (2)), we have that $A_n$ is a domain.

\begin{cor}
	The ring $A_n$ is an integral domain.
\end{cor}
\begin{prf}
	If $D\cdot D' = 0$ in $A_n$, then either $\deg(D) = -\infty$ or $\deg(D') = -\infty$, hence either $D$ or $D'$ is $0$.
\end{prf}

A slightly less trivial fact whose proof relies on the notion of degree is that the Weyl algebra is simple, that is, it possesses no nontrivial proper two-sided ideals.
\begin{thm}\label{thm:Weyl-algebra-simple}
	The algebra $A_n$ is simple.
\end{thm}
\begin{prf}
	Let $I$ be a nonzero two-sided ideal of $A_n$ and suppose $D \in I$ is a nonzero operator. If $deg(D) = 0$, then $D \in K$ and $I = A_n$. If $\deg(D) = d > 0$, then there must be some summand $x^\alpha\partial^\beta$ with nonzero coefficient and for which either $\alpha \neq 0$ or $\alpha \neq 0$. In the former case, suppose the $\alpha_i$ component of $\alpha$ is nonzero. Then $[\partial_i,D] \neq 0$ and $\deg([\partial_i,D]) \leq d - 1$. Furthermore, since $I$ is two-sided, $[\partial_i,D] \in I$. By replacing $D$ with $[\partial_i,D]$ and repeating the above process, we can construct an element of degree 0 in $I$ and hence conclude$I = A_n$. A similar argument in which we instead consider $[x_i,D]$ works in the case that $\beta \neq 0$.
\end{prf}

Despite having no nontrivial proper two-sided ideals, $A_n$ is far from being a division ring. Another application of Theorem (\ref{thm:degree-properties} (2)) shows that the only elements with either left or right inverses are the constant operators $D \in K$.

Even at this early stage, we can see features of this theory which break when $\fchar K = p > 0$. Let us look at what happens when $K = \bF_p$ for $p$ a prime and $n = 1$. Take $A_1 = K[x,\partial]\subseteq \End_K(K[x])$, i.e. let us first use Definition (\ref{defn:Weyl-algebra-construction-1}). Let $k$ be any positive integer and consider the action $\partial^p$ on $x^k \in K[x]$. If $k < p$, then $\partial^p(x^k) = 0$. If $k \geq p$, then at least one of the integers $k-p+1,k-p+2,...,k-1,k$ is divisible by $p$, and hence
\begin{align*}
	\partial^p(x^k) = k(k-1)(k-2)...(k-p+1)x^{k-p} = 0
\end{align*}
by Example (\ref{ex:commutator-as-derivative-one-dim})Since $\partial^p$ is zero on a basis for $K[x]$, it is identically zero on all of $K[x]$. This means $\partial$ is a nilpotent element, and in particular that $A_1$ is not a domain.

Now take $A_1$ to be the free algebra in $x$ and $\partial$ over $K$ modulo the relation $[\partial, x] = 1$, i.e. use Definition (\ref{defn:Weyl-algebra-construction-2}). This ring is domain since $\partial^p = \textbf{some product of $[\partial,x] - 1$}$, and in particular, the two definitions do not agree in characteristic $p$. Nonetheless, this formulation of $A_1$ is disparate from the characteristic 0 case in another respect: it is not a simple ring. For example,
\begin{align*}
	[\partial, x^p] = px^{p- 1} = 0,
\end{align*}
from which it follows that $x^p$ commutes with every element of $A_1$. Thus, $(x^p)$ is a two-sided ideal of $A_1$.

So it is that, in prime characteristic, not only are our two definitions for the Weyl algebra no longer equivalent, but each breaks some elementary property of the Weyl algebra in characteristic 0. It is therefore unclear which definition we should choose.

\subsection{Basic Properties of the Weyl Algebra}

Despite the noncommutative of $A_n$, one might be tempted to draw comparisons between the Weyl algebra and a ring of polynomials, especially given that $A_n$ admits a nice basis and possesses some notion of degree. In particular, one might wonder if $A_n$ admits any meaningful graded structure. The answer turns out to be ``sort of''. The goal of this section is primarily to define and examine this approximation of a graded structure. To do this, we first define the \emph{degree} of an element in $A_n$. We then notice that this fails to define a $K[X]$-grading for $A_n$ and provide a workaround.

We also say some words about the ideal structure of $A_n$ and holonomic modules.

One might expect a graded structure to naturally fall from the notion of degree given in section . The issue, as always, is one of noncommutativity. The element $x_1\partial_1$ ought to homogeneous, but it is the difference $\partial x - 1$ of two elements with non-equal degree. There is no way to define a collection of pairwise disjoint $K[X]$-submodules of $A_n$ whose direct sum recovers $A_n$. Nonetheless, we can still find a collection of $A_n$ submodules which resemble a grading on $A_n$. We will call this a \emph{filtration} of $A_n$, and it turns out that this filtration will come with a natural associated graded $K[X]$-module whose properties will yield new information about $A_n$.

We first define arbitrary filtered rings and modules.

\begin{defn}\label{defn:filtered-ring}
	Let $R$ be a $K$-algebra and $M$ an $R$-module. A collection $\cF = \{F_i\}_{i \geq 0}$ of $K$-vector spaces is said to be a \emph{filtration} of $R$ if
	\begin{enumerate}[(1)]
		\item $F_0\subset F_1 \subset F_2 \subset ... \subset R$
		\item $R = \bigcup_{i\geq 0} F_i$
		\item $F_i\cdot F_j \subseteq F_{i+j}$.
	\end{enumerate}
	If $R$ has a filtration it is called a \emph{filtered algebra}. Similarly, if $R$ is a filtered algebra, then a \emph{filtration of $M$} is a family $\Gamma = \{\Gamma_0\}_{i\geq 0}$ of $K$-vector spaces satisfying
	\begin{enumerate}[(1)]
		\item $\Gamma_0 \subseteq \Gamma_1 \subseteq \Gamma_2 \subseteq ... \subseteq M$,
		\item $M = \bigcup_{i\geq 0} \Gamma_i$
		\item $B_i\Gamma_j \subseteq \Gamma_{i+j}$.
	\end{enumerate}
	It is useful to set $F_i = \Gamma_i = 0$ for $i < 0$.
\end{defn}
Such a module is said to be \emph{filtered}. In this section, we additionally adopt the convention that
\begin{enumerate}
	\item[(4)] $\Gamma_i$ is a finite-dimensional $K$-algebra for each $i \geq 0$,
\end{enumerate}
which will become important in our discussion of dimension.

As promised, filtered rings and modules come with an associated graded structure. Let's first examine this for rings.

\begin{defn}\label{defn:associated-grading-rng}
	Let $R$ be a $K$-algebra and $\cF = \{F_i\}$ a filtration of $R$. The \emph{graded algebra of $R$ associated to $\cF$} is the $K$-vector space
	\begin{align*}
		\gr^\cF R = \bigoplus_{i\geq 0}(F_i/F_{i-1}).
	\end{align*}
	For each $k \geq 0$, the \emph{symbol map of order $k$} is the canonical vector space projection
	\begin{align*}
		\sigma_k:F_k\to F_k/F_{k-1}.
	\end{align*}
	We use these maps to define an algebra structure on $\gr^\cF R$. A \emph{homogeneous element} of $\gr^\cF R$ is any operator $d \in \gr^\cF R$ such that $d = \sigma_k(a)$ for some $a \in F_k$. Given two homogenous elements $\sigma_i(a)$ and $\sigma_j(b)$, we define their product by
	\begin{align*}
		\sigma_i(a)\cdot \sigma_j(b) = \sigma_{i+j}(a\cdot b).
	\end{align*}
	Extending this multiplication to all of $\gr^\cF R$ by distributivity makes $\gr^\cF R$ into a graded $K$-algebra whose homogeneous components are the individual summands $F_k/F_{k-1}$.
\end{defn}
This definition takes as input a filtration of a ring, and in particular we shouldn't expect two different filtrations on $R$ to produce isomorphic associated graded algebras.

We now return our attention to the Weyl algebra. The first and most important filtration of $A_n$ we consider is also the most obvious: we filter using degree. This is called the Bernstein filtration.
\begin{defn}\label{defn:bernstein-filtration}
	Let $B_k$ denote the $K$-subspace of $A_n$ consisting of all operators with degree less than $k$:
	\begin{align*}
		B_k = \left\{D \in A_n \midd \deg(D) \leq k\right\}.
	\end{align*}
	The \emph{Bernstein filtration} is $\cB = \{B_k\}_{k\geq 0}$.
\end{defn}
\begin{prop}\label{prop:bernstein-filtration}
	The Bernstein filtration is indeed a filtration of $A_n$ as a $K$-algebra.
\end{prop}
\begin{prf}
	\red{This promptly follows from Theorem (\ref{thm:degree-properties}), consider omitting.} That $B_k \subseteq B_{k+1}$ and $B_k$ is a $K$-vector subspace of $A_n$ are clear by the definition of $B_k$ and Theorem (\ref{thm:degree-properties}). Theorem (\ref{thm:degree-properties}) part (1) futher tells us that $B_k \cdot B_m \subseteq B_{k\dot m}$. Finally, since every operator $D \in A_n$ can be written in canonical form, $\deg(D) < \infty$ and $D \in B_k$ for some $K$, hence $A_n = \cup_{i\geq 0} B_k$.
\end{prf}
We now compute the graded algebra of $A_n$ associated to the Bernstein filtration.
\begin{thm}\label{thm:graded-algebra-of-Weyl-algebra}
	Let $S_n = \gr^\cB A_n$. The graded algebra $S_n$ is isomorphic to $K[y_1,...,y_{2n}]$.
\end{thm}
\begin{prf}
	See \cite[pg. 58]{d-mod-primer}.
\end{prf}
Although we only refer to the proof of this statement, it is nonetheless worth thinking about why this ought to be true. Since we have surjective maps $\pi_k:A_n\to B_k \to{\sigma_k} B_k/B_{k-1}$, $S_n$ is generated as an algebra by the images of elements $x_1,...,x_n,\partial_1,...,\partial_n \in A_n$. The only thing preventing us from defining a isomorphism $K[y_1,...,y_{2n}]\to S_n$ sending $y_i\mapsto x_i$ and $y_{i+n}\mapsto \partial_{i}$ for $1\leq i\leq n$ is commutativity, however, we see that
\begin{align*}
	\pi_1(\partial_i x_i) = \pi_1(x_i\partial_i + 1) = \pi_1(x_i\partial_i) + \pi_1(1) = \pi_1(x_i\partial_i),
\end{align*}
so $[\partial_i,x_i] = 0$ in $B_1/B_0$. This gives us commutativity in $S_n$ and allows us to define a surjective homomorphism $K[y_1,...,y_{2n}] \to S_n$. Since there are no additional relations between the generators $x_1,...,x_n,\partial_1,...,\partial_n$, this is an isomorphism. Coutinho's proof simply fills in the details of this sketch.

\bigskip

It also makes sense to associate a graded module to a filtered module. We do that now.
\begin{defn}\label{defn:associated-graded-module}
	Let $R$ be a filtered $K$-algebra with filtration $\cF = \{F_i\}$ and $M$ be a left $R$-module with filtration $\Gamma = \{\Gamma_i\}$. The \emph{symbol map of order $k$} of the filtration $\Gamma$ is the projection
	\begin{align*}
		\mu_k:\Gamma_k \to \Gamma_k/\Gamma_{k-1}.
	\end{align*}
	We set
	\begin{align*}
		\gr^\Gamma M = \bigoplus_{i\geq 0} \Gamma_i/\Gamma_{i-1},
	\end{align*}
	and define the $\gr^\cF R$ action on $\gr^\Gamma M$ by
	\begin{align*}
		\sigma_i(a)\cdot \mu_j(m) = \mu_{i+j}(a m)
	\end{align*}
	for $a \in F_i$ and $m \in \Gamma_j$ where $\sigma_i$ is the symbol map of order $i$ of $\cF$.
\end{defn}
The associated grading can tell us something about its filtered module.
\begin{thm}\label{thm:noeth-assoc-graded-module}
	Suppose that $R$ is a Noetherian $K$-algebra with filtration $\cF$ and $M$ is a left $R$-module with filtration $\Gamma = \{\Gamma_i\}_{i\geq 0}$. If $\gr^\Gamma M$ is a Noetherian then so is $M$.
\end{thm}
\begin{prf}
	Set $S = \gr^\cF R$ and let $N \subseteq M$ be a $R$-submodule of $M$. We prove that it is finitely generated. Define $\Gamma'_i = N \cap \Gamma_i$ for $i \geq 0$. The collection $\Gamma' = \{\Gamma'_i\}$ is then a filtration of $N$, which we call the \emph{induced filtration of $N$ by $\Gamma$}. The inclusions $\Gamma_i' \subseteq \Gamma_i$ give us an inclusion $\gr^{\Gamma'} N \subseteq \gr^\Gamma M$, and since $\gr^\Gamma M$ is Noetherian, $\gr^{\Gamma'}N$ must be a finitely generated as an $S$-module.

	Let $\{c_1,...,c_r\}$ be a generating set for $\gr^{\Gamma'}M$. We assume that each $c_i$ is homogeneous without loss of generality; each $c_i$ is a linear sum of finitely many homogeneous elements and we can therefore replace each $c_i$ by its homogeneous components without compromising the finiteness of our generating set. For each $c_i$ we can therefore find some integer $k_i$ and some $u_i \in \Gamma'_{k_i}$ such that $\mu_{k_i}(u_i) = c_i$. Let $m = \max\{k_1,...,k_r\}$, and note that $u_i \in \Gamma'_m$ for each $1\leq i\leq r$. We show that $\Gamma'_m$ generates $N$.

    Suppose $v \in \Gamma_\ell$. If $\ell \leq m$ then $v \in \Gamma'_\ell \subseteq \Gamma'_m$, and hence $v$ is in the $R$-submodule of $M$ generated by $\Gamma'_m$. Suppose now that $\ell > m$ and $\Gamma_{\ell - 1}$ is contained in the $R$-linear span of $\Gamma'_m$. Because $\{\mu_{k_1}(u_1),...,\mu_{k_r}(u_r)\}$ generates $\gr^{\Gamma'}N$ as an $S$-module, there exist $a_1,...,a_r$ such that
	\begin{align*}
		\mu_\ell(v) = \sum_{i=1}^r \sigma_{\ell-k_i}(a_i)\mu_{k_i}(u_i).
	\end{align*}
	Hence
	\begin{align*}
		\mu_\ell \left(v - \sum_{i=1}^r a_i u_i\right) = 0
	\end{align*}
	\begin{align*}
		v' = v - \sum_{i=1}^r a_i u_i \in \Gamma'_{\ell - 1}.
	\end{align*}
	The element $v$ is a linear sum of elements in $\Gamma'_m$ if and only if $v'$ is too. However, $v' \in \Gamma'_{\ell - 1}$ and is therefore in the $R$-linear span of $\Gamma'_{m}$ by the inductive hypothesis. Hence $v \in R\cdot \Gamma'_m$, and since every element of $N$ is contained in $\Gamma'_\ell$, $\Gamma'_m$ generates $N$.

	It is left to show that there is a finite subset of $\Gamma'_m$ which generates $N$. However, $\Gamma'_m$ is a finite dimensional $K$-vector space. Any $K$-basis for $\Gamma'_m$ will generate all of $\Gamma'_m$ and will therefore serve as a set of generators for $N$.
\end{prf}
Note that the set $\{u_1,...,u_r\}$ in the above proof is not necessarily a generating set for $N$. The induction step gives us an algorithm for writing any $v \in \Gamma'_\ell$ in terms of the $u_i$ only in the case that $\ell > \max\{\deg(u_1),...,\deg(u_r)\}$.

The converse of this theorem need not always hold, that is, if $R$ is Noetherian and $M$ is finitely generated, it need not be the case that $\gr^{\Gamma}M$ is finitely generated. We therefore give filtrations which produce finitely generated associated graded modules a special name: we call $\Gamma$ a \emph{good filtration} of $M$ if $\gr^\Gamma M$ is finitely generated. Good filtrations provide a framework to discuss the dimension of modules over the Weyl algebra.

\subsection{Dimension and Holonomic Modules}

\begin{defn}\label{defn:holonomic}
	A finitely generated $A_n$ module $M$ is said to be \emph{holonomic} if $M = 0$ or if $d(M)$ is minimal, i.e. $d(M) = n$ by Bernstein's lemma.
\end{defn}

\subsection{Inverse Image and Direct Image Functors of Modules over the Weyl Algebra}
\red{Sort out inverse image vs. pullback and direct image vs. pushforward language}
Here we introduce two operations on $D$-modules, pullbacks and pushforwards. We discuss the former first. Some algebraic constructions are brushed under the rug, the tensor product of bimodules for instance, and the reader may wish to refer to \cite{d-mod-primer}, \cite{d-mod_ps_rt} or \cite{ginzburg_d-mod} throughout this section.

Let $R = K[x_1,...,x_n]$, $S = K[y_1,...,y_m]$ and $M$ be a left $A_n$-module. We wish to build a left $A_m$-module from $M$ in a meaningful way. Suppose further that we are given a polynomial map $F:K^m\to K^n$ whose coordinate polynomials are $F_1,...,F_n\in K[y_1,...,y_m]$. Composition with $F$ gives a map of rings $\varphi:R\to S$ defined $f\mapsto f\circ F$, and in particular, we may view $S$ as an $R$-module via the $R$-action given by $\varphi$. The tensor product $S\otimes_R M$ is then an $S$-module denoted $F^*M$, and we want to turn it into an $A_m$-module. The action of polynomials on $F^*M$ is known, and we define the action of $\partial_{y_i}$ by
\begin{equation}\label{eqn:action-inv-img}\tag{$\ast$}
	\partial_{y_i}(q\otimes u) = \partial_{y_i}(q) \otimes u ~ + ~ \sum_{j=1}^n q\partial_{y_i}(F_j) \otimes \partial_{x_j} u.
\end{equation}
for $q\in S$ and $u\in M$ and extend by linearity. As the $y_i$'s and $\partial_{y_i}$'s generate $A_m$, to check that this does indeed give us a left $A_m$-action on $F^*M$ we need only check that (\ref{eqn:action-inv-img}) is compatible with the relations
\begin{align*}
	[\partial_{y_i}, y_j] &= \delta_{ij} \\
	[y_i,y_j] &= [\partial_{y_i},\partial_{y_j}] = 0
\end{align*}
where $\delta_{ij}$ is the Dirac delta function.

Let us check the first relation on an arbitrary generator $q\otimes u \in F^*M$, referring to the image of $y_j \in S$ in $A_m$ by $\hat{y}_j$ to avoid confusion:

% Possible confusion due to the fact that $[\partial_{y_i}, \hat{y}_j] = \delta_{ij}$ in $A_m$ whereas $\partial_{y_i}(y_j) = \delta_{ij}$ in $S$.
\begin{align*}
	\partial_{y_i}\hat{y}_j(q\otimes u)
	  &= \partial_{y_i}(y_j q \otimes u) ~+~ \sum_{k=1}^n y_jq\partial_{y_i}(F_k) \otimes \partial_{x_k}u \\
	  &= \big(\partial_{y_i}(y_j q)\big)\otimes u ~+~ y_j \sum_{k=1}^n q \partial_{y_i}(F_k) \otimes \partial_{x_k}u \\
	  &= q(\partial_{y_i}y_j)\otimes u ~+~ y_j(\partial_{y_i}(q))\otimes u ~+~ y_j \sum_{k=1}^n q \partial_{y_i}(F_k) \otimes \partial_{x_k}u \\
	  &= q\delta_{ij} \otimes u ~+~ y_j\left(\partial_{y_j}(q) \otimes u ~+~ \sum_{k=1}^n q\partial_{y_i}(F_k) \otimes \partial_{x_k}u\right) \\
	  &= \delta_{ij}(q\otimes u) ~+~ y_j\partial_{y_i}(q\otimes u) \\
	  &= \delta_{ij}(q\otimes u) ~+~ \hat{y}_j\big(\partial_{y_i}(q\otimes u)\big).
\end{align*}
Since the relation $[\partial_{y_i},\hat{y}_j]v = \delta_{ij}v$ holds when $v$ is of the form $q\otimes u \in F^*M$, it holds for arbitrary elements $v\in F^*M$ once we extend by linearity. The other two relations follow from an easy check and a few applications of Leibniz's rule, respectively. \red{In reality, I simply don't want to write out the relation $[\partial_{y_i},\partial_{y_j}]v = 0$, but perhaps it would be helpful to include in a future draft.} This means the following definition makes sense.

\begin{defn}\label{defn:inv-img}
	Let $F:K^m\to K^n$ be a polynomial map with components $F_1,...,F_n \in K[y_1,...,y_m]$ and let $M$ be a left $A_n$-module. Then $F^*M$ endowed with the $A_m$-action defined in (\ref{eqn:action-inv-img}) is a left $A_m$-module called the \emph{inverse image} of $M$ by the polynomial map $F$.
\end{defn}

\begin{rmk}\label{rmk:inv-img}
	The term ``inverse image'' comes from algebraic geometry. If $X$ is a variety and $D_X$ is its sheaf of differential operators, then a sheaf $M$ over $X$ is said to be a left $D_X$-module if $M$ is a sheaf of left modules over $D_X$. Given a morphism $f:X\to Y$ of smooth algebraic varieties and a left $D_Y$-module $M$, then $f^*M$ is simply the sheaf-theoretic inverse image defined in \cite{hartshorne}.
\end{rmk}




\section{\emph{D}-Modules in Positive Characteristic}
In this section we'll discuss the theory of $D$-Modules in positive characteristic. 
\subsection{The Structure of Differential Operators in Positive Characteristic}

\begin{defn}\label{defn:diff-op-rings}
	Let $A\to R$ be a map of commutative rings and let $M$ and $N$ be (two-sided) $R$-modules. We define
	\begin{align*}
		D^0_{R/A}(M,N) = \Hom_R(M,N)
	\end{align*}
	and then inductively define
	\begin{align*}
		D^i_{R/A}(M,N) = \{\varphi \in \Hom_A(M,N) ~\mid~ [\varphi,f] \in D^{i-1}_{R/A}(M,N) ~ \text{for all} f \in R\}.
	\end{align*}
    Note that we may identify $R$ with its image in $\End_R(M)$ by the map $f \mapsto \olf_M$ where $\olf_M: m\mapsto f\cdot m$. By $[\varphi,f]$ we mean $\varphi\circ \olf_M - \olf_N\circ \varphi$.

	We let $D_{R/A}(M,N) = \bigcup_{i=0}^\infty D^{i}_{R/A}(M,N)$. In the case that $M = N$ we write $D_{R/A}(M)$ and when $R = M = N$ we write $D_{R/A}$, sometimes omitting $R$ and $A$ when the context is clear.
\end{defn}
It is often convenient to let $D^i_{R/A}(M,N) = 0$ for $i < 0$. In fact, since $\varphi$ is $R$-linear if and only if $[\varphi, f] = 0$ for each $f \in R$, we could instead declare $D^i_{R/A}(M,N) = 0$ before proceeding with the inductive definition above.

The following is our first theorem regarding differential operators in characteristic $p > 0$.

\begin{thm}[{\cite[Lemma 1.4.8]{amon92}}]\label{thm:perfect-field-diff-op}
	Let $K$ be a perfect field with $\fchar K = p > 0$ and $R$ be essentially of finite type as a $K$-algebra. Then
	\begin{align*}
		D_{R/K} = \bigcup_{e\in \bN} \Hom_{R^{p^e}}(R,R).
	\end{align*}
\end{thm}
\begin{prf}
	The proof of this statement in \cite{amon92} is slightly more general than this statement, likely opportunity to condense it.
\end{prf}

\begin{defn}\label{defn:D-ideal}
	Let $A\to R$. A \emph{$D$-ideal} of $R$ is a $D_{R/A}$-submodule of $R$. Since $R \hookrightarrow D_{R/A}$, any such submodule is closed under multiplication by $R$ and is therefore an ideal of $R$, justifying the name.
\end{defn}

\begin{example}\label{ex:quotient-is-d-mod}
	If $I \subseteq R$ is a $D$-ideal, then $R/I$ is a $D-module$.
\end{example}

\begin{prop}\label{prop:localization-of-D-mod}
	If $W \subseteq R$ is a multiplicatively closed set and $M$ is a $D$-module, then $W^{-1}M$ is a $D$-module by the rule
	\begin{align*}
		\alpha\cdot \frac{m}{\omega} = \sum_{i = 0}^{\ord(\alpha)} \frac{\alpha^{(i)}\cdot m}{\omega^{i+1}}
	\end{align*}
	where $\alpha^{(0)} = \alpha$ and $\alpha^{(i+1)} = [\alpha^{(i)}, \ol{\omega}]$.
\end{prop}

\begin{defn}\label{defn:D-module-simple}
	Let $A\to R$ be a map of rings. We say that $R$ is \emph{$D$-module simple} if it is a simple $D$-module. We say $R$ is \emph{$D$-algebra simple} if it is a simple $D$-algebra.
\end{defn}

The ring of differential operators in prime characteristic can detect singularities. 

\begin{thm}[{\cite[Theorem 2.2 (4)]{ksmith95-d-mod-f-split}}]\label{thm:strongly-F-regular-rings-D-simple}
	Let $\fchar(R) = p > 0$ and suppose $R$ is $F$-finite. Then $R$ is strongly $F$-regular if and only if $R$ if $F$-split and is a finite product of $D$-simple rings.
\end{thm}

\newpage
\printbibliography
\end{document}
