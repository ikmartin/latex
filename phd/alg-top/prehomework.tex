\documentclass{amsart}
\theoremstyle{remark}
\newtheorem{exercise}{Problem}

\def\defn#1{\emph{#1}}
\def\R{\mathbb{R}}
\def\sset{\subseteq}
\def\iso{\cong}
\def\gend#1{\langle #1\rangle}

\begin{document}

HW 0 for M382C (Algebraic Topology) 

Daniel Allcock, Fall 2022, U.T. Austin

{\bf Due Friday August 26 at the beginning of class.}   

This homework involves non-algebraic-topology things that will
be useful/necessary in the course.  Alg Top homework, mostly from Hatcher,
will begin next week.

\begin{exercise}
    Suppose you take the disjoint union of
    finitely many triangles, and identify their edges
    in pairs, by some affine homeomorphisms.  Prove that the result is a manifold.

    (Following Hatcher, an $n$-dimensional 
\defn{manifold}
means a Hausdorff topological space, each of whose
points has an open neighborhood that is homeomorphic to $\R^n$.
Many other sources include the extra technical hypothesis
of \defn{paracompactness}.)
\end{exercise}

\begin{exercise}
    Suppose a finite group $G$ acts on a manifold~$M$.  Suppose
    the action is \defn{free}, meaning that only the identity
    element has any fixed points.  Then the orbit space $M/G$ is
    also a manifold.  (``Lying in the same $G$-orbit'' is an
    equivalence relation on~$M$.  $M/G$ means the set of equivalence
    classes.  The topology on $M$ induces one on $M/G$, which
    is the one you must work with.)
\end{exercise}

\begin{exercise}
    If freeness is dropped in the previous problem, then $M/G$
    may or may not be manifold.  
\end{exercise}

\begin{exercise}
    Let $A$ be the abelian group with generators $a,b,c,d$ and
    relations $3a+2b=4c$, $2a=4b$, $7a+7d=0$ and $22c=0$.  
    Describe the structure of $A$ as a sum of cyclic groups.
    Find the unique involution in~$A$.
\end{exercise}

\begin{exercise}
    Let $p,q$ be positive integers and
    $G$ be the group with presentation $\gend{x,y|x^p=y^q}$.
    Show that the center $Z$ of $G$ is nontrivial, and that the
    quotient $G/Z$ has trivial center.
\end{exercise}

\begin{exercise}
    Consider the group $G=\gend{a,b,c|a^b=a^2, b^c=b^3, c^2=1}$.
    Understand $G$.  (For example, is $G$ trivial?  finite?  infinite?
    Is it a semidirect product of two easier-to-understand groups?)

    (The notation $a^b$ means the conjugate of $a$ by $b$.  Different
    authors use different conventions for whether this means $bab^{-1}$
    or $b^{-1}ab$, so you have to be careful.  Most topologists, 
    including Hatcher, use the $bab^{-1}$
    definition.)
\end{exercise}

% The rest of the exercises concern cell complexes; see the appendix
% in Hatcher.
% 
% \begin{exercise}
%     Suppose $X$ is 
%     a cell complex.  Then the following are equivalent.
%     \begin{enumerate}
%         \item
%             $X$ is connected.
%         \item
%             $X$ is path-connected.
%         \item
%             The $1$-skeleton of $X$ is connected.
%     \end{enumerate}
% \end{exercise}
% 
% \begin{exercise}
%     Suppose $X$ is a cell complex.  Then $X$ is locally compact 
%     if and only if each point of $X$ has a neighborhood which meets
%     only finitely many cells.
% \end{exercise}
% 
% \begin{exercise}
%     There exists a cell complex~$X$ which is not
%     first countable.  In particular, the topology on~$X$ cannot
%     be induced by any metric on~$X$.  (Hint: you can take $X$
%     to be $1$-dimensional.)
% \end{exercise}
% 
% \begin{exercise}
%     Suppose $X$ is a cell complex and $S\sset X$ is compact.
%     Then $S$ lies in the union of finitely many cells.
%     (This is proven in the appendix, but don't look.)
% \end{exercise}
% 
% \begin{exercise}
%     Suppose $f$ is a function from a cell complex $X$ to a topological
%     space~$Y$.  Prove that $f$ is continuous if and only if
%     its restriction to each cell is continuous.  (A little care
%     is required to make this precise.  The cell complex
%     structure includes, for each $n$-cell $\alpha$, a continuous
%     function from the $n$-ball into~$X$.  This is called
%     the characteristic map of~$\alpha$.  The ``restriction''
%     of $f$ to~$\alpha$ means the composition $B^n\to X\to Y$.)
% \end{exercise}
 
%\begin{exercise}[Important: composing infinitely many homotopies]
%    Suppose $X$ is an ascending union of spaces $X_0\sset X_1\sset\cdots$,
%    and that for each $i>0$ there is a homotopy $X\to X$ whose restriction
%    to $X_i$ is a deformation retraction to $X_{i-1}$.  Show that $X$
%    deformation retracts to $X_0$.
%\end{exercise}

\begin{exercise}
    Suppose $U$ is an open subset of $\R^n$.  Show that $U$ has the structure
    of a locally finite simplicial complex.  

    We haven't formally met simplicial complexes, and
    won't for a while.  You don't need to read up on them in order to solve this problem.
    For purposes of this problem, take the definition to mean
    an expression of $U$ into a union of $n$-simplices, such that any two that meet, have
    intersection equal to a common face (of some dimension).  Locally finite means that
    every point has a neighborhood meeting only finitely many simplices.
\end{exercise}

\end{document}
